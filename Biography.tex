%!TEX root =  main.tex
\chapter*{Biography}
\pagenumbering{gobble}
%\pagestyle{empty}
%
The work in this thesis is synthesis of few individual parts:

\begin{enumerate}[start=1,label={(\bfseries \arabic*)}]
	\item The experience acquired on a university, on the topic of software engineering,
	\item The research conducted as part of the PhD studies, covering various aspects of the distributed systems,
	\item The work done in collaboration with prominent software vendors,
	\item The collaboration with researchers from different research areas.
\end{enumerate}

\noindent
Milo\v s Simi\'c is a Ph.D. student and teaching assistant within the Department of Computing and Control, Faculty of Technical Sciences, University of Novi Sad since 2015. He received his B.Sc. degree in 2014, and M.Sc. degree in 2015, all in Computer Science from the University of Novi Sad, Faculty of Technical Sciences. He is owner of two team awards: $(1)$ Best paper award (academia), and $(2)$ ThinkX in the category Community and Social Impact (industry).\\\\ 
\noindent
Over the years, Milo\v s worked with various prominent software vendors, in different fields. This allowed him combining the different skillsets developed over the years, to focuse his expertise towards designing and implemennting distributed and software systems, for various usages. His research interests include: $(1)$ distributed systems, $(2)$ (multi) cloud computing, $(3)$ edge computing, $(4)$ big data and $(5)$ service oriented architectures and microservices.\\\\
\noindent
As part his Ph.D., Milo\v s have studied the different distributed systems techniques, combined with various software engineering methodologies and practices, covering both standard-defined processes and industry-proven methods, to resolve and answer such complicated questions that are part of this thesis. Working with different softvare vendors, combined with traditional academic research, helped Milo\v s to clear his PhD vision, and guid him to the work that is described in this thesis.\\\\
\noindent
Trough colaboration with people from different research areas, this theis gain forml description and formal model that is important leverage, to describe and validate such complicated system that is described in this thesis.
%
%