%!TEX root =  main.tex
\chapter*{Biography}
\pagenumbering{gobble}
%\pagestyle{empty}
%
The work in this thesis is a synthesis of a few individual parts:

\begin{enumerate}[start=1,label={(\bfseries \arabic*)}]
	\item The experience acquired at a university on the topic of software engineering,
	\item The research conducted as part of the Ph.D. studies, covering various aspects of the distributed systems,
	\item The work done in collaboration with prominent software vendors,
	\item The collaboration with researchers from different research areas.
\end{enumerate}

\noindent
Milo\v s Simi\'c is a Ph.D. student and teaching assistant within the Department of Computing and Control, Faculty of Technical Sciences,  University of Novi Sad since 2015. He received his B.Sc. degree in 2014, and M.Sc. degree in 2015, all in Computer Science from the University of Novi Sad, Faculty of Technical Sciences. He is the owner of two team awards: \textbf{(1)} Best paper award \emph{(academia)}, and \textbf{(2)} ThinkX in the category Community and Social Impact \emph{(industry)}.\\\\
\noindent
Over the years, Milo\v s worked with various prominent software vendors, in different fields. This allowed him to combine the different skillsets developed over the years and focus his expertise towards designing and implementing distributed and non-distributed software systems, for various usages. His research interests include: \textbf{(1)} distributed systems, \textbf{(2)} (multi) cloud computing, \textbf{(3)} edge computing, \textbf{(4)} NoSQL engines and big data, and \textbf{(5)} service-oriented architectures and microservices.\\\\
\noindent
As part of his Ph.D., Milo\v s has studied different distributed systems techniques, combined with various software engineering methodologies and practices covering both standard-defined processes and industry-proven methods, to solve and answer such complicated questions that are part of this thesis. The chance to work with different software vendors and combine that knowledge with traditional academic approaches, helped Milo\v s to determine what are the main research questions that need to be answered and guided him to the work that is described in this thesis.\\\\
\noindent
Through collaboration with people from different research areas, this thesis is enriched with formal description and formal model that are important leverage to describe and validate such a complicated system. The efforts put into this research resulted in reaching only the tip of the iceberg of future opportunities. In 2021, the portion of the work that was published in the IEEE Access journal (paper cf. journal paper~\ref{accessSimic}) was presented to the eminent professors and colleagues from the \emph{Imperial College London}, as an invited lecture.\\\\

\noindent
Further challenges are yet to come.
%
%