%!TEX root =  main.tex


\section{Behavioral semantics}\label{sec:Bisimulation}

In Section~\ref{sec:Cpi-bisimilarity} we introduced a behavioral equivalence relation in $C_\pi$, strong bisimilarity, that is actually the same relation as in the $\pi$-calculus. 
This section presents a preliminary investigation of the behavioral semantics of our authorization model. We introduce the strong bisimilarity relation by relying on the labeled transition system given in the previous section. The definition of the strong bisimilarity here follows the same lines as Definition~\ref{def:Cpi-strong-bisimilarity}.
%Intuitively, the relation defined next, dubbed strong bisimulation, identifies processes that exhibit the same behavior, i.e., where each (observable) action of one process can be reproduced by the other process, and vice versa, leading again to equivalent processes.

%Intuitively, two processes behave in the same way if \comment{H: it seems you are introducing trace equivalence ;) you can have a look at it, but for now let us consider the bisimulation game...} each sequence of observable action of one process can be mimicked by the other, and vice versa. The formal definition follows.

\begin{definition}[Strong bisimilarity]
A binary relation $\cal R$ is a strong bisimulation on processes if $\cal R$ is a symmetric relation and for all $(\PP, \PQ)\in{\cal R}$ the following holds:

If $\PP\lts{\alpha}\PP'$, for some $\alpha$ and $\PP'$ where $\bn{\alpha}\cap\fn{\PQ}=\emptyset$, then $\PQ\lts{\alpha}\PQ'$ for some $\PQ'$ such that $(\PP',\PQ')\in{\cal R}$.

Strong bisimilarity, noted with $\sim$, is 
the union of all strong bisimulations. % is the strong bisimilarity relation, .
\end{definition}

Same as in the $\pi$-calculus (and in the $C_\pi$), the strong bisimilarity defined above is also an equivalence relation (cf. Proposition~\ref{prop:bisimilarity_is_equivalence}). The reflexive and symmetric property are direct from the definition and the transitive property can be shown, up to $\alpha$-conversion, in expected lines. %\del{, we have the next proposition.
%
%\begin{proposition}[Equivalence]
%Strong bisimilarity is an equivalence relation.
%\end{proposition}}\marginpar{H: I think we either only informally mention Proposition 1 or state it in which case we add the proof statement}
%
Our aim is to first show some standard properties for the strong bisimilarity defined in this section, as the ones given in Section~\ref{sec:Cpi-bisimilarity}.
The first result shows that the strong bisimilarity embeds the structural congruence.
\begin{proposition}[Structural congruence]\label{prop:struct.equiv_bis}
$\equiv \; \subseteq\; \sim$.
%If $\PP\equiv\PQ$ then $\PP\sim\PQ$.
\end{proposition}
\begin{proof}
%\comment{please announce the proof principle and add the relation explicitly}
The proof proceeds by coinduction on the definition of strong bisimulation, showing that relation 
\[
{\cal R}=\{(\PP, \PQ) \;|\; \PP\equiv\PQ\}
\]
is a strong bisimulation, hence ${\cal R}\subseteq\sim$.
Let $(\PP, \PQ)\in{\cal R}$. By Lemma~\ref{lemm:lts_struct_congruence} we have that if $\PP\equiv\PQ$ and $\PP\lts{\alpha}\PP',$ then there exists some $\PQ'$, such that $\PQ\lts{\alpha}\PQ'$ and $\PP'\equiv\PQ'$. Thus, $\cal R$ is a strong bisimulation relation.
\end{proof}

The last result attests that structurally congruent processes exhibit the same behavior and as such is a form of a sanity check of our reduction semantics. %For a further sanity check %on our model %operational semantics \comment{H: I would actually say model here rather than operational semantics, since the result relates syntax and semantics} 
%we may show that strong bisimilarity is preserved under context closure, considering a universal instantiation principle for input.
Next, we show another standard property, that the strong bisimilarity is a non-input congruence.

\begin{theorem}[Non-input congruence]\label{th:bisimlarity_is_a_congruence}
\leavevmode
\begin{itemize}
\item [(a)] If $\PP\sim\PQ$ then
\begin{enumerate}
\item $\PP\parop\PR \sim \PQ\parop\PR$
\item $\rest\NA\PP \sim \rest\NA\PQ$
\item $\send\NA\role\msg\NB.\PP \sim \send\NA\role\msg\NB.\PQ$ 
\item $\scope\NA\PP \sim \scope\NA\PQ$
\item $\sauth\NA\role\msg\NB.\PP \sim \sauth\NA\role\msg\NB.\PQ$
\item $\rauth\NA\role\msg\NB.\PP \sim \rauth\NA\role\msg\NB.\PQ$
\end{enumerate}
\item [(b)] If $\PP\subst{\NB}{\NX} \sim \PQ\subst{\NB}{\NX}$, for all $\NB$, then
\begin{enumerate}
\item $\receive\NA\role\msg\NX.\PP \sim \receive\NA\role\msg\NX.\PQ$
\item $\repreceive\NA\role\msg\NX.\PP \sim \repreceive\NA\role\msg\NX.\PQ$
\end{enumerate}
\end{itemize}
\end{theorem}

\begin{proof}
The proof is by coinduction on the definition of strong bisimulation.
\begin{itemize}
\item [(a)]
\begin{enumerate}
\item Follows by coinduction considering the relation
\[
{\cal R}=\{(\rest{\tilde{\NA}}(\PP\parop\PR), \rest{\tilde{\NA}} (\PQ\parop\PR)) \;|\; \PP\sim\PQ\}
\]
and showing that it is a strong bisimulation, i.e., ${\cal R}\subseteq \sim$. 
Let us assume that $(\rest{\tilde{\NA}}(\PP\parop\PR), \rest{\tilde{\NA}} (\PQ\parop\PR))\in{\cal R}$ and  
\begin{equation}\label{eq:congruence_parallel}
\rest{\tilde{\NA}}(\PP\parop\PR)\lts{\alpha} \PP_1
\end{equation} 
for some $\PP_1$ and $\alpha$, where $\bn{\alpha}\cap\fn{\rest{\tilde{\NA}} (\PQ\parop\PR)}=\emptyset$. 
We show that then there is some $\PQ_1$ such that $\rest{\tilde{\NA}} (\PQ\parop\PR)\lts{\alpha}\PQ_1$ and $(\PP_1,\PQ_1)\in{\cal R}$. The proof is analogous when first considering an action of $\rest{\tilde{\NA}}(\PQ\parop\PR)$. %\lts{\alpha} \PQ_1$ and $\bn{\alpha}\cap\fn{\rest{\tilde{\NA}} (\PP\parop\PR)}=\emptyset$ then $\rest{\tilde{\NA}}(\PP\parop\PR)\lts{\alpha} \PP_1$ where again $(\PP_1,\PQ_1)\in{\cal R}$. 
We distinguish three cases for deriving~(\ref{eq:congruence_parallel}): either it is derived from the observation on $\PP$ or on $\PR$ or from the synchronization between $\PP$ and $\PR$.
\\
\emph{-Observation on} $\PP:$
In this case we have
\[
\rest{\tilde{\NA}}(\PP\parop\PR)\lts{\alpha}\rest{\tilde{\NA}'}(\PP'\parop\PR)
\]
is derived from 
\[
\PP\lts{\alpha'}\PP'
\]
where either $\alpha=\alpha'$ and $\tilde{\NA}=\tilde{\NA}'$, or $\alpha={\scope\NB^i}\rest\NC\send\NB\role\msg\NC$ and $\alpha'={\scope\NB^i}\send\NB\role\msg\NC$ and $\tilde{\NA}=\tilde{\NA}',{\NC}$. By $\bn{\alpha'}\cap\fn{\PQ}=\emptyset$ and $\PP\sim\PQ$ we conclude that $\PQ\lts{\alpha'}\PQ'$ where $\PP'\sim\PQ'$. Thus, we can derive 
\[
\rest{\tilde{\NA}}(\PQ\parop\PR)\lts{\alpha}\rest{\tilde{\NA}'}(\PQ'\parop\PR)
\]
and we conclude $(\rest{\tilde{\NA}'}(\PP'\parop\PR), \rest{\tilde{\NA}'}(\PQ'\parop\PR))\in{\cal R}$.
\\
\emph{-Observation on} $\PR:$ 
%\comment{also the following case has a symmetric}\\
%\comment{I: there is a note for the symmetric case for l-close rule. If you want I can make it more explicit.}\\
%\comment{H: I'm not sure I'm following... I was thinking of adding ``Synchronization between $\PQ$ and $\PR$'' (together with ``follows similar lines'') at the end}
Follows similar lines.
\\
\emph{-Synchronization between} $\PP$ \emph{and} $\PR:$ 
\\
We give only the case when~(\ref{eq:congruence_parallel}) is derived by \rulename{(l-close)}, since other cases, including the symmetric case for \rulename{(l-close)}, follow similar lines. Consider
\[
\rest{\tilde{\NA}}(\PP\parop\PR)\lts{\tau_\omega}\rest{\tilde{\NA}}\rest\NB(\PP'\parop\PR')
\] 
is derived from 
\[
\PP\lts{\rest\NB\sigma_1}\PP'
\qquad \mbox{and} \qquad
\PR\lts{\overline{\sigma}_1}\PR'
\]
Since without loss of generality we can assume $\NB\notin\fn\PQ$, by $\PP\sim\PQ$ we have $\PQ\lts{\rest\NB\sigma_1}\PQ'$ and $\PP'\sim\PQ'$. Thus, 
\[
\rest{\tilde{\NA}}(\PQ\parop\PR)\lts{\tau_\omega}\rest{\tilde{\NA}}\rest\NB(\PQ'\parop\PR')
\] 
and we conclude $(\rest{\tilde{\NA}}\rest\NB(\PP'\parop\PR'), \rest{\tilde{\NA}}\rest\NB(\PQ'\parop\PR'))\in {\cal R}$.

\item Follows similar lines, by showing that relation 
\[
{\cal R}=\{(\rest\NA\PP \sim \rest\NA\PQ) \;|\; \PP\sim \PQ\}\;\cup\; \sim
\]
is contained in $\sim$ by conduction on the definition of strong bisimulation. 
\item Follows by showing that relation 
\[
{\cal R}=\{(\send\NA\role\msg\NB.\PP , \send\NA\role\msg\NB.\PQ)\;|\; \PP\sim\PQ \}\; \cup\; \sim
\]
is contained in $\sim$ by coinduction on the definition of strong bisimulation. Notice that the pair of processes related by $\cal R$ are either in $\sim$, in which case we conclude the proof directly, or have only the same observable (the output), in which case the pair of continuing processes is in $\sim$. 
\item Follows by showing that relation
\[
{\cal R}=\{(\scope\NA\PP,\scope\NA\PQ) \;|\; \PP\sim\PQ \}\;\cup\; \sim
\]
is contained in $\sim$ by coinduction on the definition of strong bisimulation. 
Let $(\scope\NA\PP,\scope\NA\PQ)\in{\cal R}$. and let 
$$\scope\NA\PP\lts{\alpha}\PP'.$$
There are three cases for the last applied rule while deriving the latter. We detail only the case when the last applied rule is \rulename{(l-scope-int)}. Then, $\alpha=\tau_\omega$ and  $\scope\NA\PP\lts{\tau_\omega}\PP'$ is derived from $\PP\lts{\tau_{\omega\scope\NA}}\PP'$. By $\PP\sim\PQ$ we have that $\PQ\lts{\tau_{\omega\scope\NA}}\PQ'$, where $\PP'\sim\PQ'$. By \rulename{(l-scope-int)} we have $\scope\NA\PQ\lts{\tau_\NA}\PQ'$, which finishes the proof.
\item Follows by showing that relation 
\[
{\cal R}=\{(\sauth\NA\role\msg\NB.\PP , \sauth\NA\role\msg\NB.\PQ) \;|\; \PP\sim\PQ \} \;\cup\; \sim
\]
is contained in $\sim$ by coinduction on the definition of strong bisimulation. Note that both process $\sauth\NA\role\msg\NB.\PP$ and $\sauth\NA\role\msg\NB.\PQ$ have only one observable action leading them to processes $\scope\NA\PP$ and $\scope\NA\PQ$, which are bisimilar by $\PP\sim\PQ$ and statement $4.$ of this Theorem.
\item Follows similar lines as $5.$, with witnessing relation 
\[
{\cal R}=\{(\rauth\NA\role\msg\NB.\PP , \rauth\NA\role\msg\NB.\PQ) \;|\; \PP\sim\PQ \} \;\cup\; \sim
\]
\end{enumerate}
\item [(b)]
\begin{enumerate}
\item Follows the similar lines as $3.$, considering relation 
\[
{\cal R}=\{(\receive\NA\role\msg\NX.\PP, \receive\NA\role\msg\NX.\PQ) \;|\: (\forall \NB) \PP\subst{\NB}{\NX} \sim \PQ\subst{\NB}{\NX}\}\;\cup\; \sim
\]
Notice that if $(\receive\NA\role\msg\NX.\PP , \receive\NA\role\msg\NX.\PQ)\in{\cal R}$ the two processes have only input observable. Performing the same input leads to processes $\PP\subst{\NB}{\NX}$ and $ \PQ\subst{\NB}{\NX}$, for some $\NB$, that are bisimilar by the assumption.
\item Follows by showing that relation  
\[
{\cal R}=\{(\PR \parop \repreceive\NA\role\msg\NX.\PP, \PS \parop \repreceive\NA\role\msg\NX.\PQ) \;|\; \PR\sim\PS \;\wedge\; (\forall \NB) \PP\subst{\NB}{\NX} \sim \PQ\subst{\NB}{\NX}\}
\]
is a strong bisimulation, i.e., ${\cal R}\subseteq \sim$. Let $(\PR \parop \repreceive\NA\role\msg\NX.\PP, \PS \parop \repreceive\NA\role\msg\NX.\PQ)\in {\cal R}$ and let 
\begin{equation}\label{eq:congruence_rep_inp}
\PR \parop \repreceive\NA\role\msg\NX.\PP\lts{\alpha}\PP'
\end{equation}
for some $\PP'$ and $\alpha$ such that $\bn{\alpha}\cap\fn{\PS \parop \repreceive\NA\role\msg\NX.\PQ}=\emptyset$. We distinguish three cases for deriving~(\ref{eq:congruence_rep_inp}): either it is derived from the observation on $\PR$ or on $\repreceive\NA\role\msg\NX.\PP$ or from the synchronization between $\PR$ and $\repreceive\NA\role\msg\NX.\PP$.
\\
-\emph{Observation on} $\PR:$
\[
\PR \parop \repreceive\NA\role\msg\NX.\PP\lts{\alpha}\PR' \parop \repreceive\NA\role\msg\NX.\PP
\]
is derived from $\PR\lts{\alpha}\PR'$. Since $\bn{\alpha}\cap\fn{\PS}=\emptyset$ and $\PR\sim\PS$ it follows that $\PS\lts{\alpha}\PS'$, where $\PR'\sim\PS'$. By $\bn{\alpha}\cap\fn{\repreceive\NA\role\msg\NX.\PQ}=\emptyset$ we conclude
\[
\PS \parop \repreceive\NA\role\msg\NX.\PQ\lts{\alpha}\PS' \parop \repreceive\NA\role\msg\NX.\PQ
\]
and $(\PR' \parop \repreceive\NA\role\msg\NX.\PP,\PS' \parop \repreceive\NA\role\msg\NX.\PQ)\in{\cal R}$.
\\
-\emph{Observation on} $\repreceive\NA\role\msg\NX.\PP:$
\[
\PR \parop \repreceive\NA\role\msg\NX.\PP\lts{\receive\NA\role\msg\NB}\PR \parop \scope\NA\PP\subst{\NB}{\NX} \parop \repreceive\NA\role\msg\NX.\PP
\]
is derived from $\repreceive\NA\role\msg\NX.\PP\lts{\receive\NA\role\msg\NB} \scope\NA\PP\subst{\NB}{\NX} \parop \repreceive\NA\role\msg\NX.\PP$. Then, we can also derive 
\[
\PS \parop \repreceive\NA\role\msg\NX.\PQ\lts{\receive\NA\role\msg\NB}\PS \parop \scope\NA\PQ\subst{\NB}{\NX} \parop \repreceive\NA\role\msg\NX.\PQ
\]
and we only have to show that $\PR \parop \scope\NA\PP\subst{\NB}{\NX}\sim \PS \parop \scope\NA\PQ\subst{\NB}{\NX}$. 
Since $\PP\subst{\NB}{\NX}\sim\PQ\subst{\NB}{\NX}$ for any $\NB$, by statement $(a)\, 4.$ of this Theorem, 
we have $\scope\NA\PP\subst{\NB}{\NX}\sim\scope\NA\PQ\subst{\NB}{\NX}$. 
Using $\PR\sim\PS$, statement $(a)\,1.$ of this Theorem and transitivity and commutativity of strong bisimilarity 
we conclude $\PR \parop \scope\NA\PP\subst{\NB}{\NX}\sim \PS \parop \scope\NA\PQ\subst{\NB}{\NX}$.
\\
-\emph{Synchronization between} $\PR$ \emph{and} $\repreceive\NA\role\msg\NX.\PP:$
\\
We will detail only the case when the synchronization is derived by rule 
\rulename{(l-comm)}. In this case 
\[
\PR \parop \repreceive\NA\role\msg\NX.\PP \lts{\tau_\omega} \PR' | \scope\NA\PP\subst{\NB}{\NX} \parop \repreceive\NA\role\msg\NX.\PP 
\]
is derived from 
\[
\PR\lts{{\scope\NA^i}\send\NA\role\msg\NB}\PR' \quad \mbox{and} \quad \repreceive\NA\role\msg\NX.\PP\lts{\receive\NA\role\msg\NB}\scope\NA\PP\subst{\NB}{\NX} \parop \repreceive\NA\role\msg\NX.\PP 
\]
%Since without loss of generality we can assume $\NB\notin\fn{\PS \parop \repreceive\NA\role\msg\NX.\PQ}$, 
By $\PR\sim\PS$ we derive $\PS\lts{{\scope\NA^i}\send\NA\role\msg\NB}\PS'$, where $\PR'\sim\PS'$.
Then, also
\[
\PS \parop \repreceive\NA\role\msg\NX.\PQ \lts{\tau_\omega} \PS' | \scope\NA\PQ\subst{\NB}{\NX} \parop \repreceive\NA\role\msg\NX.\PQ.
\]
Similarly as in the previous case, we can show that $\PR' | \scope\NA\PP\subst{\NB}{\NX}\sim \PS' | \scope\NA\PQ\subst{\NB}{\NX}$. %and by statement $2.$ of this Theorem we can derive $\rest\NB(\PR' | \scope\NA\PP\subst{\NB}{\NX})\sim \rest\NB(\PS' | \scope\NA\PQ\subst{\NB}{\NX})$, which finishes the proof.

\end{enumerate}
\end{itemize}
\end{proof}


Theorem~\ref{th:bisimlarity_is_a_congruence} asserts that a computational context cannot distinguish between the behaviors of the bisimilar processes, as placing two bisimilar processes in the same context results in two processes that are also bisimilar. Thus, each language construct can be seen as a proper function of behavior, as their composition with equivalent (object) behaviors yields equivalent (image) behaviors.

As we noted in Section~\ref{subsec:reduction}, our accounting principle makes it difficult to manipulate authorization scope construct over the parallel composition. Using the strong bisimilarity relation we can formalize these intuitions.
In the following, by $\PP  \not\sim \PQ$ we denote that $\PP$ and $\PQ$ are not bisimilar.

%\comment{H: 4 and 6 are nice :)}

\begin{proposition}[Behavioral inequalities]\label{prop:behavior_auth_and?parallel}
For each of the following inequalities there exist processes $\PP$ and $\PQ$ and name $\NA$ %such
 that witness them. %the following inequalities hold:
\begin{enumerate}
\item $\scope\NA(\PP\parop\PQ) \not\sim \scope\NA\PP\parop\scope\NA\PQ$.
\item $\scope\NA\scope\NA(\PP\parop\PQ) \not\sim \scope\NA\PP\parop\scope\NA\PQ$.
%\item $\scope\NA(\PP\parop\PQ)\not\sim \PP\parop\scope\NA\PQ$,
\item $\scope\NA(\PP\parop\PQ)\not\sim \PP\parop\scope\NA\PQ$ if $\NA\notin\fn\PP$.
\item $\scope\NA\scope\NA\PP \not\sim \scope\NA\PP$.
\item $\scope\NA\PP \not\sim \PP$ if $\NA\notin\fn\PP$.
\end{enumerate}
\end{proposition}

\begin{proof} To prove each inequality we give a proper counter-example.
\begin{enumerate}
\item Consider processes $\PP=\send\NA\role\msg\NB.0$ and $\PQ=\receive\NA\role\msg\NX.0$. Then, process $\scope\NA\PP\parop\scope\NA\PQ$ has $\tau$-transition 
\[
\scope\NA\send\NA\role\msg\NB.0\parop\scope\NA\receive\NA\role\msg\NX.0 \lts{\tau} \scope\NA 0 \parop \scope\NA 0
\]
while process $\scope\NA(\PP\parop\PQ)$ has only $\tau$-transition with pending authorization
\[
\scope\NA(\send\NA\role\msg\NB.0\parop\receive\NA\role\msg\NX.0) \lts{\tau_{\scope\NA}} \scope\NA 0 \parop \scope\NA 0
\]
\item Consider processes $\PP=\sauth\NA\role\msg\NA.0$ and $\PQ=0$. We have that 
\[
\scope\NA\scope\NA(\PP\parop\PQ)\lts{\sauth\NA\role\msg\NA} \scope\NA 0 \parop 0
\]
while the only possible action for $\scope\NA\PP\parop\scope\NA\PQ$ is
\[
\scope\NA\PP\parop\scope\NA\PQ\lts{\scope\NA\sauth\NA\role\msg\NA} \scope\NA 0 \parop \scope\NA 0
\]
%\item Consider again $\PP=\sauth\NA\role\msg\NA.0$ and $\PQ=0$. Then 
%\[
%\scope\NA(\PP\parop\PQ)\lts{\scope\NA\sauth\NA\role\msg\NA} \scope\NA 0 \parop 0,
%\]
%while the only possible action for $\PP\parop\scope\NA\PQ$ is
%\[
%\PP\parop\scope\NA\PQ\lts{\sauth\NA\role\msg\NA} \scope\NA 0 \parop \scope\NA 0.
%\]
\item Consider processes $\PP=\receive\NB\role\msg\NX.\send\NX\role\msg\NC.0$ and $\PQ=0$. Then, we have 
\[
\scope\NA(\PP\parop\PQ)\lts{{\scope\NB}\receive\NB\role\msg\NA} \scope\NA (\scope\NB\send\NA\role\msg\NC.0 \parop 0) \lts{\send\NA\role\msg\NC} \scope\NB\scope\NA 0 \parop 0
\]
while the only possible action for $\PP\parop\scope\NA\PQ$ after receiving $\NA$ on $\NB$ is pending on authorization $\scope\NA$
\[
\PP\parop\scope\NA\PQ\lts{{\scope\NB}\receive\NB\role\msg\NA} \scope\NB\send\NA\role\msg\NC.0  \parop \scope\NA 0 \lts{{\scope\NA}\send\NA\role\msg\NC} \scope\NB\scope\NA 0 \parop \scope\NA 0
\]
\item Consider $\PP=\sauth\NA\role\msg\NA.0$. We may observe
\[
\scope\NA\scope\NA\PP\lts{\sauth\NA\role\msg\NA} \scope\NA 0 \quad\mbox{while}\quad \scope\NA\PP\lts{\scope\NA\sauth\NA\role\msg\NA} \scope\NA 0
\]
\item Consider again process $\PP=\receive\NB\role\msg\NX.\send\NX\role\msg\NC.0$. Then, 
\[
\scope\NA\PP\lts{{\scope\NB}\receive\NB\role\msg\NA} \scope\NA \scope\NB\send\NA\role\msg\NC.0 \lts{\send\NA\role\msg\NC} \scope\NB\scope\NA 0
\]
On the other hand 
\[
\PP\lts{{\scope\NB}\receive\NB\role\msg\NA}  \scope\NB\send\NA\role\msg\NC.0 \lts{{\scope\NA}\send\NA\role\msg\NC} \scope\NB\scope\NA 0
\]
\end{enumerate}
\end{proof}


The last proposition formally attests our reduction semantics design choices.
%\marginpar{H: Please check undefined and multiply defined references (see compilation log)},
The results given in $1.$ and $3.$ show that the structural congruence rules that we do not adopt (that relate authorization scoping and parallel composition) as mentioned in Section~\ref{subsec:reduction} indeed are  unsound. %ness of the structural congruence laws mentioned previously  that relate authorization scoping and parallel composition constructs.  
Regardless if an authorization is distributed to both or to a single branch of the parallel composition the obtained process may exhibit a different behavior. Even if the name specified in the authorization is not free in one of the branches, the mentioned distribution may affect the behavior of a process.
We remark that $\scope\NA(\PP\parop\PQ)\not\sim \PP\parop\scope\NA\PQ$ given in $3.$ also hold when $\NA\in\fn\PP$ (e.g., $\scope\NA(\send\NA\role\msg\NB.\inact\parop\inact)\not\sim \send\NA\role\msg\NB.\inact\parop\scope\NA\inact$). Statement $1.$ provides the main evidence that the structural congruence rule $\scope\NA(\PP\parop\PQ) \equiv \scope\NA\PP\parop\scope\NA\PQ$ introduced in the operational semantics of~\cite{clar:eke} conflicts with our design choices, namely with our accounting principle. 
Notice also that the difference of authorization scoping and name restriction is exposed in $3.$, showing that scope extrusion of the first operator is not safe since the name specified in the authorization is free while the name restriction is a binder.

Statement $2.$ attests that even the symmetric distribution of two authorizations over parallel 
composition may change the process behavior.
Our accounting authorizations principle is attested in $4.$: providing a different number of the same authorization to a process can yield processes that exhibit different behaviors.
Similarly to $3.$, statement $5.$ reflects the fact that the authorization is a non-binder construct. Therefore, as a result of name passing, authorizations for names not free in the process may eventually be used.


Next, we present several equations relating authorization scoping and active prefixes.
%We can also show some behavioral equations that inform on the relation between authorization scoping and active prefixes.

\begin{proposition}[Behavioral (in)equalities]\label{prop:behavior_auth_and_prefixes}
\leavevmode
\begin{enumerate}
%\item \comment{H: Thinking of $\scope\NA\scope\NB \PP \sim \scope\NB\scope\NA \PP$ I realized there is little point in including this principle in structural congruence... I remember some related discussions but I want to make sure that I'm not missing something.}\\
%\comment{I: I agree, the rule is not necessary for the reduction...}
%\item[]{}
\item For any process $\PP$, names $\NA,\NB$ and prefix $\alpha_\NB$ such that $\NB \neq \NA$ and $\alpha_\NB \neq \sauth\NB\role\msg\NA$
%$\notin\{\send\NA\role\msg\NB, \receive\NA\role\msg\NX, \repreceive\NB\role\msg\NX,\sauth\NA\role\msg\NB, \sauth\NB\role\msg\NA, 
%\rauth\NA\role\msg\NB\}$ 
we have that $\scope\NA\alpha_\NB.\PP \sim \alpha_\NB.\scope\NA\PP$.%\marginpar{H: I'm using $\alpha_{\NA}$ introduced previously since $\alpha$ is used in the LTS (observable actions), and I'm avoiding the use of $\n{}$ since it is defined for such observable actions}
\item There exist process $\PP$, name $\NA$ and prefixes $\alpha_{\NB},\alpha_{\NC}$, where $\NA$ does not occur, such that $\scope\NA\alpha_\NB.\alpha_\NC.\PP \not\sim \alpha_\NB.\alpha_\NC.\scope\NA\PP$. %, where $\NA\notin \n{\alpha, \beta}$.
%
%\item $\scope\NA\repreceive\NB\role\msg\NX.\PP \not\sim \;\repreceive\NB\role\msg\NX.\scope\NA\PP$,

%\comment{H: Does $\scope\NA\alpha.\PP \sim \alpha.\scope\NA\PP$ hold under some conditions (e.g., $\NA$ is not used in $\alpha$)? hopefully not allowing to prove 
%$\scope\NA \alpha.\beta.\PP \sim \alpha.\beta.\scope \NA \PP$ when $\NA$ can be received in $\alpha$ and used in $\beta$... Your thoughts welcome}\\
%\comment{I: You are right for both. $\scope\NA\alpha.\PP \sim \alpha.\scope\NA\PP$ is not true only  for $\alpha=\alpha_\NA$ and $\alpha=\sauth\NB\role\msg\NA$. For the latter, as you said, $\scope\NA\receive\NB\role\msg\NX.\send\NX\role\msg\NC.\PP$ can receive $\NA$ and then have authorized output along the receive name, while $\receive\NB\role\msg\NX.\send\NX\role\msg\NC.\scope\NA\PP$ after receiving $\NA$ cannot perform authorized output.}
\item For any process $\PP$, name $\NA$ and prefix $\alpha_\NB$ such that $\alpha_\NB\neq\sauth\NA\role\msg\NA$, we have that  $\scope\NA\scope\NA\alpha_\NB.\PP \sim \scope\NA\alpha_\NB.\scope\NA\PP$.
\item For any process $\PP$ and name $\NA$ %\del{and prefix $\alpha_\NB$} 
we have that $\scope\NA\scope\NA\scope\NA\sauth\NA\role\msg\NA.\PP \sim \scope\NA\scope\NA\sauth\NA\role\msg\NA.\scope\NA\PP$.
\item For any process $\PP$, names $\NA, \NB, \NC_1, \ldots, \NC_n$ %\del{and prefix $\alpha_\NB$} 
we have that $\scope{\NC_1}\ldots\scope{\NC_n}\scope\NA\scope\NB\sauth\NA\role\msg\NB.\PP \sim \scope\NA\scope\NB\sauth\NA\role\msg\NB.\scope{\NC_1}\ldots\scope{\NC_n}\PP$.

%\comment{H: I guess $\scope\NA\scope\NA\scope\NA\alpha.\PP \sim \scope\NA\scope\NA\alpha.\scope\NA\PP$ holds, right? Perhaps even generalisedto $a,b,c$ where $a,b$ are used in $\alpha$ in which
%case $c$ can cross the prefix...}\\
%\comment{I: I agree again.}
\end{enumerate}
\end{proposition}


\begin{proof}
\begin{enumerate}\item[]{}
\item Follows by showing that relation 
\[
{\cal R}=\{(\scope\NA\alpha.\PP, \alpha.\scope\NA\PP)\;|\; \alpha\notin\{\send\NA\role\msg\NB, \receive\NA\role\msg\NX,\repreceive\NA\role\msg\NX,\sauth\NA\role\msg\NB, \sauth\NB\role\msg\NA, \rauth\NA\role\msg\NB\}\} \;\cup\; \sim
\] 
is contained in $\sim$ by coinduction on the definition of strong bisimulation. Note that the only action of both processes $\scope\NA\alpha.\PP$ and $\alpha.\scope\NA\PP$ is determined by the prefix $\alpha$, and that the action is not pending on the authorization $\scope\NA$. Each action of one process can be mimicked by the other leading to the same process, hence the proof follows by reflexivity of strong bisimilarity.
\item Consider $\alpha=\receive\NB\role\msg\NX$ and $\beta=\send\NX\role\msg\NC$. 
Then one possible transition for the first process is 
\[\scope\NA\alpha.\beta.\PP\lts{{\scope\NB}\receive\NB\role\msg\NA} \scope\NA\scope\NB\send\NA\role\msg\NC.\PP\subst{\NA}{\NX}\lts{\send\NA\role\msg\NC}\scope\NB\scope\NA\PP\subst{\NA}{\NX}
\]
while 
\[
\alpha.\beta.\scope\NA\PP\lts{{\scope\NB}\receive\NB\role\msg\NA} \scope\NB\send\NA\role\msg\NC.\scope\NA\PP\subst{\NA}{\NX}\lts{{\scope\NA}\send\NA\role\msg\NC}\scope\NB\scope\NA\PP\subst{\NA}{\NX}
\]
\item The witnessing relation in this case is
\[
{\cal R}=\{(\scope\NA\scope\NA\alpha.\PP, \scope\NA\alpha.\scope\NA\PP)\;|\; \alpha\in\{\send\NA\role\msg\NB, \receive\NA\role\msg\NX,\sauth\NA\role\msg\NC, \sauth\NC\role\msg\NA, \rauth\NA\role\msg\NB\}, \NC\not=\NA\} \;\cup\; \sim
\] 
\item The proof follows by considering witnessing relation 
\[
{\cal R}=\{(\scope\NA\scope\NA\scope\NA\sauth\NA\role\msg\NA.\PP , \scope\NA\scope\NA\sauth\NA\role\msg\NA.\scope\NA\PP)\} \;\cup\; \sim
\] 
\item Follows directly from statements 1., 2., and 4. of this Proposition, Theorem~\ref{th:bisimlarity_is_a_congruence} and Proposition~\ref{prop:struct.equiv_bis}.
\end{enumerate}
\end{proof}



The last proposition shows principles that can be used when trying to obtain  a semantic normal form characterization of processes. Statements $1.$, $3.$ and $4.$ attest that
an authorization can be pushed and pulled across the active prefix if the authorization is not needed to perform the action specified by the prefix, or in the presence the needed authorizations.  Statement $2.$ shows that the authorization can only be pushed across the immediately active prefix. The first four statements are generalized in $5.$: in the presence of the required authorizations all others can be pushed across the active prefix. % which says that when the required authorizations are present, all others can be pushed across.

%\marginpar{H: highlight $1.$ holds even $\alpha_\NB = b(a)$}

To conclude, we remark that the inequalities we have presented in this section can be seen as a justification of our novel approach to define the reduction semantics (using contexts and the $\operator$ operator) since a normal form characterization of processes in our model seems hard to obtain. %The main reason is the inability to manipulate authorization scoping over parallel composition.
%
Regardless of the fact that the equalities given in Proposition~\ref{prop:behavior_auth_and_prefixes} show we can manipulate 
authorizations over active prefixes, 
the inequalities given in Proposition~\ref{prop:behavior_auth_and?parallel} inform on the difficulty 
in manipulating authorization scoping over parallel composition.

