%!TEX root =  main.tex
\chapter*{Rezime}
\pagestyle{plain}

Razni softverski sistemi promenili su na\v cin na koji ljudi komuniciraju, u\v ce i vode posao, a me\dj usobno povezani ra\v cunarski sistemi imaju brojne pozitivne primene u svakodnevnom \v zivotu. Tokom protekle decenije obim obrade i koli\v cina podataka znatno su se pove\'cali~\cite{ChiangZ16}. Ovo pove\'canje obima za posledicu je imalo sve ve\'cu upotrebu distribuiranih sistema da bi se ti poslovi mogli obaviti uspe\v sno.

Pro\v sirena stvarnost (AR), igre preko mre\v ze, automatsko prepoznavanje lica, autonomna vozila ili aplikacije internet stvari (IoT) proizvode izuzetno velike koli\v cine podataka. Ovakva optere\'cenja zahtevaju ka\v snjenje ispod nekoliko desetina milisekundi~\cite{ChiangZ16}. Ovakvi zahtevi su izvan onoga \v sto centralizovani ra\v cunarski modeli, poput ra\v cunarstva u oblaku, mogu da ponude~\cite{ChiangZ16}. \v cak i mali problemi mogu dovesti do velikog zastoja u komunikaciji aplikacija i uslugama od kojih ljudi zavise. 

Primer koji se nedavno desio jo\v s jedan je u nizu otkaza na Amazon Web Services (AWS) platformi. Ovim je platforma bila nedostupna korisnicima i aplikacijama, a kao rezultat nedostupnosti platforme velika koli\v cina aplikacija i servisa, koji se izvr\v savaju preko interneta, postaje potpuno nedostupna korisnicima i na kraju neupotrebljiva. Da bismo razumeli kako smo do\v sli do tog problema, prvo moramo razumeti \v sta je zapravo ra\v cunarstvo u oblaku i videti kako je ovaj model organizovan.

Ra\v cunarstvo u oblaku mo\v zemo definisati kao skup ra\v cunarskih resursa koji se mogu ponuditi korisnicima kroz takozvani uslu\v zni softver~\cite{Vogels}. Hardware i software u velikim centrima za obradu podataka pru\v zaju usluge svojim korisnicima preko interneta~\cite {AboveTheCloud}. Resursi poput CPU-a, GPU-a, skladi\v sta podataka i mre\v ze mogu se koristiti za nekakvu obradu podataka, ili osloboditi i to na zahtev i po potrebi korisnika~\cite {ZhangCB10}. 

Klju\v cna snaga ra\v cunarstva u oblaku su servisi koji su ponu\dj eni korisnicima kao usluge ili uslu\v zni softver~\cite{Vogels}. Tradicionalni model ra\v cunarstva u oblaku pru\v za ogromne procesne i skladi\v sne resurse i to po potrebi, na zahtev korisnika elasti\v cno, kako bi podr\v zao razli\v cite potrebe razli\v citih aplikacija.  Ovo svojstvo odnosi se na sposobnost ra\v cunarstva u oblaku da dozvoli korisnicima alokaciju dodatnih resursa ili osloba\dj anje postoje\'cih kako bi se podudarali sa radnim optere\'cenjima aplikacije i to na zahtev~\cite{AssuncaoVB18}.

Problem, izme\dj u ostalog, nastaje kada je potrebno da se (veliki) podaci prebace sa svog izvori\v sta u oblak. Ovaj proces dovodi do velike latencije ili ka\v snjenja u sistemu~\cite {HossainRH18}. Na primer, Boeing 787s generi\v se pola terabajta podataka po jednom letu, dok autonomni automobil generi\v se dva petabajta podataka tokom samo jedne vo\v znje. 

Me\dj utim, propusni opseg nije dovoljno veliki da bi podr\v zao takve zahteve~\cite{CaoZS18}. Prenos podataka nije jedini problem sa kojim se ra\v cunarstvo u oblaku susre\'ce. Aplikacije kao \v sto su autonomni automobili, bespilotne letelice ili balansiranje snage u elektri\v cnim mre\v zama, zahtevaju obradu podataka u realnom vremenu da bi ispravno donosile odluke i reagovale na razne promene~\cite{CaoZS18}.

Centralizovana arhitektura ra\v cunarstva u oblaku, sa ogromnim kapacitetima centara za obradu podataka, stvara efikasnu ekonomiju obima. Ovom strategijom dolazi se do smanjenja administrativnih tro\v skova celokupnog sistema~\cite{BariBEGPRZZ13}. Me\dj utim, kada takav sistem do\dj e do svojih granica, centralizacija donosi vi\v se problema nego \v sto ih mo\v ze re\v siti~\cite{GunawiHSLSAE16, LopezMEDHIBFR15}. Uprkos svim prednostima ovog modela, servisi i usluge vremenom se suo\v cavaju sa ozbiljnom degradacijom kvaliteta odziva i performansi usled velike propusnosti i ka\v snjenja~\cite {KarimIWGSYO16}. To mo\v ze dovesti do nesagledivih posledica po biznis, ali potencijalno, i po ljudske \v zivote. 

Razne organizacije koriste usluge ra\v cunarstva u oblaku i oslanjaju se na njega kako bi izbegle izuzetno velike infrastrukturalne investicije~\cite {MonsalveCC18} poput pravljenja i odr\v zavanja sopstvenih centara za obradu podataka. Oni koriste resurse koje su obezbedili drugi pru\v zaoci usluga~\cite{Satyanarayanan17} i pla\'caju shodno tome koliko vremenski koriste usluge - a pay as you go model.

Cilj ove teze je predstavljanje i upotreba formalnih modela na osnovu kojih mo\v zemo opisati, formalno verifikovati protokole i implementirati radni okvir za distribuirani sistem, koriste\'ci geografski rasprostranjena okru\v zenja nalik ra\v cunarstvu u oblaku. Opisani sistem mogu koristiti ne samo obi\v cni korisnici, ve\'c ga i pru\v zaoci usluga ra\v cunarstva u oblaku mogu integrisati u svoju platformu i svoje servise kako bi se minimalizovao zastoj kriti\v cnih sistemskih segmenata. \v citav sistem mo\v zemo posmatrati kao skup mikro oblaka ili sloj obrade, koji u oblak \v salje samo va\v zne podatke, smajuju\'ci tro\v skove korisnicima, ali i obezbe\dj uju\'ci ve\'cu dostupnost usluga ra\v cunarstva u oblaku.

Distribuirane softverske sisteme nije jednostavno implementirati, niti modelovati. Problem \v cesto nastaje zbog problema u komunikaciji \v cvorova preko mre\v ze koja nije sigurna, a ni pouzdana. Poruke mogu da kasne; mogu da stignu u razli\v citom redosledu ili da ne stignu nikada. Tako\dj e, \v cvorovi u sistemu mogu da prestanu da rade potpuno nasumi\v cno stvaraju\'ci dodatne komplikacije. James Gosling i Peter Deutsch, nekada\v snji saradnici iz Sun Microsistems-a, kreirali su listu problema za mre\v zne aplikacije poznate kao \textit{8 zabluda distribuiranih sistema}:

\begin{enumerate}[start=1,label={(\bfseries \arabic*)}]
	\item \textbf{Mre\v za je pouzdana.} Uvek \'ce se ne\v sto katastrofalno desiti sa mre\v zom koja je prili\v cno nepouzdana - prekid napajanja, prekid kabla, katastrofe u okru\v zenju itd;
	\item \textbf{Latencija ne postoji.} Lokalno ka\v snjenje nije problem, ali se situacija vrlo brzo pogor\v sava kada pre\dj emo na komunikaciju preko interneta i scenario gde se koristi izuzetno kompleksna mre\v zna komunikacija ra\v cunarstva u oblaku;
	\item \textbf{Propusnost je beskona\v cna.} Iako se \v sirina propusnog opsega stalno pove\'cava i sve je bolja i bolja; isto tako raste i koli\v cina podataka koju poku\v savamo da prebacimo na obradu ili skladi\v stenje;
	\item \textbf{Mre\v za je sigurna.} Trendovi internet napada pokazuju izuzetno veliki rast napada, a ovo jo\v s vi\v se postaje problem u ra\v cunarstvu u oblaku javnog tipa;
	\item \textbf{Topologija se ne menja.} Mre\v zna topologija obi\v cno je izvan kontrole korisnika, a topologija mre\v ze stalno se menja usled brojnih razloga - dodati ili uklonjeni novi ure\dj aji, serveri, prekidi u komunikaciji itd;
	\item \textbf{Postoji samo jedan administrator.} Danas postoje brojni administrativi za veb servere, baze podataka, ke\v s memoriju i sli\v cno, ali, tako\dj e, kompanije sara\dj uju sa drugim kompanijama ili pru\v zaocima usluga ra\v cunarstva u oblaku;
	\item \textbf{Tro\v skovi transporta ne postoje.} Ova tvrdnja nikako nije ta\v cna iz prostog razloga \v sto moramo serijalizovati informacije i podatke koje \v saljemo \v sto tro\v si resurse i pove\'cava ukupno ka\v snjenje. Ovde nije problem samo u ka\v snjenju, ve\'c u tome \v sto svaka serijalizacija informacija zahteva dodatno vreme i dodatne resurse;
	\item \textbf{Mre\v za je homogena.} Danas je homogena mre\v za izuzetak, a ne pravilo. Imamo razli\v cite servere, sisteme, klijente koji komuniciraju. Implikacija ovoga je da, pre ili kasnije, moramo pretpostaviti da nam je potrebna interoperabilnost izme\dj u ovih sistema. Mogli bismo da imamo i neke za\v sti\'cene protokole, koji nisu javno dostupni, a koji bi tako\dj e mogli tro\v siti dodatno vreme. Pri tome, oni mogu ostati bez podr\v ske, pa bi ih trebalo izbegavati.
\end{enumerate}

Ove zablude definisane su pre vi\v se od deceniju. Pre vi\v se od \v cetiri decenije po\v celi smo da gradimo distribuirane sisteme, ali karakteristike i osnovni problemi ostaju isti. Zanimljiva je \v cinjenica da dizajneri i arhitekte i dalje pretpostavljaju da tehnologija sve re\v sava. Me\dj utim, to nije slu\v caj sa distribuiranim sistemima i ove zablude ne bi trebalo zaboraviti i ignorisati. Distribuirane sisteme te\v sko je korektno primeniti, a te\v sko ih je i modelovati, testirati, odr\v zavati, ali i implementirati usled ovih problema.

Programeri i dizajneri distribuiranih sistema i dan-danas \v cesto zaboravljaju na definisane probleme, \v sto neretko dovodi do izuzetno velikih pote\v sko\'ca. Na\v cin da se to u ranim fazama otkrije jeste kori\v s\'cenje formalnih matemati\v ckih metoda za opisivanje i modelovanje ovih sistema. Ove metode sa\v cinjavaju razne tehnike koje slu\v ze za specifikaciju i verifikaciju kompleksnih sistema i koje su zasnovane na matemati\v ckim i logi\v ckim principima.

Suo\v cavamo se sa ozbiljnim problemima koji mogu nastati zbog ka\v snjenja ili, ako usluga u oblaku postane nedostupna, zbog napada malicioznih korisnika - hakera, ali i usled prostog kvara na mre\v zi~\cite{GunawiHSLSAE16}. Istra\v zivanje je dovelo do novih ra\v cunarskih oblasti poput tzv. ivi\v cnog ra\v cunarstva (EC). 

EC je model u kome se procesne i skladi\v sne mogu\'cnosti ra\v cunarstva u oblaku prebacuju u blizini izvora podataka~\cite{Satyanarayanan17}. Kao posledicu toga, imamo da se ra\v cunarstvo u oblaku pro\v siruje novim mogu\'cnostima. Smanjuje se ka\v snjenje \v sto onda dovodi do novih mogu\'cnosti za aplikacije budu\'ce generacije~\cite{NingLSY20}. 

Tokom prethodnih godina, pojavili su se razni modeli koji spu\v staju obradu i skladi\v stenje podataka bli\v ze izvori\v stu, poput fog ra\v cunarstva~\cite{BonomiMNZ14}, cloudlet-a~\cite {MonsalveCC18} i mobilnih ivi\v cnih ra\v cunara (MEC)~\cite{WangZZWYW17}. U ovom radu sve ove modele nazivamo ivi\v cnim \v cvorovima. 

Svi pomenuti modeli koriste koncept prenosa skladi\v snih i procesnih mogu\v cnosti iz oblaka bli\v ze izvorima podataka,~\cite{KhuneP19} dok su zahtevnije obrade i dalje zadr\v zane u oblaku iz vrlo prostog razloga - dostupnost znatno ve\'ce koli\v cine resursa~\cite{NingLSY20}. EC modeli uvode male servere koji se arhitekturalno nalaze izme\dj u izvora podataka i oblaka. Tipi\v cno je za ove servere da imaju manje mogu\'cnosti u pore\dj enju sa serverima u oblaku~\cite{ChenHLLW15}. 

Prednost malih servera je u tome \v sto se oni mogu na\'ci skoro bilo gde, na primer u baznim stanicama~\cite{WangZZWYW17}, gradskim centralama, kafi\'cima, ili mogu biti rasprostranjeni po geografskim regionima, a sve to kako bi se izbeglo ka\v snjenje i pove\'cala propusnost~\cite{MonsalveCC18}. 

Oni mogu poslu\v ziti kao za\v stitni sloj~\cite{SatyanarayananK19} ili kao nivo obrade pre nego \v sto podaci budu poslati u oblak. Sa druge strane, korisnici dobijaju jedinstvenu mogu\'cnost dinami\v cke i selektivne kontrole informacija koje bivaju poslate u oblak. 

Jo\v s jednu prednost ovih servera predstavili su  Aroca~\cite{ArocaG12} i saradnici. Naime, njihovi rezultati pokazali su da mali serveri zadr\v zavaju dobre performanse prilikom pokretanja servera i klasterskog okru\v zenja. Malo slabije performanse pokazali su u slu\v caju trenutno dostupnih skladi\v sta podataka, ali to mo\v ze biti podsticaj da se polje istra\v zivanja skladi\v sta podataka dopuni novim modelima, optimizovanim za male servere.

Jedna jedina opcija te\v sko \'ce odgovarati potrebama svih aplikacija u budu\'cnosti tako da ra\v cunarstvo u oblaku ne bi trebalo da bude na\v sa granica i jedina opcija. Razni modeli, nastali na bazi malih servera, pokazuju mogu\'cnost da se obrada podataka mo\v ze obaviti bli\v ze izvori\v stu, dok te\v ski prora\v cuni mogu ostati u oblaku zbog ve\'ce dostupnosti resursa. U oblak treba slati samo informacije koje su klju\v cne za druge usluge ili aplikacije~\cite{inproceedingsSimic1}, a ne sve kako predla\v ze standardni model oblaka. 

Ideja malih servera sa razli\v citim ra\v cunskim, skladi\v snim i mre\v znim resursima pokre\'ce zanimljive istra\v ziva\v cke ideje i, kao takva, motivacija je za ovu tezu. Kori\v s\'cenje resursa, koji su organizovani lokalno kao mikro oblaci, oblaci zajednice ili ivi\v cni oblaci,~\cite{RydenOCW14} predla\v zu Riden i saradnici.

Usled problema koji mogu nastati u doglednoj budu\'cnosti kori\v s\'cenjem sve ve\'ceg broja ra\v cunarskih sistema, koji su povezani na internet, kao i  zbog ograni\v cenja ra\v cunarstva u oblaku u trenutnoj izvedbi, akademska zajednica kao i industrija po\v cele su da istra\v zuju i razvijaju odr\v ziva re\v senja. Neka istra\v zivanja vi\v se su usredsre\dj ena na prilago\dj avanje postoje\'cih re\v senja zahtevima EC-a, dok druga eksperimenti\v su sa novim idejama i re\v senjima.

U svom radu~\cite{GreenbergHMP09} Greenberg i saradnici isti\v cu da se mikro centri za obradu podataka (MDC) koriste prvenstveno kao \v cvorovi u mre\v zama za distribuciju sadr\v zaja i u drugim \say{sramotno distribuiranim} aplikacijama. MDC su zanimljiv model u podru\v cju brzih inovacija i razvoja. Greenberg i saradnici~\cite{GreenbergHMP09} uvode koncept MDC-a kao centra za obradu podataka koji se nalazi u blizini velike populacije, smanjuju\'ci pritom fiksne tro\v skove tradicionalnih centara za obradu podataka. Samim tim, minimalna veli\v cina MDC-a definisana je potrebama lokalnog stanovni\v stva~\cite{GreenbergHMP09, AbbasZTS18}, pru\v zaju\'ci agilnost kao klju\v cnu karakteristiku. Ovde agilnost zna\v ci sposobnost dinami\v ckog rasta i smanjenja potrebe za resursima kao i upotrebe resursa sa optimalne lokacije~\cite{GreenbergHMP09}. 

Zonska organizacija malih servera, koju su predstavili Guo i saradnici,~\cite{GuoRG20} u primeni kod pametnih vozila daje zanimljivu perspektivu o EC-u. Autori su pokazali kako modeli koji dele oblast na zone omogu\'cavaju kontinuitet dinami\v ckih usluga i smanjuju primopredaju veze. Tako\dj e, pokazali su kako da se pokrivenost malim serverima prenese na ve\'cu zonu, \v cime se pro\v siruju ra\v cunarska snaga i kapacitet skladi\v stenja podataka.

EC poti\v ce iz peer-to-peer sistema,~\cite{LopezMEDHIBFR15} kako su to pretpostavili Lopez i saradnici, ali ga pro\v siruju u novim pravcima i pru\v zaju mogu\'cnost integracije sa ra\v cunarstvom u oblaku.

U svom radu, Kurniawan i saradnici~\cite{inbookKurniawan} pokazali su vrlo lo\v su skalabilnost u centralizovanim modelima mre\v za za isporuku sadr\v zaja(CDN) u oblaku. Autori su predlo\v zili decentralizovano re\v senje koriste\'ci nano centre za obradu podataka koje \v cine mre\v zni ure\dj aji u ku\'ci~\cite{inbookKurniawan}. Ovi centri za obradu podataka opremljeni su, tako\dj e, sa ne\v sto skladi\v snog prostora. Autori su pokazali mogu\'cu upotrebu nano centara za obradu podataka \v cak i za neke velike primene sa jednom izuzetno bitnom predno\v s\'cu - mnogo manjom potro\v snjom energije.

MDC-ovi sa zonskom organizacijom servera dobra su polazna osnova za izgradnju EC-a (koja mo\v ze biti ponu\dj ena kao servis korisnicima), ali i mikro ra\v cunarstva u oblaku jer mo\v zemo relativno jednostavno pro\v siriti ra\v cunarsku snagu i skladi\v sni kapacitet koji opslu\v zuju lokalno stanovni\v stvo. Me\dj utim, da bismo to postigli, potreban nam je dostupniji i elasti\v cniji sistem sa manje ka\v snjenja. 

Ako pogledamo dizajn ra\v cunarstva u oblaku, svaki deo doprinosi otpornijem i skalabilnom sistemu. Regioni ili centri za obradu podataka izolovani su i nezavisni jedni od drugih, a tako\dj e sadr\v ze resurse koji su potrebni aplikacijama za nesmetan rad. Regioni su sa\v cinjeni od nekoliko dostupnih zona~\cite{SouzaMFAK19}. Ako neka od zona, iz bilo kog razloga, postane nedostupna, ima ih jo\v s koje mogu da opslu\v ze korisni\v cke zahteve i ceo sistem mo\v ze da nastavi nesmetano da radi. Uz neke adaptacije, EC i mikro ra\v cunarstvo u oblaku mogli bi koristiti vrlo sli\v cnu strategiju.

Male servere ili \v cvorove mo\v zemo grupisati u klastere, a vi\v se klastera \v cvorova u ve\'cu logi\v cku celinu \say{region}, pove\'cavaju\'ci dostupnost i pouzdanost sistema i njegovih aplikacija. Kada pri\v camo o EC-u i mikro ra\v cunarstvu u oblaku, mislimo na geografski rasprostranjene distribuirane sisteme tako da imamo malo druga\v ciji scenario nego u standardnom modelu ra\v cunarstva u oblaku. 

Koncept \say{regiona} u ra\v cunarstvu oblaka je fizi\v cki element~\cite{SouzaMFAK19}, dok se u mikro ra\v cunarstvu u oblaku pojam region mo\v ze koristiti za opisivanje skupova klastera \v cvorova preko proizvoljne geografske oblasti. 

Regioni se sastoje od najmanje jednog klastera, ali mogu se sastojati i od vi\v se njih tako da se postigne otporniji, skalabilniji i dostupniji sistem. Da bi se osiguralo manje ka\v snjenje u sistemu, u normalnim okolnostima treba izbegavati veliku udaljenost izme\dj u klastera. U tradicionalnom modelu ra\v cunarstva u oblaku pro\v sirenje regiona zahteva fizi\v cko povezivanje novih modula sa ostatkom infrastrukture~\cite{Hamilton07}, \v sto mo\v ze izazvati nedostupnost tog regiona neko vreme. 

U mikro ra\v cunarstvu u oblaku regioni mogu prihvatiti nove ili osloboditi postoje\'ce klastere. Isto tako i klasteri mogu prihvatiti nove ili osloboditi postoje\'ce \v cvorove dinami\v cki, bez direktnog povezivanja novih modula.

Vi\v se regiona \v cine slede\'ci logi\v cki sloj -- \say{topologiju}. Topologija se sastoji od najmanje jednog regiona, a mo\v ze se prostirati i na vi\v se regiona. Prilikom dizajniranja topologije, posebno ako regioni treba da dele informacije ili da na neki na\v cin sara\dj uju, po\v zeljno je izbegavati veliku udaljenost izme\dj u regiona ako je to mogu\'ce. 

Sa ovim vrlo jednostavnim konceptima mo\v zemo pokriti bilo koju geografsku oblast sa sposobno\v s\'cu da smanjimo ili pro\v sirimo postoje\'ce klastere, regione pa \v cak i topologije. Organizacija klastera, regiona i topologija u mikro ra\v cunarstvu u oblaku isklju\v civo je stvar dogovora i, kao takva, sli\v cna je modeliranju u sistemima velikih podataka~\cite{SonbolOAA20, WangCAL14}.

Na primer, klasteri mogu biti veliki kao \v citav jedan grad ili mali kao svi ure\dj aji u jednom doma\'cinstvu i sve izme\dj u ova dva ekstrema. Grad bi mogao predstavljati jedan region sa delovima koji su organizovani u klastere. Topologiju grada mo\v zemo formirati tako \v sto \'cemo grad podeliti na vi\v se regiona koji sadr\v ze vi\v se klastera. Topologiju dr\v zave mo\v zemo formirati tako \v sto \'cemo je podeliti na regione pri \v cemu su gradovi regioni i tako dalje. 

\v cvorovi unutar svakog klastera treba da izvr\v savaju neki od protokola za odr\v zavanje definicije klastera odnosno pripadnosti \v cvorova klasteru. Neki od \textit{Gossip} protokola poput \textit{SWIM}-a~\cite{DasGM02} mogu se koristiti u saradnji sa mehanizmima replikacije podataka~\cite {LiBCL20, CauCBFCEB16, CRDTS_Nuno} \v cine\'ci ceo sistem otpornijim na potencijalne gre\v ske. Treba prihvatiti \v cinjenicu da \'ce \v cvorovi u takvom sistemu iz raznih razloga biti nedostupni. To ne mo\v zemo izbe\'ci, ali mo\v zemo projektovati sistem oko te ideje tako da servisi ipak budu dostupni koriste\'ci pritom neki od kopija servisa.

U modelu koji opisuje razne resurse kao usluge~\cite{DuanFZSNH15} EC i mikro ra\v cunarstvo u oblaku nalaze se izme\dj u CaaS-a i PaaS-a, u zavisnosti od potreba korisnika.

Dobro definisan sistem mogao bi se ponuditi kao usluga korisnicima kao i bilo koji drugi resurs ra\v cunarstva u oblaku. Mo\v zemo ga ponuditi istra\v ziva\v cima i programerima da naprave nove aplikacije usmerene vi\v se ka raznim potrebama ljudi. Ako nam je potrebno vi\v se resursa na jednoj strani, mo\v zemo uzeti odre\dj enu koli\v cinu resursa i premestiti je tamo gde nam ti resursi stvarno trebaju. Sa druge strane, kompanije koje pru\v zaju usluge ra\v cunarstva u oblaku mogu integrisati model u svoj postoje\'ci sistem, skrivaju\'ci nepotrebnu slo\v zenost iza nekog komunikacionog interfejsa ili predlo\v zenog modela aplikacije.

Da bi se postiglo takvo pona\v sanje, neophodno je imati dinami\v cko upravljanje resursima i upravljanje ure\dj ajima. Moramo uvek imati dostupne informacije o resursima, konfiguraciji i zauzetosti \v cvorova~\cite{GubbiBMP13, WangZZWYW17} i klastera u celini. Tradicionalni centri za obradu podataka predstavljaju dobro organizovan i povezan sistem. Sa druge strane, MDC-ovi se sastoje od razli\v citih ure\dj aja koji to nisu~\cite{JiangCGZW19}. Ovaj problem dovodi nas do problema kojim se bavi ova teza.

EC i MDC modelima nedostaje jasna dinami\v cka organizacija geografski raspore\dj enih \v cvorova, dobro definisan model mati\v cnih aplikacija i jasno razdvajanje nadle\v znosti u sistemu. Kao takvi, ne mogu se ponuditi kao usluga korisnicima. EC sistemi obi\v cno postoje nezavisno jedni od drugih, rasuti bez me\dj usobne povezanosti i saradnje. Nude ih pru\v zaoci usluga koji korisnike uglavnom zaklju\v cavaju u sopstveni ekosistem \v cesto bez mogu\'cnosti izbora servisa van njihovog kataloga usluga. Grupisani \v cvorovi treba da budu organizovani lokalno, \v cine\'ci sistem kompletnim, a aplikacije dostupnijim i pouzdanijim, pro\v siruju\'ci resurse izvan pojedina\v cnog \v cvora ili male grupe \v cvorova. Takav sistem treba da odr\v zava dobre performanse za izgradnju servera i klastera~\cite{ArocaG12}.

Da bi opisali fizi\v cke usluge, Jin~\cite {JinCJL14} i saradnici predla\v zu tri osnovna koncepta i preciziraju njihove odnose. Ovi koncepti su: \textbf{(1)} ure\dj aji, \textbf{(2)} resursi i \textbf{(3)} servisi. 

Podela nadle\v znosti bitan je deo svakog sistema, posebno ako se stvara platforma koja se nudi korisnicima kao usluga. Model podele nadle\v znosti, koji ova teza predla\v ze, zasnovan je na ovim konceptima, prilago\dj en druga\v cijem slu\v caju kori\v s\'cenja i  podeljen u tri sloja \v sto se mo\v ze videti na slici~\ref {fig:fig10}. 

Donji sloj \v cine razli\v citi ure\dj aji ili kreatori podataka i korisnici usluga odnosno servisa. Drugi sloj predstavlja resurse. Resursi imaju prostorne karakteristike i ukazuju na mogu\'cnosti za obradu odnosno skladi\v stenje podataka \v cvorova na kojima se izvr\v savaju~\cite{JinCJL14}. Programeri u bilo kom trenutku moraju znati iskori\v s\'cenost resursa kao i stanje i dostupnost aplikacija. 

Resursi predstavljaju EC \v cvorove i, da bi \v cvor bio deo sistema, mora zadovoljiti \v cetiri jednostavna pravila: 

\begin{enumerate}[start=1,label={(\bfseries \arabic*)}]
\item Mora biti sposoban da pokrene operativni sistem sa sistemom datoteka;
\item Mora biti u mogu\'cnosti da pokrene neki od dostupnih alata za izolaciju aplikacija, na primer \textit{container} ili \textit{unikernel}; 
\item Mora imati dostupne resurse za kori\v s\'cenje (npr. CPU, GPU, disk itd.);
\item Mora imati stalnu internet vezu.
\end{enumerate}

Servisi pru\v zaju resurse aplikacijama putem definisanog interfejsa i \v cine ih dostupnim preko interneta~\cite {JinCJL14}. Servisi odmah odgovaraju na klijentske zahteve, ako je to mogu\'ce, ili mogu da skladi\v ste prona\dj enu informaciju za neke budu\'ce korisni\v cke upite~\cite {SatyanarayananBCD09, YaoXWYZP20}. Servisi koji se izvr\v savaju u oblaku treba da budu u stanju da prihvate unapred obra\dj ene podatke i odgovorni su za obradu i skladi\v stenje podataka \v ciji kapacitet prevazilazi mogu\'cnosti EC \v cvorova. Ovi servisi tako\dj e treba da budu zamenska opcija u slu\v caju da prethodno definisani sistem bude nedostupan iz bilo kog razloga.

Ovo pro\v sirenje ra\v cunarstva u oblaku produbljuje i ja\v ca na\v se dosada\v snje razumevanje oblasti u celini. Razdvajanjem nadle\v znosti modela mati\v cnih aplikacija i objedinjenjem organizacije \v cvorova, idemo ka ideji EC-a kao usluge i mogu\'cnosti dinami\v ckog pravljenja mikro oblaka koji bi mogli da obrade podatke na samom njihovom izvoru.

Me\dj utim, infrastruktura za takav sistem ne\'ce biti postavljena sve dok proces pode\v savanja i kori\v s\'cenja ne bude trivijalan~\cite{SatyanarayananBCD09}. Odlazak od \v cvora do \v cvora dosadan je i dugotrajan proces, naro\v cito kada se uzme u obzir geografska rasprostranjenost dostupnih \v cvorova. 

Model koji se predla\v ze u ovoj tezi re\v sava gorepomenuti problem pomo\'cu  dinami\v ckog pode\v savanja i formiranja klastera, regiona i topologija i oslanja se na \v cetiri protokola:

\begin{enumerate}[start=1,label={(\bfseries \arabic*)}]
	\item \textbf{Provera stanja \v cvora} - protokol obave\v stava sistem o stanju svakog \v cvora; 
	\item \textbf{Formiranje klastera} - protokol formira nove klastere, regione i topologije;
	\item \textbf{Provera idempotencije} - protokol proverava da li klaster, region ili topologija postoje, i da li je potrebno pokrenuti protokol za formiranje;
	\item \textbf{Pregled detalja} - protokol prikazuje trenutno stanje sistema korisniku kroz razne nivoe detalja.
\end{enumerate}

Da bismo formalno opisali servere ili \v cvorove (pojmovi se koriste naizmeni\v cno) u sistemu, mo\v zemo koristiti teoriju skupova. Prethodno definisane protokole mo\v zemo formalno modelirati koriste\'ci~\cite{HuY17} pro\v sirenje \emph{multiparty asynchronous session types} (MPST)~\cite {HondaYC08} - klasa tipova pona\v sanja skrojena za opisivanje distribuiranih protokola oslanjaju\'ci se na asinhrone komunikacije.

Ova matemati\v cka teorija nije korisna samo kao formalni opisi protokola, ve\'c je mo\v zemo iskoristitii kao teoriju za verifikaciju da li na\v si protokoli zadovoljavaju MPST sigurnost (nema dostupnog stanja gre\v ske) i napredak (akcija se na kraju izvr\v sava, pod pretpostavkom po\v stenja).

Proces modelovanja odvija se u dva koraka:

\begin{enumerate}[start=1,label={(\bfseries \arabic*)}]
	\item \textbf{Prvi korak} u modeliranju komunikacija sistema pomo\'cu MPST teorije je da pru\v zimo \emph{globalni tip}. To je globalni opis celokupnog protokola sa neutralne ta\v cke posmatranja.
	\item \textbf{Drugi korak} u modeliranju komunikacija sistema pomo\'cu MPST teorije je pru\v zanje sintaksi\v cke projekcije protokola na svakog u\v cesnika u komunikaciji iskazane kao \emph{lokalni tip}, koji se zatim koristi za proveru tipa i implementacije krajnje ta\v cke.
\end{enumerate}

Na osnovu prethodno opisanih ideja i mogu\'cnosti, defini\v semo problem koji ova teza obra\dj uje kroz slede\'ca tri istra\v ziva\v cka pitanja:

\begin{enumerate}[start=1,label={(\bfseries \arabic*)}]\label{rez:questions}
	\item \textit{Da li mo\v zemo da organizujemo geografski distribuirane \v cvorove na sli\v can na\v cin kao ra\v cunarstvo u oblaku, prilago\dj ene druga\v cijem okru\v zenju sa jasnom podelom nadle\v znosti i poznatim modelom razvoja aplikacija za korisnike?}
	\item \textit{Da li mo\v zemo da ponudimo ovako organizovane \v cvorove kao uslugu programerima i istra\v ziva\v cima za budu\'ce aplikacije usmerene vi\v se ka ljudima, a zasnovane na poznatom pay as you go modelu?}
	\item \textit{Da li mo\v zemo da formuli\v semo model na takav na\v cin da je formalno ispravan, lak za pro\v sirivanje, razumevanje i obrazlo\v zenje?}
\end{enumerate}

Ako su prethodna istra\v ziva\v cka pitanja potvrdna, onda pro\v sirenje nalik na oblak pro\v siruje resurse van granica pojedina\v cnog \v cvora \v sto \v citav sistem, kao i same aplikacije koje bi se izvr\v savale u njemu, \v cini dostupnijim i pouzdanijim.

Satyanarayanan i saradnici u svom radu~\cite{SatyanarayananK19} pokazuju da MDC-ovi mogu poslu\v ziti kao za\v stitni sloj. Simi\'c i saradnici u svom radu~\cite{inproceedingsSimic1}  opisuju takav sistem kao nivo obrade podataka na njihovom izvoru, dok korisnici dobijaju jedinstvenu mogu\'cnost dinami\v ckog i selektivnog upravljanja informacijama koje se \v salju u oblak. 

Godinama nakon svog osnivanja, EC vi\v se nije samo ideja~\cite{SatyanarayananK19}, ve\'c neophodan alat za nove tipove aplikacija koje dolaze.

Na osnovu prethodno definisanih istra\v ziva\v ckih pitanja i motivacija~\ref{rez:questions}, izvodimo hipoteze na kojima se temelji ova teza rezimirano na slede\'ci na\v cin:

\begin{enumerate}[start=1,label={(\bfseries \arabic*)}]
	\item \textbf{Hipoteza:} \textit{Mogu\'ce je organizovati \v cvorove na standardni na\v cin, zasnovan na arhitekturi ra\v cunarstva u oblaku i prilago\dj en druga\v cije rasprostranjenom geografskom okru\v zenju, pru\v zaju\'ci korisnicima mogu\'cnost da na najbolji mogu\'ci na\v cin organizuju \v cvorove i klastere po raznim geografskim oblastima kako bi opslu\v zivali samo lokalno stanovni\v stvo u neposrednoj blizini.}
	\item \textbf{Hipoteza:} \textit{Mogu\'ce je ponuditi dobijeni model istra\v ziva\v cima i programerima da kreiraju nove aplikacije usmerene vi\v se ka ljudima. Ako nam je potrebno vi\v se resursa na jednoj strani, mo\v zemo uzeti odre\dj enu koli\v cinu resursa i premestiti je na mesto gde su oni zaista potrebni ili ih organizovati na bilo koji drugi \v zeljeni na\v cin.}
	\item \textbf{Hipoteza:} \textit{Mogu\'ce je predstaviti jasnu podelu nadle\v znosti za budu\'ci sistem, koji bi bio pru\v zen korisnicima kao usluga, i uspostaviti dobro organizovan sistem u kojem svaki deo ima intuitivnu i jasnu ulogu.}
	\item \textbf{Hipoteza:} \textit{Mogu\'ce je predstaviti objedinjeni model, koji podr\v zava heterogene \v cvorove, sa jasnim setom tehni\v ckih zahteva koje budu\'ci \v cvorovi moraju ispuniti ako \v zele da postanu deo sistema.}
	\item \textbf{Hipoteza:} \textit{Mogu\'ce je predstaviti jasan aplikativni model, intuitivan korisnicima, kako bi se mogao iskoristiti puni potencijal novonastale infrastrukture.}
\end{enumerate}

Iz prethodno definisanih hipoteza izvodimo primarne ciljeve ove teze pri \v cemu o\v cekivani rezultati uklju\v cuju slede\'ce:

\begin{enumerate}[start=1,label={(\bfseries \arabic*)}]
	\item \textit{Postojanje konstrukcije modela sa jasnom podelom nadle\v znosti, po ugledu na organizaciju ra\v cunarstva u oblaku, koji bi bio prilago\dj en druga\v cijem okru\v zenju izvr\v savanja sa jasnim aplikativnim modelom koji \'ce mo\'ci da iskoristi novu, prilago\dj enu arhitekturu. Ovaj cilj odnosi se na prvo istra\v ziva\v cko pitanje, a definisano je kroz poglavlje~\ref{chapter:Micro_clouds}.}
	\item \textit{Definisani model je dostupniji i elasti\v can sa manje ka\v snjenja u pore\dj enju sa pojedina\v cnim malim serverima i, kao takav, \v siroj javnosti mo\v ze se ponuditi kao bilo koja druga usluga u oblaku. Ovo se odnosi na drugo istra\v ziva\v cko pitanje i tema je poglavlja~\ref{chapter:Micro_clouds}.}
	\item \textit{Definisani model dobro je opisan formalno, uz oslanjanje na \v cvrstu matemati\v cku osnovu, ali tako\dj e je pogodan za pro\v sirivanje (i formalno i tehni\v cki), lak je za razumevanje i obrazlo\v zenje. Ovo se odnosi na tre\'ce istra\v ziva\v cko pitanje i tema je poglavlja~\ref{chapter:Micro_clouds}.}
\end{enumerate}

\noindent
Ova teza predstavlja mogu\'ce re\v senje za organizaciju geografski rasprostranjenih mikro oblaka sa EC \v cvorovima, uz dodatak nekoliko dokazanih apstrakcija iz ra\v cunarstva u oblaku poput zona i regiona.

Ove apstrakcije omogu\'cavaju pokrivanje bilo kog geografskog podru\v cja i daju dostupniji i pouzdaniji sistem. Organizacija i reorganizacija ovih elemenata vr\v si se opisom \v zeljenog stanja bez direktnog slanja komandi sistemu, a veli\v cina regiona i klastera odre\dj uje se potrebama stanovni\v stva.

Predstavili smo preslikavanje ra\v cunarstva u oblaku na EC i uz to smo prikazali formalni model sistema sa jasnom podelom nadle\v znosti i mati\v cnim modelom aplikacije za budu\'ci EC koji bi bio ponu\dj en kao usluga. 

Teza tako\dj e prikazuje prototip implementiranog re\v senja, koriste\'ci prethodno opisane koncepte i formalne modele. Implementirani prototip mo\v ze se koristiti kao samostalno re\v senje tamo gde se kasnije mogu dodati potrebni podsistemi, ali tako\dj e pru\v za mogu\'cnost integracije u postoje\'ca re\v senja. Dati su primeri domena gde bi sistem mogao da se koristi zajedno sa primerima aplikacija od kojih bi korisnici imali benefit. 

\noindent
Teza je organizovana u pet poglavlja.

U \textbf{poglavlju~\ref{chapter:Intro}} dali smo opis motivacije sa jasno definisanim istra\v ziva\v ckim pitanjima i hipotezama na koje \v zelimo da odgovorimo ovom tezom.

U \textbf{poglavlju~\ref{chapter:Field_overview}} dali smo kratak uvod u temu distribuiranih sistema, sa fokusom na podru\v cja koja su va\v zna za razumevanje ove teze i svih njenih delova.

Pokazali smo \v sta su distribuirani sistemi ili bar op\v sti konsenzus kako neki sistem mo\v zemo opisati ili posmatrati kao distribuirani sistem. Predstavili smo probleme koje ovi sistemi stvaraju i za\v sto ih je tako te\v sko implementirati, koristiti i odr\v zavati.

Tako\dj e, predstavili smo nekoliko primera distribuiranih ra\v cunarskih aplikacija koje mo\v zemo primeniti za efikasno iskori\v s\'cavanje velikog broja \v cvorova u distribuiranom sistemu. Dalje smo pokazali \v sta je skalabilnost i za\v sto je ona va\v zna za distribuirane sisteme sa nekoliko primera organizacionih mogu\'cnosti, poput peer-to-peer i master-slave sistema, kao i protokola za opis grupa ili zajednica \v cvorova koji sara\dj uju, a koji su va\v zni u distribuiranom okru\v zenju iz razli\v citih razloga. Dali smo primere raznih varijanti ra\v cunarstva u oblaku koje mo\v zemo iskoristiti za svoje potrebe.

Zatim smo opisali nekoliko tehnika virtuelizacije koje se mogu koristiti za pakovanje i raspore\dj ivanje kako aplikacija, tako i infrastrukture. Prikazali smo razne tehnike bitne za raspore\dj ivanje aplikacija i infrastrukture u okru\v zenju ra\v cunarstva u oblaku, ali i razliku izme\dj u distriburanih sistema i nekoliko modela koji se \v cesto smatraju distribuiranim poput paralelnog i decentralizovanog ra\v cunarstva.

U \textbf{poglavlju~\ref{chapter:Review}} prikazali smo sli\v cne radove raznih istra\v ziva\v ca ili kompanija. Isto kao i u prethodnom poglavlju, fokusiramo se samo na stvari koje su na neki na\v cin povezane sa ovom tezom.

Prikazali smo razli\v cite platforme, gde autori menjaju ili prilago\dj avaju postoje\'ca re\v senja (kao \v sto su Kubernetes ili OpenStack) da rade u oblastima poput ivi\v cnog ra\v cunarstva i mobilnog ra\v cunarstva. Dalje smo predstavili implementacije nekoliko platformi koje koriste \v cvorove, a koje su korisnici ponudili na dobrovoljnoj bazi, da bi se izvr\v sila nekakva obrada ili skladi\v stenje podataka na njima, kao na primer drop computing i Nebule izme\dj u ostalih.

Pokazali smo kako \v cvorovi mogu biti organizovani po geografskim podru\v cjima na zone, ali i kako mikro centri za obradu podataka mogu da pomognu ra\v cunarstvu u oblaku da prihvata zahteve lokalnog stanovni\v stva koje koristi resurse u neposrednoj blizini. Dalje smo opisali razli\v cite tehnike kako se zadaci sa mobilnih ure\dj aja mogu prebaciti na ivi\v cne \v cvorove, ali tako\dj e i razli\v cite modele primene koji bi mogli iskoristiti ove tehnike.

Na kraju ovog poglavlja predstavili smo gde je mesto ove teze u pore\dj enju sa drugim sli\v cnim modelima i drugim sli\v cnim istra\v zivanjima.

\textbf{Poglavlje~\ref{chapter:Micro_clouds}} \v cini sr\v z ove teze. U ovom poglavlju razdvojili smo sve najva\v znije aspekte koje treba da zadovoljimo kako bismo pomogli ra\v cunarstvu u oblaku sa problemima kao \v sto su ka\v snjenje i obrada podataka posebno u doba mobilnih ure\dj aja i IoT-a.

Predlo\v zeni model zasnovan je na MDC-ima koji su zonski organizovani i koji \'ce opslu\v zivati lokalno stanovni\v stvo ili stanovni\v stvo u blizini. Predstavili smo model koji se zasniva na ra\v cunarstvu u oblaku, ali je prilago\dj en za druga\v ciji scenario i sli\v cne slu\v cajeve kori\v s\'cenja.

Pokazali smo kako mo\v zemo dinami\v cki formirati nove klastere, regione i topologije i kako ih mo\v zemo koristiti zajedno sa mobilnim ure\dj ajima i aplikacijama poput internet stvari (IoT). Ovaj novoformirani sistem oslanja se na jasan model podele nadle\v znosti, usvojen iz postoje\'cih istra\v zivanja i prilago\dj en za novu troslojnu arhitekturu. Formirani model mo\v ze da slu\v zi kao sloj za obradu podataka na samom izvori\v stu, ili skoro na samom izvori\v stu, kao sloj za za\v stitu privatnosti korisnika i kontrolu sadr\v zaja koji se \v salje u oblak. Predstavljeni sistem izuzetno je prilagodljiv i podlo\v zan pro\v sirivanju prema razli\v citim dimenzijama, odnosno potrebama i zahtevima.

Predstavljeni model mo\v ze biti ogroman kao cela dr\v zav ili malen kao pojedina\v cno doma\'cinstvo i sve izme\dj u toga. Veli\v cina klastera stvar je dogovora i predstavlja mogu\'cnost izbora. Predstavili smo kako programeri mogu iskoristiti novu infrastrukturu i koji sve modeli aplikacija mogu postojati, ali i kako administratori mogu rasporediti razvijene servise na novoformiranu infrastrukturu koriste\'ci opisni ili deskriptivni model, umesto eksplicitnog slanja komandi i koraka sistemu.

Pred kraj ovog poglavlja, prikazali smo posledice ovog modela, odnosno kako se isti mo\v ze koristiti kao sastavni deo postoje\'cih sistema (kao skladi\v ste informacija o \v cvorovima) ili se mo\v ze koristiti kao novi model u kom mo\v zemo razviti nove podsisteme. Predstavili smo protokole za stvaranje takvog sistema i modelirali ih koriste\'ci formalne matemati\v cke metode ili, konkretno, teoriju asinhronih tipova sesija. Sistem sledi formalni model i lako ga je pro\v siriti, kako formalno, tako i prakti\v cno.

Na samom kraju poglavlja dali smo ograni\v cenja ovog sistema i ,ujedno, ove teze, ali i svega onoga \v cega moramo biti svesni \textbf{ako} ako takva tehnologija bude kori\v s\'cena u realnim situacijama.

U \textbf{poglavlju~\ref{chapter:Implementation}} pokazali smo implementirani okvir zasnovan na znanju i istra\v zivanijma skupljenim iz prethodnih poglavlja. Ovde smo tako\dj e detaljno opisali operacije koje se mogu obaviti u prototipu; kako se implementirani model uklapa i gde mu je mesto u prethodno opisanom modelu podele nadle\v znosti.

Dalje smo izneli rezultate na\v sih eksperimenata u kontrolisanom okru\v zenju kao i ograni\v cenja implementiranog radnog okvira u trenutnoj fazi razvoja. Tako\dj e, opisali smo mogu\'ce primene ovog sistema, ali i to gde bi ovaj model mogao da se koristi kada bi u\v sao u upotrebu.

\textbf{Poglavlje~\ref{chapter:Conclusion}} predstavlja poslednje poglavlje ove teze. U ovom poglavlju zaklju\v cili smo tezu onim \v sto je ura\dj eno sa onim \v sta se mo\v ze uraditi u pogledu budu\'cih pravaca istra\v zivanja.\\\\

\noindent
\textbf{Klju\v c re\v ci:} distribuirani sistemi, ra\v cunarstvo u oblaku, vi\v sestruko ra\v cunarstvo u oblaku, mikroservisi, softver kao servis, ivi\v cno ra\v cunarstvo, mikro ra\v cunarstvo u oblaku, veliki podaci.