%!TEX root =  main.tex
\chapter*{Rezime}
\pagestyle{plain}

Razni softverski sistemi promenili su na\v cin na koji ljudi komuniciraju, u\v ce i vode posao, a me\dj usobno povezani ra\v cunarski sistemi imaju brojne pozitivne primene u svakodnevnom \v zivotu. Tokom protekle decenije, koli\v cina prora\v cuna i obim podataka su se znatno pove\'cali~\cite {ChiangZ16}, \v sto je za posledicu imalo sve ve\'cu upotrebu distriburanih sistema da bi se ti poslovi mogli obaviti uspe\v sno.

Pro\v sirena stvarnost, igre preko mre\v ze, automatksko prepoznavanje lica, autonomna vozila ili aplikacije Internet stvari (IoT) proizvode izuzetno velike koli\v ine podataka. Ovakva optere\'enja zahtevaju ka\v snjenje ispod nekoliko desetina milisekundi~\cite {ChiangZ16}. Ovi zahtevi su izvan onoga \v sto centralizovani ra\v cunarski modeli (npr. poput ra\v cunarstva u oblaku) moge da ponude~\cite {ChiangZ16}.\v Cak i mali problemi mogu dovesti do velikog zastoja u aplikacijama i uslugama od kojih ljudi zavise. 

Primer koji se desio nedavno, je još jedan u nizu prekida koji se dogodio na Amazon Web Services (AWS) platformi. Ovim je platforma bila nedostupna korisnicima i aplikacijama, a kao rezultat nedostupnosti platforme velika koli\v cina aplikacija i servisa koji se izvr\v savaju preko interneta postaje nedostupna korisnicima. Da bi razumeli kako smo do\v sli do tog problema, prvo moramo razumeti \v sta je zapravo ra\v cunarstvo u oblaku.

Računarstvo u oblaku mo\v zemo da defini\v semo kao agregaciju ra\v cunarskih resursa koji se mogu ponuditi korisnicima, kroz uslu\v zni softver~\cite {Vogels}. Hardver i softver u velikim centrima za obradu podataka pru\v zaju usluge korisnicima preko Interneta~\cite {AboveTheCloud}. Resursi poput CPU-a, GPU-a, skladi\v sta podataka i mre\v ze su resursi, i mogu se koristiti ili ne i to na zahtev i po potrebi korisnika~\cite {ZhangCB10}. 

Klju\v cna snaga ra\v cunarstva u oblaku su servisi koji su ponu\dj eni korisnicima kao usluge~\cite {Vogels}. Tradicionalni model ra\v cunarstva u oblaku pru\v za ogromne procesne i skladi\v sne resurse i to po potrebi, na zahtev korisnika --- elasti\v cno, kako bi podr\v zao razli\v cite potrebe aplikacija.  Ovo svojstvo se odnosi na sposobnost ovog modela da dozvoli alokaciju dodatnih resursa ili osloba\dj postoje\'cih, da bi se podudarali sa radnim optere\'c enjima aplikacije i to na zahtev~\cite {AssuncaoVB18}.

Problem nastaje (izme\dj u ostalog) kada je potrebno da se (veliki) podaci prebace sa izvora u oblak. Ovaj proces dovodi do velike latencije ili ka\v snjenja u sistemu~\cite {HossainRH18}. Na primer, Boeing 787s generi\v se pola terabajta podataka po jednom letu, dok autonomni automobil generiše dva petabajta podataka tokom samo jedne vo\v znje. Me\dj utim, propusni opseg nije dovoljno velik da bi podr\v zao takve zahteve~\cite {CaoZS18}. Prenos podataka nije jedini problem sa \v cime se ra\v cunarsto u oblaku susre\'ce. Aplikacije kao \v sto su autonomni automobili, bespilotne letelice ili balansiranje snage u elektri\v cnim mre\v zama zahtevaju obradu podatka u realnom vremenu da bi ispravno donosili odluke, i reagovali na promene~\cite {CaoZS18}. Suo\v cavamo se sa ozbiljnim problemima, ako usluga u oblaku postane nedostupna zbog napada milicioznih korisnika --- hakera, ili prostog kvara na mre\v zi~\cite{GunawiHSLSAE16}.

Centralizovana arhitektura ra\v cunarstva u oblaku sa ogromnim kapacitetima centara za obradu podataka, stvara efikasnu ekonomiju obima. Ovom strategiom se dolazi do smanjenja administrativnih tro\v kova celokupnog sistema~\cite{BariBEGPRZZ13}. Me\dj utim, kada takav sistem dod\dj e do svojih granica, centralizacija donosi vi\v se problema nego \v sto ih mo\v ze re\v siti~\cite{GunawiHSLSAE16, LopezMEDHIBFR15}. Uprkos svim prednostima ovog modela, servisi i usluge vremenom se suo\v cavaju sa ozbiljnom degradaciom kvaliteta odziva i performansi, usled velike propusnosti i ka\v snjenja~\cite {KarimIWGSYO16}. 

To mo\v ze dovesti do nesagledivih posledica po biznis, ali potencijalno i ljudske živote. Organizacije koriste usluge u oblaku kako bi izbegle velike infrastrukturalne investicije~\cite {MonsalveCC18}, poput pravljenja i održavanja sopstvenih centara za obradu podataka. Oni koriste resurse koje su obezbedili drugi~\cite{Satyanarayanan17}, i pla\'caju shodno tome koliko vremenski koriste usluge -- a pay as you go model.

Cilj ove teze je predstavi upotrebu formalnih modela na osnovu kojih mo\v zemo opisati, formalno verifikovati protokole i implementirati radni okvir za distribuirani sistem koriste\'ci geografski rasprostranjena okru\v zenja, nalik ra\v cunarstvu u oblaku. Opisani sistem mogu da koriste ne samo obi\v cni korisnici, ve\'ci pru\v zaoci usluga ra\v cunarstva u oblaku mogu ga integrisai u svoje servise kako bi se minimalizovao zastoj kritičnih sistemskih segmenata. \v Citav sistem mo\v zemo posmatrati kao skup mikro oblaka ili sloj obrade, koji u oblak \v salje samo va\v zne podatke smajuju\'c i tro\v skove korisnicima, ali i obezbe\dj uju\'ci ve\'cu dostupnost usluga ra\v cunarstva u oblaku.

Distribuirane softverske sisteme nije jednostavno implementirati niti modelovati. Problem \v cesto nastaje zbog problema u komunikaciji \v cvorova preko mre\v ze koja nije sigurna ni pouzdana. Poruke mogu da kasne ili da ne stignu nikada. Tako\dj e, \v cvorovi u sistemu mogu da prestanu da rade potpuno nasumi\v cno stvaraju\'ci dodatne komplikacije. James Gosling i Peter Deutsch, tada\v snji saradnici iz Sun Microsistems, kreirali su listu problema za mre\v zne aplikacije poznate kao \textit{8 zabluda distribuiranih sistema}:

\begin{enumerate}[start=1,label={(\bfseries \arabic*)}]
	\item \textbf{Mre\v za je pouzdana}; uvek \'ce se ne\v sto katastrofalno desiti sa mre\v zom koja je prili\v cno nepouzdana - prekid napajanja, prekid kabla, katastrofe u okru\v zenju itd;
	\item \textbf{Latencija ne postoji}; lokalno ka\v snjenje nije problem, ali se situacija vrlo brzo pogor\v sava kada pre\dj emo na komunikaciju preko Interneta i scenario gde se koristi izuzetno kompleksna mre\v zna komunikacija ra\v cunarstva u oblaku.
	\item \textbf {Propusnost je beskona\v cna}; iako se \v sirina propusnog opsega stalno pove\'cava i sve je bolja i bolja, isto tako raste i koli\v cina podataka koju poku\v savamo da prebacimo na obradu ili skladi\v stenje.
	\item \textbf {Mre\v za je sigurna}; Trendovi internet napada pokazuju izuzetno veliki rast napda, a ovo jo\v s vi\v se postaje problem u ra\v cunarstvu u oblaku javnog tipa.
	\item \textbf {Topologija se ne menja}; mre\v zna topologija je obi\v cno izvan kontrole korisnika, a topologija mre\v ze se stalno menja iz brojnih razloga - dodati ili uklonjeni novi ure\dj aji, serveri, prekidi u komunikaciji itd.
	\item \textbf {Postoji samo jedan administrator}; danas postoje brojni administrativi za veb servere, baze podataka, ke\v s memoriju i sli\v cno, ali tako\dj e kompanije sara\dj uje sa drugim kompanijama ili pru\v zaocima usluga ra\v cunarstva u oblaku.
	\item \textbf {Tro\v skovi transporta ne postoje}; ovo nikako nije ta\v cno iz prostog razloga \v sto moramo serializovati informacije i podatke koje saljemo, \v sto tro\v si resurse i pove\'cava ukupnu ka\v snjenje. Ovde nije problem samo u ka\v snjenju, ve\'c u tome \v sto  svaka serializacija informacija zahteva dodatno vreme i dodatne resurse.
	\item \textbf {Mre\v za je homogena}; danas je homogena mre\v za izuzetak, a ne pravilo. Imamo razli\v cite servere, sisteme, klijente koji komuniciraju. Implikacija ovoga je da pre ili kasnije moramo pretpostaviti da nam je potrebna interoperabilnost izme\dj u ovih sistema. Mogli bismo da imamo i neke za\v sti\'cene protokole koji nisu javno dostupni koji bi tako\dj e mogli tro\v siti dodatno vreme, pri tome oni mogu ostati bez podrške, pa bismo ih trebali izbegavati.
\end{enumerate}

Ove zablude su definisane pre vi\v se od deceniju, i više od \v cetiri decenije otkako smo po\v celi da gradimo DS, ali karakteristike i osnovni problemi ostaju prili\v cno isti. Zanimljiva je \v cinjenica da dizajneri, arhitekte i dalje pretpostavljaju da tehnologija sve re\v sava. To nije slu\v caj u DS-u i ove zablude ne bi trebalo zaboraviti. Zbog ovih problema, DS je te\v sko korektno primeniti, a te\v sko ih je testirati i odr\v zavati.

Programeri i dizajneri distribuiranih sistema i dan danas zaborave na ove zablude i to dovodi o velikih problema. Na\v cin da se to u ranim fazama otkrije je kori\v s\'cenje formalnih, matemati\v cki zasnovanih metoda. Ove metode \v cine razne tehnike koje slu\v ze za specifikaciju i verifikaciju kompleksnih sistema koje su zasnovane na matemati\v ckim i logi\v ckim principima.

Da bi se prevazi\v slo ka\v snjenje u oblaku, istra\v zivanje je dovelo do novih ra\v cunarskih oblasti poput ivi\v cnog ra\v cunarstva (EC). EC je model u kome se procesne i skladi\v sne mogu\'cnosti ra\v cunarstva u oblaku, u blizini izvora podataka~\cite{Satyanarayanan17}. Ra\v cunarstvo u oblaku je pro\v sireno novim mogu\'cnostima \v sto dovodi do novih ideja za aplikacije budu\'ce generacije~\cite{NingLSY20}. Tokom godina pojavili su se razni modeli poput fog ra\v cunarstva~\cite{BonomiMNZ14}, cloudleta~\cite {MonsalveCC18} i mobilnih ivi\v cnih ra\v cunara (MEC)~\cite{WangZZWYW17}. U ovom radu sve ove modele nazivamo ivi\v cnim \v cvorovima. 

Svi pomenuti modeli koriste koncept prenosa skladi\v snih i procesnih mogu\v cnosti, iz oblaka bli\v ze izvorima podataka~\cite{KhuneP19} dok su zahtevnije obrade ostaju u oblaku iz vrlo prostog razloga -- dostupnosti znatno ve\'ckoli\v cine resursa~\cite{NingLSY20}. EC modeli uvode male servere koji se arhitekturalno nalaze izme\dj u izvora podataka i oblaka. Tipi\v cno je za ove servere da imaju mnogo manje mogu\'cnosti u pore\dj enju sa servera u oblaku~\cite{ChenHLLW15}. 

Prednost malih servera je u tome \v sto se oni mogu na\'ci skoro bilo gde, na primer u baznim stanicama~\cite{WangZZWYW17}, gradskim centralama, kafi\'cima ili rasprostranjeni po geografskim regionima a sve to, kako bi se izbeglo kašnjenje, kao i pove\'cala propusnost~\cite{MonsalveCC18}. Oni mogu poslu\v ziti kao za\v stitni zidovi~\cite{SatyanarayananK19} ili kao nivo obrade pre nego \v sto podaci budu poslati u oblak. Sa druge strane, korisnici dobijaju jedinstvenu mogu\'cnost dinami\v cke i selektivne kontrole informacija koje bivaju poslate u oblak. Jo\v s jenda prednost ovih servera predstavili su su Aroca~\cite{ArocaG12} i saradnici, gde su njihovi rezultati pokazali da mali serveri zadr\v zavaju dobre perfomanse prilikom pokretanja servera i klasterskog okru\v zenja. Malo slabije performanse su pokazali u slu\v caju trenutno dostupnih skladi\v sta podataka, ali to mo\v ze biti podsticaj da se polje istr\v zivanja skladi\v sta podataka dopuni novim modelima optimizovanim za male servere.

Jedna opcija \'ce te\v sko mo\'cda odgovara potrebama svih aplikacija u budu\'cnosti, tako da ra\v cunarstvo u oblaku ne bi trebalo da bude naša granica i jedina opcija. Razni modeli nastali na bazi malih servera, pokazuju mogu\'cnost da se obrada podataka mo\v ze obaviti bli\v ze izvori\v stu, a sve u cilju smanjenja ka\v snjenje zahteva klijentskih aplikacija primenom malih servera, dok teški proračuni mogu ostati u oblaku zbog ve\'ce dostupnosti resursa. Slediti ideju, da u oblak poslati samo informacije koje su klju\v cne za druge usluge ili aplikacije~\cite{inproceedingsSimic1}, a ne sve kako predla\v ze standardni model oblaka. Ideja malih servera sa razli\v citim ra\v cunskim, skladi\v snim i mre\v znim resursima pokre\'ce zanimljive istraz\v ziva\'cke ideje i motivacija  su za ovu tezu. Koriste\'ci resurse koji su organizovani lokalno kao mikro oblaci, oblaci zajednice ili ivi\v cni oblaci~\cite {RydenOCW14} predla\v zu Riden i saradnici.

Suo\v ceni sa stvarnim problemima i problemima koji mogu nastati u doglednoj budu\'cnosti ali i ograni\v cenjima ra\v cunarstva u oblaku u trenutnoj izvedbi, akademska zajednica, kao i industrija po\v celi su da istra\v zuju i razvijaju odr\v ziva re\v senja. Neka istra\v zivanja su vi\v se usredsre\dj ena na prilago\dj avanje postoje\'cih re\v senja da odgovaraju EC, dok druga eksperimenti\v su sa novim idejama i re\v senjima.

U svom radu~\cite{GreenbergHMP09} Greenberg i saradnici ističu da se mikro centri za obradu podataka (MDC) koriste prvenstveno kao \v cvorovi u mre\v zama za distribuciju sadr\v zaja i drugim \say{sramotno distribuiranim} aplikacijama. MDC su zanimljiv model u podru\v cju brzih inovacija i razvoja. Greenberg i saradnici~\cite{GreenbergHMP09} uvode MDC-ove kao centar za obradu podataka koji rade u blizini velike populacije, smanjuju\'ci fiksne tro\v skove tradicionalnih centara za obradu podataka. Minimalna veli\v cina MDC-a definisana je potrebama lokalnog stanovništva~\cite{GreenbergHMP09, AbbasZTS18}, pri\v zaju\'ci agilnost kao klju\v cnu karakteristiku. Ovde agilnosti zna\v ci sposobnost dinami\v ckog rasta i smanjenja potrebe za resursima i upotrebe resursa sa optimalne lokacije~\cite{GreenbergHMP09}. 

Zonska organizacija malih servera koju su predstavili Guo i saradnici~\cite{GuoRG20} u primeni kod pametnih vozila, daje zanimljivu perspektivu o EC. Autori su pokazali kako modeli zasnovani na zonama omogućavaju kontinuitet dinami\v ckih usluga i smanjuju primopredaju veze. Tako\dj e, su pokazali kako da se pove\'ca pokrivenost malim serverima na ve\'cu zonu, \v cime se pro\v siruju ra\v cunarska snaga i kapacitet skladištenja podataka.

EC poti\v ce iz peer-to-peer sistema~\cite{LopezMEDHIBFR15} kako su to pretpostavili L{\'{o}}pez i saradnici, ali ga pro\v siruje u novim pravcima i pru\v za mogu\'cnost integracije sa ra\v cunarstvom u oblaku.

U som radu Kurniavan i saradnici~\cite {inbookKurniawan} su pokazali vrlo lo\v su skalabilnost u centralizovanim modelima mre\v za za isporuku sadr\v zaja u oblaku (CDN). Predlo\v zili su decentralizovano re\v senje koriste\'ci nano centre za obradu podataka sa\v cinjenu od mrežnih ure\dj aja u ku\'c~\cite{inbookKurniawan}. Ovi DC-ovi su tako\dj e opremljeni sa ne\v sto skladi\v sta. Autori su pokazali moguc\'u upotrebu nano centara za obradu podataka za neke velike primene, sa mnogo manjom potro\v snjom energije.

MDC-ovi sa zonskom organizacijom servera dobra su polazna osnova za izgradnju EC-a koja mo\v ze biti ponu\dj na kao servis i mikro ra\v cunarstva u oblaku, jer možemo pro\v siriti ra\v cunarsku snagu i skladi\v sni kapacitet koji opslu\v zuju lokalno stanovništvo. Da bi to postigli, potreban nam je dostupniji i elasti\v cniji sistem sa manje ka\v snjenja. 

Ako pogledamo dizajn CC, svaki deo doprinosi otpornijem i skalabilnom sistemu. Regioni ili centri za obradu podataka (DC) su izolovani i nezavisni jedni od drugih, a tako\dj e sadr\v ze resurse koji su potrebni aplikacijama za nesmetan rad. S\v cinjeni su od nekoliko dostupnih zona~\cite {SouzaMFAK19}. Ako zona iz bilo kog razloga postane nedostupna, ima ih jo\v s koje mogu da opslu\v ze korisni\v cke zahteve. Uz neke adaptacije, EC i mikro ra\v cunarsto u oblaku bi mogla da koristi vrlo sli\v cnu strategiju.

\v Cvorove mo\v zemo grupisati u klastere, a vi\v se klastera \v cvorova u ve\'cu logi\v cku celinu -- \say{region}, povećavaju\'ci dostupnost i pouzdanost sistema kao i njegovih aplikacija. Kada pri\v camo o EC i mikro ra\v cunarstvu u oblaku, \v cesto mislimo na geografski rasprostranjene distribuirane sistemime, tako da imamo malo druga\v ciji scenario nego u standardnom CC modelu. 

Koncept \say{regiona} u ra\v cunarstvu oblaka je fizi\v cki element~\cite{SouzaMFAK19}, dok se u mikro ra\v cunarstvu u oblaku region mo\v ze koristiti za opisivanje skupova klastera \v cvorova, preko proizvoljne geografske oblasti. 

Regioni se sastoje od najmanje jednog klastera, ali mogu se sastojati i od vi\v se njih, tako da se postigne otporniji, skalabilniji i dostupniji sistem. Da bi se osiguralo manje ka\v snjenje u sistmu, u normalnim okolnostima treba izbegavati veliku udaljenost izme\dj u klastera. U tradicionalnom modelu ra\v cunarstva u oblaku, pro\v sirenje regiona zahteva fizi\v cko povezivanje modula sa ostatkom infrastrukture~\cite {Hamilton07}. 

U mikro ra\v cunarstvu u oblaku, regioni mogu prihvatiti nove, ili osloboditi postoje\'ce klastere ali i klasteri mogu prihvatiti nove ili osloboditi postoje\'ce \v cvorove.

Vi\v se regiona \v cine slede\'ci logi\v cki sloj -- \say{topologiju}. Topologija se sastoji od najmanje jednog regiona, a mo\v ze se prostirati i na vi\v se regiona. Prilikom dizajniranja topologije, posebno ako regioni trebaju da dele informacije ili treba na neki na\v cin da sara\dj uju, treba izbegavati veliku udaljenost izme\dj u regiona, ako je to mogu\'ce. 

Ovim vrlo jednostavnim konceptima, mo\v zemo da pokrijemo bilo koju geografsku oblast sa sposobno\v s\'cu da smanjimo ili pro\v sirimo postoje\'ce klastere, regione pa \v cak i topologije. Organizacija klastera, regiona i topologija u mikro ra\v cunarstvu u oblaku je isklju\v civo stvar dogovora, i kao takva sli\v cna je modeliranju u sistemima velikih podataka~\cite {SonbolOAA20, WangCAL14}.

Na primer, klasteri mogu biti veliki kao \v citav jedan grad ili mali kao svi ure\dj aji u jednom doma\'cinstvu i sve izme\dj u ova dva ekstrema. Grad bi mogao predstavljati jedan region, sa delovima grada koji su organizovani u klastere. Topologiju grada mo\v zemo formirati tako što \'cemo grad podeliti na vi\v se regiona, koji sadr\v ze vi\v se klastera. Topologiju dr\v zave mo\v zemo formirati tako \v sto \'cemo \'cemo je podeliti na regione, pri \v cemu su gradovi regioni. 

\v Cvorovi unutar svakog klastera treba da izvr\v savaju neki od protokol-a za odr\v zavanje klastera. Neki od \textit{Gossip} protokola poput \textit{SWIM}-a\cite {DasGM02}, mogu se koristiti u saradnji sa mehanizmima replikacije podataka~\cite {LiBCL20, CauCBFCEB16, CRDTS_Nuno} \v cine\'ci ceo sistem otpornijim na potencijalne gre\v ske. 

U modelu koji opisuje sve resurse kao usluge~\cite {DuanFZSNH15}, EC i mikro ra\v cunarstvo u oblaku kao usluga se nalazi izme\dj u CaaS i PaaS, u zavisnosti od potreba korisnika.

Dobro definisan sistem mogao bi se ponuditi kao usluga korisnicima, kao i bilo koji drugi resurs ra\v unarstva u obaku. Mo\v zemo ga ponuditi istra\v ziva\v cima i programerima da naprave nove aplikacije usmerene vi\v se ka raznim potrebama ljudi. Ako nam je potrebno vi\v se resursa na jednoj strani, mo\v zemo uzeti jedanu koli\v cinu resursa i premestiti gde nam ti resursi trebaju. Sa druge strane, kompanije koje pru\v zaju usluge ra\v cunarstva u oblaku mogu da integrisati model u svoj postoje\'ci sistem, skrivaju\'ci nepotrebnu slo\v zenost iza nekog komunikacionog interfejsa ili predlo\v zenog modela aplikacije.

Da bi se postiglo takvo pona\v sanje, neophodno je dinami\v cko upravljanje resursima i upravljanje ure\dj ajima. Moramo uvek imati dostupne informacije o resursima, konfiguraciju i zauzetosti \v cvorova~\cite{GubbiBMP13, WangZZWYW17}. Tradicionalni centri za obradu podataka predstavljaju dobro organizovan i povezan sistem. Sa druge strane, MDC-ovi se sastoje od razli\v itih ure\dj aja koji nisu~\cite{JiangCGZW19}. Ova ideja nas dovodi do problema sa kojim se bavi ova teza.

EC i MDC modelima nedostaje jasna dinami\v cka organizacija geografski raspore\dj nenih \v cvorova, dobro definisan model mati\v cnih aplikacija i jasno razdvajanje nadle\v znosti. Kao takvi ne mogu se ponuditi kao usluga korisnicima. EC obi\v cno postoje nezavisno jedni od drugih, rasuti su bez me\dj usobne komunikacije i saradnje, a nude ih pru\v zaoci usluga, koji korisnike uglavnom zaklju\v cavaju u sopstveni ekosistem. Grupisani \v cvorovi treba da budu organizovani lokalno, \v cine\'ci kompletan sistem i aplikacije dostupnijim i pouzdanijim, ali tako\dj e pro\v siruju\'ci resurse izvan pojedina\v cnog \v cvora ili male grupe \v cvorova, odr\v zavaju\'ci dobre performanse za izgradnju servera i klastera~\cite{ArocaG12}.

Da bi opisao fizi\v cke usluge, Jin~\cite {JinCJL14} i saradnici predla\v zu tri osnovna koncepta i precizira njihove odnose. Ovi koncepti su: $(1)$ ure\dj aji, $(2)$ resursi i $(3)$ servisi. Podela nadle\v znosti je bitan deo svakog sistema, posebno ako se stvara platformu koja se nudi kao usluga. Model podele nadle\v znosti koji ova teza predla\v ze zasnovan je na ovim konceptima, prilago\dj en za druga\v ciji slu\v caj kori\v s\'cenja, i podeljen u tri sloja \v sto se mo\v ze videti na slici~\ref {fig:fig10}. 

Donji sloj \v cine razli\v citi ure\dj aji ili kreatori podataka i korisnici usluga odnosno servisa. Drugi sloj predstavlja resurse. Resursi imaju prostornu karakteristiku i ukazuju na mogu\'cnosti njihovih hosting ure\dj aja za obradu odnosno skladi\v stenje podataka~\cite{JinCJL14}. Programeri u bilo kom trenutku moraju znati iskori\v s\'cenost resursa, kao i stanje i dostupnost aplikacija. 

Resursi predstavljaju EC \v cvorove, i da bi \v cvor bio deo sistema, on mora zadovoljiti \v cetiri jednostavna pravila: 

\begin{enumerate}[start=1,label={(\bfseries \arabic*)}]
\item moraju biti sposobni da pokrenu operativni sistem sa sistemom datoteka;
\item moraju biti u mogu\'cnosti da pokrene neki alat za izolaciju aplikacija, na primer kontejnere ili unikernele; 
\item moraju imati dostupne resurse za kori\v s\'cenje (npr. CPU, GPU, disk itd.) ;
\item moraju imaju stabilnu internet vezu;
\end{enumerate}

Servisi pru\v zaju resurse putem definisanog interfejsa i \v cine ih dostupnim preko Interneta~\cite {JinCJL14}. Servisi odmah odgovaraju na klijentske zahteve, ako je to mogu\'ce, ili mogu da skladi\v ste prona\dj enu informaciju za neke budu\'ce zahteve~\cite {SatyanarayananBCD09, YaoXWYZP20}. Servisi koji se izvrv\v savaju u oblaku treba da budu u stanju da prihvate unapred obra\dj ene podatke i odgovorne su za obradu i skladi\v stenje podataka \v ciji kapacitet prevazilazi mogu\'cnosti EC \v cvorova.

Ovo pro\v sirenje ra\v cunarstva u oblaku produbljuje i ja\v ca na\v se dosadasnje razumevanje oblasti u celini. Razdvajanjem nadle\v znosti, modela mati\v cnih aplikacija i objedinjenem organizacije \v cvorova, idemo ka ideji EC-a kao usluge, i mogucnosti dinami\v ckog pravljenja mikro oblaka koji bi mogli da obrade podatke na samom njihovom izvoru.

Me\dj utim, infrastrukture ne\'ce biti postavljena sve dok proces pode\v savanja i kori\v s\'cenja ne bude trivijaln~\cite{SatyanarayananBCD09}. Odlazak od \v cvora do \v cvora je dosadan i dugotrajan proces. Naro\v cito kada se uzme u obzir geografska rasprostranjenost \v cvorova. 

Sistem koji je predlo\v zen u ovoj tezi, re\v sava problem pomo\'cu  pode\v svanja i formiranja prethodno definisanih elemenata, i oslanja se na tri protokola:

\begin{enumerate}[start=1,label={(\bfseries \arabic*)}]
	\item \textbf{provera stanja \v cvora} protokol obave\v stava sistem o stanju svakog \v cvora; 
	\item \textbf{formiranje klastera} protokol formira nove klastere, regione i topologije;  
	\item \textbf{pregled detalja} protokol prikazuje trenutno stanje sistema korisniku kroz razne nivoe detalja;
\end{enumerate}

Da bismo formalno jednostavno opisali servere ili \ cvorove (pojmovi se koriste naizmeni\v cno) u sistemu, mo\v zemo koristiti teoriju skupova. Prethodno definisane protokole mo\v zemo formalno modelirati korist\'ci~\cite {HuY17}, pro\v sirenje \emph{multiparty asynchronous session types} (MPST)~\cite {HondaYC08} --- klasa tipova pona\v sanja skrojena za opisivanje distribuiranih protokola oslanjaju\'ci se na asinhrone komunikacije.

Ova matemati\v cka teorija nije samo korisna kao formalni opisi protokola, ve\'c se mo\v zemo iskoristiti i mogu\'cnosti teorije za verifikaciju da li na\v si protokoli zadovoljavaju MPST sigurnost (nema dostupnog stanja gre\v ske) i napredak (akcija se na kraju izvr\v sava, pod pretpostavkom po\v stenja). 

Proces modelovanja se odvija u dva koraka:

\begin{enumerate}[start=1,label={(\bfseries \arabic*)}]
	\item \textbf{Prvi korak} u modeliranju komunikacija sistema pomoću MPST teorije je da pru\v zimo \emph{globalni tip}, to je globalni opis celokupnog protokola sa neutralnog ta\v cka posmatranja.
	\item \textbf{Drugi korak} u modeliranju komunikacija sistema pomoću MPST teorije je pru\v zanje sintaksi\v cke projekcije protokola na svakog u\v cesnika kao lokalni tip, koji se zatim koristi za proveru tipa i implementacije krajnje ta\v cke.
\end{enumerate}

Na osnovu ovih ideja, definišemo problem koji ova teza obra\dj uje kroz slede\'ca tri istra\v ziva\v cka pitanja:

\begin{enumerate}[start=1,label={(\bfseries \arabic*)}]\label{rez:questions}
	\item \textit{Da li mo\v zemo da organizujemo geografski distribuirane \v cvorove na sli\v can na\v cin kao ra\v cunarstvo u oblaku, adaptiran za razli\v cito okru\v zenje, sa jasnom podelim nadle\v znosti, i poznatim modelom razvoja aplikacija za korisnike?}
	\item \textit{Da li mo\v zemo da ponudimo ovako organizovane \v cvorove kao uslugu programerima i istra\v ziva\v cima za budu\'caplikacije usmerene vi\v se ka ljudima, a zasnovane na poznatom modelu -- pay as you go?}
	\item \textit{Da li mo\v zemo da napravimo model na takav na\v cin da je formalno ispravan, lak za pro\v sirivanje, razumevanje i obrazlo\v enje?}
\end{enumerate}

Ako su prethogna istra\v ziva\v cka pitanja potvrdna, onda pro\v sirenje nalik na oblak \v cini \v citav sistem, ali i same aplikacije koje bi se izvr\v savale u njemu dostupnijim i pouzdanijim, ali tako\dj e pro\v siruje resurse van granica pojedina\v cnog \v cvora. Satianaraianan i saradnici u svom radu~\cite{SatyanarayananK19} pokazuju da MDC-ovi mogu poslu\v ziti kao za\v stitni zidovi, dok Simi\'c i saradnici, u svom radu~\cite {inproceedingsSimic1} koriste sli\v cnu ideju kao nivo obrade podatka na njihovom mestu nastanka -- izvoru. Istovremeno, korisnici dobijaju jedinstvenu mogu\'cnost dinami\v ckog i selektivnog upravljanja informacijama koje se \v salju u oblak. Godinama nakon svog osnivanja, EC vi\v se nije samo ideja~\cite {SatyanarayananK19}, ve\'cneophodan alat za nove aplikacije koje dolaze.

Na osnovu prethodno definisanih istraživačkih pitanja i motivacije~\ref {rez:questions}, izvodimo hipoteze oko kojih se temelji teza, a koje se mo\v gu rezimirati na slede\'ci na\v cin:

\begin{enumerate}[start=1,label={(\bfseries \arabic*)}]
	\item \textbf{Hipoteza:} \textit{Mogu\'ce je organizovati \v cvorove na standardni na\v cin zasnovan na arhitekturi zasnovanu na ra\v cunarstvu u oblaku, prilago\dj en druga\v cijem geografski rasprostranjenim okruženju. Dati korisnicima mogu\'cnost da na najbolji mogu\'ci na\v cin organizuju \v cvorove po raznim geografskim oblastima kako bi opslu\v zivali samo lokalno stanovni\v stvo u neposrednoj blizini.}
	\item \textbf{Hipoteza:} \textit{Mogu\'ce je ponudite dobijeni model istra\v ziva\v cima i programerima da kreiraju nove aplikacije usmerene vi\v se ka ljudima. Ako nam je potrebno vi\v se resursa na jednoj strani, mo\v zemo uzeti odre\dj neu koli\v cinu resursa i premestiti na mesto gde su oni potrebni, ili ih organizovati na bilo koji drugi \v zeljeni na\v cin.}
	\item \textbf{Hipoteza:} \textit{Mogu\'ce je predstaviti jasnu podelu nadle\v znosti za budu\'ci sistem koji bi bio pru\v zen korisnicima kao uslugu, i uspostaviti dobro organizovan sistem u kojem svaki deo ima intuitivnu i jasnu ulogu.}
	\item \textbf{Hipoteza:} \textit{Mogu\'ce je predstaviti objedinjeni model koji podr\v zava heterogene \v cvorove, sa jasnim setom tehni\v ckih zahteva koje budu\'ci \v cvorovi moraju ispuniti, ako \v zele da postanu deo sistema.}
	\item \textbf{Hipoteza:} \textit{Mogu\'ce je predstaviti jasan aplikativni model intuitivan korisnicima, kako bi mogli da iskoriste puni potencijal novonastale infrastrukture.}
\end{enumerate}

Iz prethodno definisanih hipoteza izvodimo primarne ciljeve ove teze, gde o\v cekivani rezultati uklju\v cuju:

\begin{enumerate}[start=1,label={(\bfseries \arabic*)}]
	\item \textit{Konstrukcija modela sa jasnom podelom nadle\v znost, po ugledu na organizaciju ra\v cunarstva u oblaku, prilago\dj eno druga\v cijem okru\v zenju izvr\v savanja, sa jasnim aplikativnim modelom koji \'ce mo\'c i da iskoristi novu, adaptoranu arhitekturu. Ovo se odnosi na prvo istra\v ziva\v cko pitanje, a tema je poglavlja~\ref{chapter:Micro_clouds}.}
	\item \textit{Definisani model je dostupniji, elasti\v can sa manje ka\v cnjenja u pore\dj enju sa pojedina\v cnim malim serverima, i kao takav se \v siroj javnosti mo\v ze ponuditi kao usluga, kao bilo koja druga usluga u oblaku. Ovo se odnosi na drugo istra\v ziva\v cko pitanje i tema je poglavlja~\ref{chapter:Micro_clouds}.}
	\item \textit{Konstruisani model je dobro opisan formalno, koriste\'ci \v cvrstu matemati\v cku osnovu, ali tako\dj e i lako pro\v siriv i formalni i tehni\v cki, lak je za razumevanje i obrazlo\v zenje. Ovo se odnosi na tre\'ce istra\v ziva\v cko pitanje i tema je poglavlja~\ref{chapter:Micro_clouds}.}
\end{enumerate}

\noindent
Ova teza predstavlja mogu\'ce re\v senje za organizaciju geografski rasprostranjenih mikro-oblaka sa EC čvorovima, uz dodatak nekoliko dokazanih apstrakcija iz ra\v cunarstva u oblaku poput zona i regiona.

Ove apstrakcije omogu\'cavaju pokrivanje bilo kog geografskog područja i daju dostupniji i pouzdaniji sistem. Organizacija i reorganizacija ovih apstrakcija vr\v si se opisom zeljenog stanja bez direknog slanja komandi sistemu, a njihova veli\v cina odre\dj uje se potrebama stanovni\v stva.

Predstavili smo preslikavanje ra\v cunarstva u oblaku na EC, i uz to smo prikazali formalni model sistema, sa jasnim modelom i sistemom podela nadle\v znosti i mati\v cnim modelom aplikacije za budu\'ci EC koji bi bio ponu\dj en kao usluga. 

Teza tako\dj e prikazuje implementirano re\v senje koriste\'ci prethodno opisane koncepte i formalne modele. Implementirano re\v senje mo\v ze koristiti kao samostalno re\v senje gde se kasnije mogu dodati potrebni podsistemi, ali tako\dj e pru\v za mogu\'cnost integracije u postoje\'ca re\v senja. Dati su primeri domena gde bi sistem mogao da se koristi zajdno sa primerima aplikacija gde bi korisnici imali benefit. Teza je organizovana u pet poglavlja.

U \textbf{poglavlju~\ref{chapter:Intro}} dali smo kratki uvod u temu distribuiranih sistema, sa fokusom na podru\v cja koja su va\v zna za razumevanje ove teze.

Pokazali smo \v sta su distribuirani sistemi ili bar op\v sti konsenzus kako neki sistem mo\v zemo opisati ili posmatrati kao distribuirani. Predstavili smo probleme koje ovi sistemi stvaraju i za\v sto ih je tako te\v sko implementirati, koristiti i odr\v zavati.

Tako\dj e smo predstavili nekoliko primera distribuiranih ra\v cunarskih aplikacija koje možemo koristiti da efikasno iskoristimo veliki broj \v cvorova u distribuiranom sistemu. Dalje smo predstavili \v sta je skalabilnost i za\v sto je ona va\v zna za distribuirane sisteme, sa nekoliko primera organizacionih mogu\'cnosti poput peer-to-peer mre\v za i protokola za opis grupa ili zajednica \v cvorova koji sara\dj uju, a koji su va\v zni u distribuiranom okru\v zenju iz razli\v citih razloga. Dali smo primere raznih varijanti ra\v cunarstva u oblaku koje mo\v zemo iskoristiti za svoje potrebe.

Zatim smo opisali nekoliko tehnika virtuelizacije koje se mogu koristiti za pakovanje i raspore\dj ivanje kako aplikacija tako i infrastrukture. Prikazali smo razne tehnike bitne za raspore\dj ivanje aplikacija i infrasktruture u okru\v zenju ra\v cunarstva u oblaku, ali i razliku između distriburanih sistema i nekoliko modela koji se \v cesto smatraju distribuiranim poput paralelnog ra\v i decentralizovanog ra\v cunarstva.

Na kraju smo dali opis motivacije za ovavku tezu sa izvedenim istra\v zivackim pitanjima i hipotezama na koja \v zelim da odgovorimo ovom tezom.

U \textbf{poglavlju~\ref{chapter:Review}} smo prikazale sli\c ne radove koje su radili razni drugi istra\v ziva\v ci ili kompanije. Isto kao i kod prethognog pogavlja, opet se fokusiramo samo na stvari koje su povezane sa ovom tezom.

Prikazali smo razli\v cite platforme, gde ljudi menjaju ili prilago\dj avaju postoje\'ca re\v senja kao \v sto su Kubernetes ili OpenStack da bi ih prilagodili da rade u oblastima poput ivi\v cnog ra\v cunarstva i mobilnog ra\v cunarstva. Dalje smo predstavili implementacije nekoliko platformi koje koriste \v cvorove koje su korisnici ponudili na dobrovoljnoj bazi, da bi se izvr\v sila nekakva obrada ili skladi\v stenje podataka na njima kao na primer drop computing i sistemima poput Nebule, izme\dj u ostalih.

Pokazali smo kako \v cvorovi mogu biti organizovani po geografskim podru\v cijima na zone ali kako mikro centri za obradu podataka mogu da pomognu ra\v cunarstvu u oblaku da slu\v zi zahteve lokalnog stanovni\v stva koje se nalazi u neposrednoj blizini isth. Dalje smo olisali razli\v cite tehnike kako se zadaci sa mobilnih ure\dj aja mogu prebaciti na ivi\v cne \v cvorove, ali tako\dj e i razli\v cite modele primene koji bi mogli iskoristiti ove tehnike.

Na kraju ovog poglavlja, predstavljamo gde je mesto ove teze u poređ\dj enju sa drugim sli\v cnim modelima i drugim sli\v cnim istra\v zivanjima.

\textbf{Poglavlje~\ref{chapter:Micro_clouds}} predstavlja sr\v z ove teze. Razdvajamo sve najvažnije aspekte koje treba da imamo kako bismo pomogli CC-u u vezi sa problemima sa kašnjenjem, velikim podacima sa ogromnim obimima podataka posebno u doba mobilnih uređaja i IoT-a.

Predlo\v zeni model zasnovan je na MDC-ima koji su zonsko organizovani i koji \'ce opslu\v zivati lokalno stanovni\v stvo i stanovni\v stvo u blizini. Predstavljamo model koji se zasniva na ra\v cunarstvu u oblaku, ali je prilago\dj en za druga\v ciji scenario i slu\v cajeve kori\v sc\'enja.

Pokazali smo kako mo\v zemo dinami\v cki da formiramo nove klastere, regione i topologije i kako ih možemo koristiti zajedno sa mobilnih ure\dj ajima i aplikacijama poput internet stvari (IoT). Ovaj novoformirani sistem ili sistem sistema oslanja se na jasan model podele nadle\v znosti, usvojen iz postojećih istra\v zivanja i prilago\dj en za novu troslojnu arhitekturu. Formirani model mo\v ze da slu\v zi kao sloj za obradu podatka na na samom izvori\v stu ili skoro na samom izvori\v stu, kao za\v stitni zid ili sloj za za\v stitu privatnosti. Predstavljeni sistem je izuzetno prilagodljiv u razli\v citim dimenzijama, odnosno potrebama i zahtevima.

Predstavljeni model mo\v ze biti ogroman kao cela dr\v zava, ili malen kao pojedina\v cno doma\'cinstvo i sve izme\dj u toga. Veli\v cina klastera je stvar dogovora, i mogu\'cnost je  izbora. Tako\dj e smo predstavili kako programeri mogu da iskoriste novu infrastrukturu, i koji sve modeli aplikacija mogu postojati, a tako\dj e kako administratori mogu rasporediti razvijene servise na novoformiranu infrastrukturu.

Pred kraj obog poglavlja, predstavljamo posledice ovog modela i kako se isti mo\v ze koristiti kao sastavni deo postoje\'cih sistema, na primer kao skladište informacija o \v cvorovima ili se mo\v ze koristiti kao novi model gde mo\v zemo razviti nove podsisteme i aplikacije. Predstavili smo protokole za stvaranje takvog sistema i modeliramo ih koriste\'ci formalne matemati\v cke metode ili knkretno, teoriju asinhronih tipova sesija. Sistem sledi formalni model i lako ga je pro\v siriti, kako formalno toko i prakti\v cno.

Na samom kraju pogavlja dajemo ograni\v cenja ovog sistema sistema i ujedno ove teze, i stvari kojih moramo biti svesni, \textbf{ako} takva tehnologija bude kori\v s\'cena u realnim situacijama.

U \textbf{poglavlju~\ref{chapter:Implementation}}, predstavljamo implementirani okvir zasnovan na znanju i istra\v zivanijma skupljenim iz prethodnih poglavlja. Tako\dj e smo detaljno opisali operacije koje se mogu obaviti u radnom okru\v zenju, i kako se implementirani model uklapa i gde mu je mesto u prethodno opisanom modelu podele nadle\v znosti.

Dalje predstavljamo rezultate na\v sih eksperimenata u kontrolisanom okru\v zenju, kao i to koja su ograni\v cenja implementiranog radnog okvira u trenutnoj fazi razvoja. Tako\dj e smo opisali i mogu\'ce primene ovog sistema, i gde bi ovaj model mogao da se koristi kada bi u\v sao u upotrebu.

\textbf{Poglavlje~\ref{chapter:Conclusion}} predstavlja poslednje poglavlje ove teze. U ovom poglavlju zaklju\v cili smo tezu sa onim što je ura\dj eno, \v sta mo\v ze da se uradi u budu\'cnosti u vidu budu\'cih pravaca istra\v zivanja.\\\\

\noindent
\textbf{Klju\v c re\v ci:} distribuirani sistemi, ra\v cunarstvo u oblaku, vi\v sestruko ra\v cunarstvo u oblaku, mikroservisi, softver kao servis, ivi\v cno ra\v cunarstvo, mikro ra\v cunarstvo u oblaku.