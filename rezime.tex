%!TEX root =  main.tex
\chapter*{Rezime}
\pagestyle{plain}

Razvojem i razli\v citom primenom softverskih i hardverskih sistema menja se i na\v cin na koji ljudi komuniciraju, u\v ce i realizuju svoje aktivnosti. Kao posledica sve ve\'ce primene ovih sistema u razli\v citim oblastima obim obrade i koli\v cina podataka znatno su se pove\'cali~\cite{ChiangZ16}. Ovo pove\'canje za posledicu je imalo sve ve\'cu upotrebu distribuiranih sistema da bi se ti poslovi mogli obaviti uspe\v sno.

Pro\v sirena stvarnost (AR), igre preko mre\v ze, automatsko prepoznavanje lica, autonomna vozila ili internet stvari (IoT) proizvode izuzetno velike koli\v cine podataka \v sto zna\v cajno uti\v ce na opterećenje kao i na ka\v snjenje~\cite{ChiangZ16}. Ovakvi zahtevi su izvan onoga \v sto se centralizovanim ra\v cunarskim modelima, poput ra\v cunarstva u oblaku, mo\v ze ponuditi~\cite{ChiangZ16}. \v Cak i mali problemi mogu dovesti do velikog zastoja u komunikaciji aplikacija i uslugama od kojih ljudi zavise.  

Primer koji se nedavno desio jo\v s jedan je u nizu otkaza na Amazon Web Services (AWS) platformi~\cite{GunawiHSLSAE16}. Ovim je platforma bila nedostupna korisnicima i aplikacijama, a kao rezultat nedostupnosti platforme velika koli\v cina aplikacija i servisa, koji se izvr\v savaju preko interneta, postaje potpuno nedostupna korisnicima i na kraju neupotrebljiva. 

%Da bi se razumelo kako smo do\v sli do tog problema, prvo se mora razumeti \v sta je zapravo ra\v cunarstvo u oblaku i videti kako se ovaj model organizuje.

Da bi se razumelo kako je problem nastao, potrebno je objasniti \v sta je zapravo ra\v cunarstvo u oblaku, predstaviti organizaciju i definisati arhitekturu ovog ra\v cunarskog modela.

Ra\v cunarstvo u oblaku definisano je kao skup ra\v cunarskih resursa koji se mogu ponuditi korisnicima kroz takozvani uslu\v zni softver~\cite{Vogels}. Hardver i softver u velikim centrima za obradu podataka pru\v zaju usluge svojim korisnicima preko interneta~\cite {AboveTheCloud}. Resursi poput CPU-a, GPU-a, skladi\v sta podataka i mre\v ze mogu se koristiti za nekakvu obradu podataka, ili osloboditi i to na zahtev i po potrebi korisnika~\cite{ZhangCB10}. 

Prednost ra\v cunarstva u oblaku su razni servisi koji su ponu\dj eni korisnicima kao usluge ili uslu\v zni softver~\cite{Vogels}. Tradicionalnim modelom ra\v cunarstva u oblaku pružaju se ogromni procesni i skladi\v sni resursi i to po potrebi, na zahtev korisnika, kako bi se podr\v zale razli\v cite potrebe aplikacija. Ovo svojstvo odnosi se na sposobnost ra\v cunarstva u oblaku da dozvoli korisnicima alokaciju dodatnih resursa ili oslobađanje postoje\'cih, kako bi se podudarali sa radnim optere\'cenjima aplikacija~\cite{AssuncaoVB18}. 

Jedna od problema u ovakvim sistemima nastaje kada je potrebno da se (veliki) podaci prebace sa svog izvori\v sta u oblak. Ovim dolazi do velike latencije ili ka\v snjenja u sistemu~\cite {HossainRH18}. Na primer, Boeing 787 generi\v se pola terabajta podataka po jednom letu, dok autonomni automobil generi\v se dva petabajta podataka tokom samo jedne vo\v znje~\cite{CaoZS18}. 

Me\dj utim, propusni opseg nije dovoljno veliki da bi podr\v zao takve zahteve~\cite{CaoZS18}. Prenos podataka nije jedini problem sa kojim se ra\v cunarstvo u oblaku susre\'ce. Aplikacije kao \v sto su autonomni automobili, bespilotne letelice ili balansiranje optere\'cenja u elektri\v cnim mre\v zama, zahtevaju obradu podataka u realnom vremenu da bi se ispravno donosile odluke i reagovalo na razne promene~\cite{CaoZS18}.

Centralizovana arhitektura ra\v cunarstva u oblaku, sa ogromnim kapacitetima centara za obradu podataka, stvara efikasno upravljanje resursima. Ovom strategijom dolazi se do smanjenja administrativnih tro\v skova celokupnog sistema~\cite{BariBEGPRZZ13}. Me\dj utim, kada takav sistem do\dj e do svojih granica, centralizacija uzrokuje mnoge druge probleme~\cite{GunawiHSLSAE16, LopezMEDHIBFR15}. Uprkos svim prednostima ovog modela, servisi i usluge vremenom se suo\v cavaju sa ozbiljnom degradacijom kvaliteta odziva i performansi usled velike propusnosti i ka\v snjenja~\cite{KarimIWGSYO16}. To mo\v ze dovesti do nesagledivih posledica po poslovanje, ali potencijalno i te\v ze efekte poput uticaja na ljudske \v zivote. 

Razne organizacije koriste usluge ra\v cunarstva u oblaku i oslanjaju se na njega kako bi izbegle izuzetno velike infrastrukturalne investicije~\cite {MonsalveCC18} poput pravljenja i odr\v zavanja sopstvenih centara za obradu podataka. Oni koriste resurse koje su obezbedili drugi pru\v zaoci usluga~\cite{Satyanarayanan17} i pla\'caju shodno tome koliko vremenski koriste usluge --- \emph{a pay as you go} model.

Cilj ove teze je predstavljanje i upotreba formalnih modela na osnovu kojih se mogu opisati, i formalno verifikovati protokoli, kao i implementirati radni okvir za distribuirani sistem, koriste\'ci geografski rasprostranjena okru\v zenja nalik ra\v cunarstvu u oblaku. Opisani sistem mogu koristiti ne samo obi\v cni korisnici, ve\'c ga i pru\v zaoci usluga ra\v cunarstva u oblaku mogu integrisati u svoju platformu i svoje servise kako bi se minimalizovao zastoj kriti\v cnih sistemskih segmenata. \v Citav sistem se mo\v ze posmatrati kao skup distribuiranih mikro oblaka ili sloj obrade podataka, koji u oblak \v salje samo va\v zne podatke, smajuju\'ci tro\v skove korisnicima, ali i obezbe\dj uju\'ci ve\'cu dostupnost usluga ra\v cunarstva u oblaku.

Distribuirani softverski sistemi su prilično slo\v zeni za modelovanje i implementaciju. Jedan od problema u ovakvim sistemima \v cesto nastaje zbog problema u komunikaciji \v cvorova preko mre\v ze koja nije sigurna i pouzdana. Poruke mogu da kasne, mogu da stignu u razli\v citom redosledu ili da ne stignu. Tako\dj e, \v cvorovi u sistemu mogu prestati sa radom potpuno nasumi\v cno stvaraju\'ci dodatne komplikacije. James Gosling i Peter Deutsch, kreirali su listu problema za mre\v zne aplikacije poznate kao \textit{8 zabluda distribuiranih sistema}~\cite{articleRotem}:

\begin{enumerate}[start=1,label={(\bfseries \arabic*)}]
	\item \textbf{Mre\v za je pouzdana.} Uvek \'ce se ne\v sto neplanirano desiti sa mre\v zom koja je prili\v cno nepouzdana - prekid napajanja, prekid kabla, problemi u okru\v zenju, itd.
	\item \textbf{Ka\v snjenje ne postoji.} Lokalno ka\v snjenje nije problem, ali se situacija vrlo brzo pogor\v sava kada se komunikacija oslanja na internet, i slu\v cajeve gde se koristi izuzetno kompleksna mre\v zna komunikacija ra\v cunarstva u oblaku.
	\item \textbf{Propusnost je beskona\v cna.} Iako se \v sirina propusnog opsega stalno pove\'cava i sve je bolja i bolja, srazmerno tome raste i koli\v cina podataka koja se prebacuje na obradu ili skladi\v stenje.
	\item \textbf{Mre\v za je sigurna.} Trendovi internet bezbednosti pokazuju izuzetno veliki rast napada, a ovo jo\v s vi\v se postaje problem u ra\v cunarstvu u oblaku javnog tipa.
	\item \textbf{Topologija se ne menja.} Mre\v zna topologija obi\v cno je izvan kontrole korisnika, a topologija mre\v ze stalno se menja usled brojnih razloga - dodati ili uklonjeni novi ure\dj aji, serveri, prekidi u komunikaciji itd.
	\item \textbf{Postoji samo jedan administrator.} Danas postoje brojni administrativi za veb servere, baze podataka, ke\v s memoriju i sli\v cno, ali, tako\dj e, kompanije sara\dj uju sa drugim kompanijama ili pru\v zaocima usluga ra\v cunarstva u oblaku.
	\item \textbf{Tro\v skovi transporta ne postoje.} Ova tvrdnja nikako nije ta\v cna iz prostog razloga \v sto moramo serijalizovati informacije i podatke koje \v saljemo \v sto dodatno tro\v si resurse i pove\'cava ukupno ka\v snjenje. Ovde nije problem samo u ka\v snjenju, ve\'c u tome \v sto svaka serijalizacija informacija zahteva dodatno vreme i dodatne resurse.
	\item \textbf{Mre\v za je homogena.} Danas je homogena mre\v za izuzetak, a ne pravilo. Postoje razli\v citi serveri, sistemi, klijenti koji komuniciraju. Implikacija ovoga je da, pre ili kasnije, mora se pretpostaviti da je potrebna interoperabilnost izme\dj u ovih sistema. Mogu se koristiti i za\v sti\'ceni protokol koji nisu javno dostupni, koji mogu dodatno tro\v siti vreme. Ovi protokoli mogu ostati bez podr\v ske, pa ih treba izbegavati.
\end{enumerate}

Iako razvoj distribuiranih sistema traje ve\'c nekoliko decenija, problemi koji se javljaju prilikom njihovog razvoja su i dalje identi\v cni. 

Prilikom razvoja distribuiranih sistema, programeri i dizajneri \v cesto zaboravljaju na definisane probleme, \v sto neretko dovodi do izuzetno velikih pote\v sko\'ca. Na\v cin da se to u ranim fazama otkrije jeste kori\v s\'cenje formalnih matemati\v ckih metoda za opisivanje i modelovanje ovih sistema. Ove metode sa\v cinjavaju razne tehnike koje slu\v ze za specifikaciju i verifikaciju kompleksnih sistema i koje su zasnovane na matemati\v ckim i logi\v ckim principima.

Kao odgovor na probleme koji mogu nastati kao posledica ka\v snjenja, nedostupnosti usluga u ra\v cunarstva u oblaku, malicioznih napada, ali i usled kvara nekog od resursa na mre\v zi~\cite{GunawiHSLSAE16}, nastala je nova paradigma tzv. ivi\v cno ra\v cunarstvo (eng. edge computing - EC)~\cite{Satyanarayanan17}. 

EC je model u kome se procesne i skladi\v sne mogu\'cnosti ra\v cunarstva u oblaku prebacuju u blizini izvora podataka~\cite{Satyanarayanan17}. Kao posledicu toga, ra\v cunarstvo u oblaku pro\v sireno je novim mogu\'cnostima. Smanjuje se ka\v snjenje \v sto onda dovodi do novih mogu\'cnosti za aplikacije budu\'ce generacije~\cite{NingLSY20}.   

Tokom prethodnih godina, pojavili su se razni modeli koji spu\v staju obradu i skladi\v stenje podataka bli\v ze izvori\v stu, poput fog ra\v cunarstva~\cite{BonomiMNZ14}, cloudlet-a~\cite {MonsalveCC18} i mobilnih ivi\v cnih ra\v cunara (MEC)~\cite{WangZZWYW17}. U ovoj disertaciji, svi pomenuti modeli se smatraju ivi\v cnim ra\v cunarstvom, a njihovi cvorovi ivi\v cnim \v cvorovima. 

Svi pomenuti modeli koriste koncept prenosa skladi\v snih i procesnih mogu\v cnosti iz oblaka bli\v ze izvorima podataka,~\cite{KhuneP19} dok su zahtevnije obrade i dalje zadr\v zane u oblaku iz vrlo prostog razloga --- dostupnost znatno ve\'ce koli\v cine resursa~\cite{NingLSY20}. EC modeli uvode male servere koji se arhitekturalno nalaze izme\dj u izvora podataka i oblaka. Tipi\v cno je za ove servere da imaju manje mogu\'cnosti u pore\dj enju sa serverima u oblaku~\cite{ChenHLLW15}. 

Prednost malih servera je u tome \v sto se oni mogu nalaziti na razli\v citim lokacijama, na primer u baznim stanicama~\cite{WangZZWYW17}, gradskim centralama, restoranima, bolnicama, \v skolama, kompanijama, ili mogu biti rasprostranjeni po geografskim regionima, a sve to kako bi se izbeglo ka\v snjenje i pove\'cala propusnost~\cite{MonsalveCC18}.   

Oni mogu poslu\v ziti kao za\v stitni sloj~\cite{SatyanarayananK19} ili kao nivo obrade pre nego \v sto podaci budu poslati u oblak. Sa druge strane, korisnici dobijaju jedinstvenu mogu\'cnost dinami\v cke i selektivne kontrole informacija koje bivaju poslate u oblak. 

Jo\v s jednu prednost ovih servera predstavili su Aroca i saradnici~\cite{ArocaG12}. Naime, njihovi rezultati su pokazali da mali serveri zadr\v zavaju dobre performanse prilikom pokretanja zahtevnih aplikacija i klasterskog okru\v zenja. Malo slabije performanse pokazali su u slu\v caju trenutno dostupnih skladi\v sta podataka, ali to mo\v ze biti podsticaj da se polje istra\v zivanja skladi\v sta podataka dopuni novim modelima, optimizovanim za male servere.

Jedan model te\v sko \'ce odgovarati potrebama svih aplikacija u budu\'cnosti, tako da ra\v cunarstvo u oblaku ne bi trebalo da bude na\v sa kona\v cna granica i jedina opcija. Razni modeli, nastali na bazi malih servera, pokazuju mogu\'cnost da se obrada podataka mo\v ze obaviti bli\v ze izvori\v stu, dok te\v ski prora\v cuni mogu ostati u oblaku zbog ve\'ce dostupnosti resursa. U oblak treba slati samo informacije koje su klju\v cne za druge usluge ili aplikacije~\cite{inproceedingsSimic1}, a ne sve kako predla\v ze standardni model oblaka. 

Ideja malih servera sa razli\v citim ra\v cunskim, skladi\v snim i mre\v znim resursima pokre\'ce zanimljive istra\v ziva\v cke ideje i, kao takva, motivacija je za ovu tezu. Kori\v s\'cenje resursa, koji su organizovani lokalno kao mikro oblaci, oblaci zajednice ili ivi\v cni oblaci,~\cite{RydenOCW14} predla\v zu Riden i saradnici.

Usled problema koji mogu nastati u doglednoj budu\'cnosti kori\v s\'cenjem sve ve\'ceg broja ra\v cunarskih sistema koji su povezani na internet, kao i  zbog ograni\v cenja ra\v cunarstva u oblaku u trenutnoj izvedbi, akademska zajednica kao i industrija po\v cele su da istra\v zuju i razvijaju odr\v ziva re\v senja. Neka istra\v zivanja vi\v se su usredsre\dj ena na prilago\dj avanje postoje\'cih re\v senja zahtevima EC-a, dok druga eksperimenti\v su sa novim idejama i re\v senjima.

U svom radu~\cite{GreenbergHMP09} Greenberg i saradnici isti\v cu da se mikro centri za obradu podataka ($\upmu$DCs) koriste prvenstveno kao \v cvorovi u mre\v zama za distribuciju sadr\v zaja i u drugim \say{lo\v se distribuiranim} aplikacijama. $\upmu$DC su zanimljiv model u podru\v cju brzih inovacija i razvoja. Autori uvode koncept $\upmu$DC-a kao centra za obradu podataka koji se nalazi u blizini velike populacije, smanjuju\'ci pritom fiksne tro\v skove tradicionalnih centara za obradu podataka. Samim tim, minimalna veli\v cina $\upmu$DC-a definisana je potrebama lokalnih korisnika~\cite{GreenbergHMP09, AbbasZTS18}, pru\v zaju\'ci agilnost kao klju\v cnu karakteristiku. Ovde agilnost podrazumeva sposobnost dinami\v ckog rasta i smanjenja potrebe za resursima kao i upotrebe resursa sa optimalne lokacije~\cite{GreenbergHMP09}. 

Satyanarayanan i saradnici u svom radu~\cite{SatyanarayananK19} pokazuju da $\upmu$DC-ovi mogu poslu\v ziti kao za\v stitni sloj. Simi\'c i saradnici u svom radu~\cite{inproceedingsSimic1}  opisuju takav sistem kao nivo obrade podataka na njihovom izvoru, dok korisnici dobijaju jedinstvenu mogu\'cnost dinami\v ckog i selektivnog upravljanja informacijama koje se \v salju u oblak. 

Zonska organizacija malih servera, koju su predstavili Guo i saradnici~\cite{GuoRG20} u primeni kod pametnih vozila daje zanimljivu perspektivu o EC-u. Autori su pokazali kako modeli koji dele oblast na zone omogu\'cavaju kontinuitet dinami\v ckih usluga i smanjuju primopredaju veze. Tako\dj e, pokazali su kako da se pokrivenost malim serverima prenese na ve\'cu zonu, \v cime se pro\v siruju ra\v cunarska snaga i kapacitet skladi\v stenja podataka.

EC poti\v ce iz peer-to-peer sistema, kako su to pretpostavili Lopez i saradnici, ali ga pro\v siruju u novim pravcima i pru\v zaju mogu\'cnost integracije sa ra\v cunarstvom u oblaku~\cite{LopezMEDHIBFR15}.

U svom radu, Kurniawan i saradnici~\cite{inbookKurniawan} pokazali su vrlo lo\v su skalabilnost u centralizovanim modelima mre\v za za isporuku sadr\v zaja(CDN) u oblaku. Autori su predlo\v zili decentralizovano re\v senje koriste\'ci nano centre za obradu podataka koje \v cine mre\v zni ure\dj aji u ku\'ci~\cite{inbookKurniawan}. Ovi centri za obradu podataka opremljeni su, tako\dj e, sa ne\v sto skladi\v snog prostora. Pokazana je mogu\'ca upotreba nano centara za obradu podataka \v cak i za neke velike primene sa jednom izuzetno bitnom predno\v s\'cu - mnogo manjom potro\v snjom energije.

$\upmu$DC-ovi sa zonskom organizacijom servera dobra su polazna osnova za izgradnju EC-a (koja mo\v ze biti ponu\dj ena kao servis korisnicima), ali i mikro ra\v cunarstva u oblaku jer se mo\v ze relativno jednostavno pro\v siriti ra\v cunarska snaga i skladi\v sni kapacitet koji opslu\v zuju lokalne korisnike. Me\dj utim, da bi se to postiglo, potreban nam je dostupniji i elasti\v cniji sistem sa manje ka\v snjenja. 

Dizajn ra\v cunarstva u oblaku je takav da svaki deo doprinosi otpornijem i skalabilnom sistemu. Regioni ili centri za obradu podataka izolovani su i nezavisni jedni od drugih, a takođe sadr\v ze resurse koji su potrebni aplikacijama za nesmetan rad. Regioni su sa\v cinjeni od nekoliko dostupnih zona~\cite{SouzaMFAK19} i ako neka od zona postane nedostupna, druge zone preuzimaju opslu\v zivanje korisni\v ckih zahteve i time ceo sistem, kao celina, nastavlja neometan rad. Po uzoru na ovakvu arhitekturu, EC bi mogli koristiti vrlo sli\v cnu strategiju formiraju\'ci model mikro ra\v cunarstva u oblaku ($\upmu$C), gde se mali serveri ili \v cvorovi grupi\v su u \emph{klastere}, a vi\v se klastera u ve'cu logi\v cku celinu nazvanu \emph{region}, pove\'cavaju\'ci dostupnost i pouzdanost sistema i njegovih aplikacija.

U tom kontekstu, pod $\upmu$C-om smatraju se geografski rasprostranjeni distribuirani sistemi koji organizaciono li\v ce na ra\v cunarstvo u oblaku, ali se nalaze u neposrednoj blizini korisnika i opslu\v zuju njihove zahteve. Ta razlika pru\v za druga\v ciju organizaciju nego u standardnom modelu ra\v cunarstva u oblaku. 

Koncept regiona u ra\v cunarstvu oblaka je fizi\v cki element~\cite{SouzaMFAK19}, dok se u $\upmu$C-u pojam \emph{region} mo\v ze koristiti za opisivanje skupova \emph{klastera} \v cvorova preko proizvoljne geografske oblasti. Regioni se sastoje od najmanje jednog klastera, ali mogu se sastojati i od vi\v se njih tako da se postigne otporniji, skalabilniji i dostupniji sistem. 

Da bi se osiguralo manje ka\v snjenje u sistemu, u normalnim okolnostima treba izbegavati veliku udaljenost izme\dj u klastera. U tradicionalnom modelu ra\v cunarstva u oblaku pro\v sirenje regiona zahteva fizi\v cko povezivanje novih modula sa ostatkom infrastrukture~\cite{Hamilton07}, \v sto mo\v ze izazvati nedostupnost tog regiona neko vreme.
 
U $\upmu$C-u regioni mogu prihvatiti nove ili osloboditi postoje\'ce klastere. Isto tako i klasteri mogu prihvatiti nove ili osloboditi postoje\'ce \v cvorove \emph{dinami\v cki} bez direktnog povezivanja novih modula. Vi\v se regiona \v cine novi logički sloj --– \emph{topologiju}. 

Topologija se sastoji od najmanje jednog regiona, a mo\v ze se prostirati i na vi\v se regiona. Prilikom dizajniranja topologije, posebno ako regioni treba da dele informacije ili da na neki na\v cin sara\dj uju, po\v zeljno je izbegavati veliku udaljenost izme\dj u regiona ako je to mogu\'ce, da bi se smanjilo ka\v snjenje u sistemu. 

Primenom  \emph{klastera}, \emph{regiona} i \emph{topologije} mogu\'ce je pokriti bilo koju geografsku oblast sa sposobno\v s\'cu da se smanje ili pro\v sire postoje\'ci \emph{klasteri}, \emph{regioni} pa \v cak i \emph{topologije}.  

Organizacija \emph{klastera, regiona i topologija} u $\upmu$C-u isklju\v civo je stvar potrebe korisnika, i kao takva sli\v cna je modeliranju u sistemima velikih podataka ~\cite{SonbolOAA20, WangCAL14}. Na primer, klasteri mogu obuhvatiti \v citav jedan grad ili manji koji obuhvataju ure\dj aji u jednom doma\'cinstvu i sve izme\dj u ova dva ekstrema. Grad bi mogao predstavljati jedan region, sa delovima grada koji su organizovani u klastere. Topologija grada mo\v ze se formirati tako \v sto se grad podeli na vi\v se regiona koji sadr\v ze vi\v se klastera. Topologija dr\v zave mo\v ze se formirati tako \v sto se kroz prirodne i/ili administrativne regije defini\v su regioni, iako je i bilo koja druga podela mogu\'ca.

\v Cvorovi unutar svakog klastera treba da izvr\v savaju neki od protokola za odr\v zavanje definicije klastera odnosno pripadnosti \v cvorova klasteru. Neki od \textit{Gossip} protokola poput \textit{SWIM}-a~\cite{DasGM02} mogu se koristiti u saradnji sa mehanizmima replikacije podataka~\cite {LiBCL20, CauCBFCEB16, CRDTS_Nuno} \v cine\'ci ceo sistem otpornijim na potencijalne gre\v ske. Treba prihvatiti \v cinjenicu da \'ce \v cvorovi u takvom sistemu iz raznih razloga biti nedostupni. To se ne mo\v ze izbe\'ci, ali se sistem mo\v ze projektovati tako da servisi ipak budu ipak budu dostupni koriste\'ci pritom neki od kopija servisa.

U modelu koji opisuje razne resurse kao usluge~\cite{DuanFZSNH15} EC i mikro ra\v cunarstvo u oblaku nalaze se izme\dj u CaaS-a i PaaS-a, u zavisnosti od potreba korisnika.

Prethodno definisan model $\upmu$C mogao bi se ponuditi kao usluga korisnicima kao i bilo koji drugi resurs ra\v cunarstva u oblaku. U slu\v caju kada je potrebno vi\v se resursa na jednoj strani, mogu\'ce ih je reorganizovati da se iskoriste gde su oni stvarno potrebni. Sa druge strane, kompanije koje pru\v zaju usluge ra\v cunarstva u oblaku mogu integrisati predlo\v zeni model u svoj postoje\'ci sistem, skrivaju\'ci nepotrebnu slo\v zenost iza nekog komunikacionog interfejsa ili predlo\v zenog modela aplikacije ili usluge.

Da bi se i EC modeli mogli iskoristiti na ovaj na\v cin potrebna je jasna dinami\v cka organizacija geografski raspore\dj enih \v cvorova, dobro definisan model aplikacija i jasno razdvajanje nadle\v znosti u sistemu. Kao takvi, bili bi jako slo\v zeni da se ponude kao usluga korisnicima. 

EC sistemi obi\v cno postoje nezavisno jedni od drugih, rasuti bez me\dj usobne povezanosti i saradnje. Nude ih pru\v zaoci usluga koji korisnike uglavnom ograni\v cavaju na usluge unutar sopstvenog ekosistema, \v cesto bez mogu\'cnosti izbora servisa van njihovog kataloga usluga. 

Grupisani čvorovi treba da budu organizovani lokalno, \v cine\'ci sistem kompletnim, a aplikacije dostupnijim i pouzdanijim, pro\v siruju\'ci resurse izvan pojedina\v cnog \v cvora ili male grupe \v cvorova. Takav sistem treba da odr\v zava dobre performanse za izgradnju servera i klastera~\cite{ArocaG12}.

Da bi se postiglo takvo pona\v sanje, neophodno je imati dinami\v cko upravljanje resursima i upravljanje ure\dj ajima. Potrebno je uvek imati dostupne informacije o resursima, konfiguraciji i zauzetosti \v cvorova~\cite{GubbiBMP13, WangZZWYW17} i klastera u celini. Tradicionalni centri za obradu podataka predstavljaju dobro organizovan i povezan sistem. Sa druge strane, $\upmu$DC-ovi se sastoje od razli\v citih ure\dj aja koji to nisu~\cite{JiangCGZW19}. Ovaj problem dovodi nas do problema kojim se bavi ova teza.

Da bi opisali fizi\v cke usluge, Jin~\cite {JinCJL14} i saradnici predla\v zu tri osnovna koncepta i preciziraju njihove odnose. Ovi koncepti su: \textbf{(1)} ure\dj aji, \textbf{(2)} resursi i \textbf{(3)} servisi. 

Podela nadle\v znosti bitan je deo svakog sistema, posebno ako se stvara platforma koja se nudi korisnicima kao usluga. Model podele nadle\v znosti, koji ova teza predla\v ze, zasnovan je na ovim konceptima, prilago\dj en druga\v cijem slu\v caju kori\v s\'cenja i  podeljen u tri sloja \v: \textbf{(1)} ure\dj aje, \textbf{(2)} resurse, i \textbf{(3)} servise.

Prvi sloj \v cine razli\v citi ure\dj aji odnosno kreatori podataka i korisnici usluga odnosno servisa. Drugi sloj predstavlja resurse, koji imaju prostorne karakteristike i ukazuju na mogu\'cnosti za obradu odnosno skladi\v stenje podataka \v cvorova na kojima se izvr\v savaju~\cite{JinCJL14}. Programeri u bilo kom trenutku moraju znati iskori\v s\'cenost resursa kao i stanje i dostupnost aplikacija. 

Resursi predstavljaju EC \v cvorove i, da bi \v cvor bio deo sistema, mora zadovoljiti \v cetiri jednostavna pravila: 

\begin{enumerate}[start=1,label={(\bfseries \arabic*)}]
\item Mora biti sposoban da pokrene operativni sistem sa sistemom datoteka;
\item Mora biti u mogu\'cnosti da pokrene neki od dostupnih alata za izolaciju aplikacija, na primer \textit{container} ili \textit{unikernel}; 
\item Mora imati dostupne resurse za kori\v s\'cenje (npr. CPU, GPU, disk itd.);
\item Mora imati internet vezu.
\end{enumerate}

Tre\'ci sloj cine servisi koji pruzaju resurse aplikacijama putem definisanog interfejsa i \v cine ih dostupnim preko interneta~\cite {JinCJL14}. Oni odmah odgovaraju na klijentske zahteve, ako je to mogu\'ce, ili mogu da skladi\v ste prona\dj enu informaciju za neke budu\'ce korisni\v cke upite~\cite {SatyanarayananBCD09, YaoXWYZP20}. Servisi koji se izvr\v savaju u oblaku treba da budu u stanju da prihvate unapred obra\dj ene podatke i odgovorni su za obradu i skladi\v stenje podataka \v ciji kapacitet prevazilazi mogu\'cnosti EC \v cvorova. Ovi servisi tako\dj e treba da budu zamenska opcija u slu\v caju da prethodno definisani sistem bude nedostupan iz bilo kog razloga.

Razdvajanjem nadle\v znosti modela i objedinjenjem organizacije \v cvorova, te\v zi se ka pristupu EC-a kao usluge i mogu\'cnosti dinami\v ckog formiranja distribuirnaih mikro oblaka koji bi mogli da obrade podatke na samom njihovom izvoru.

Me\dj utim, infrastruktura za takav sistem ne\'ce biti potpuno efikasna sve dok proces konfigurisanja i kori\v s\'cenja ne bude zna\v cajno pojednostavljen~\cite{SatyanarayananBCD09}. Ru\v cno pode\v savanje \v cvorova predstavlja naporan i dugotrajan proces, naro\v cito kada se uzme u obzir geografska rasprostranjenost dostupnih \v cvorova. 

Model koji se predla\v ze u ovoj tezi re\v sava gorepomenuti problem pomo\'cu  dinami\v ckog pode\v savanja i formiranja klastera, regiona i topologija i oslanja se na \v cetiri protokola:

\begin{enumerate}[start=1,label={(\bfseries \arabic*)}]
	\item \textbf{Provera stanja \v cvora} - protokol obave\v stava sistem o stanju svakog \v cvora; 
	\item \textbf{Formiranje klastera} - protokol formira nove klastere, regione i topologije;
	\item \textbf{Provera idempotencije} - protokol proverava da li klaster, region ili topologija postoje, i da li je potrebno pokrenuti protokol za formiranje;
	\item \textbf{Pregled detalja} - protokol prikazuje trenutno stanje sistema korisniku kroz razne nivoe detalja.
\end{enumerate}

Za formalni opis server ili \v cvorovi (u disertaciji ovi pojmovi se koriste naizmeni\v cno) u sistemu, kori\v s\'cena je teorija skupova. Prethodno definisane protokole mogu\'ce je formalno modelirati upotrebom~\cite{HuY17} pro\v sirenje \emph{multiparty asynchronous session types} (MPST)~\cite {HondaYC08} - klasa tipova pona\v sanja formirana za opisivanje distribuiranih protokola oslanjaju\'ci se na asinhrone komunikacije.

Ova matemati\v cka terorija se mo\v ze iskoristitit i za verifikaciju, da li definisani protokoli zadovoljavaju MPST sigurnost (nema dostupnog stanja gre\v ske) i napredak (akcija se na kraju izvr\v sava, pod pretpostavkom po\v stenja).

Proces modelovanja odvija se u dva koraka:

\begin{enumerate}[start=1,label={(\bfseries \arabic*)}]
	\item \textbf{Prvi korak} u modeliranju komunikacija sistema pomo\'cu MPST teorije predstavlj definisanje \emph{globalnog tipa}, \v sto predstavlja globalni opis celokupnog protokola sa neutralne ta\v cke posmatranja.
	\item \textbf{Drugi korak} u modeliranju komunikacija sistema pomo\'cu MPST teorije je pru\v zanje sintaksi\v cke projekcije protokola na svakog u\v cesnika u komunikaciji iskazane kao \emph{lokalni tip}, koji se zatim koristi za proveru tipa i implementacije krajnje ta\v cke.
\end{enumerate}

Na osnovu prethodno opisanih ideja i mogu\'cnosti, defini\v se se problem koji ova teza obra\dj uje kroz slede\'ca tri istra\v ziva\v cka pitanja:

\begin{enumerate}[start=1,label={(\bfseries \arabic*)}]\label{rez:questions}
	\item Kako se mo\v ze definisati formalan pro\v siriv model sa jasnom podelom nadle\v znosti, po ugledu na organizaciju ra\v cunarstva u oblaku, koji bi bio prilago\dj en druga\v cijem okru\v zenju izvr\v savanja sa jasnim aplikativnim modelom koji \'ce moći da iskoristi novu, prilago\dj enu arhitekturu?
	\item Kako je mogu\'ce ovako organizovane \v cvorove predstaviti kao uslugu korisnicima za razvoj budu\'cih aplikacija, po poznatom modelu naplate po utro\v sku (eng. pay as you go)?
	\item Kako se mo\v ze definisati formalno ispravan model koji \'ce biti lak za pro\v sirenje?
\end{enumerate}

Ako su prethodna istra\v ziva\v cka pitanja potvrdna, onda pro\v sirenje nalik na oblak pro\v siruje resurse van granica pojedina\v cnog \v cvora \v sto \v citav sistem, kao i same aplikacije koje bi se izvr\v savale u njemu, \v cini dostupnijim i pouzdanijim.

Na osnovu prethodno definisanih istra\v ziva\v ckih pitanja i motivacija, definisane su sledece hipoteze:

\begin{enumerate}[start=1,label={(\bfseries \arabic*)}]
	\item \textbf{Hipoteza:} \textit{Mogu\'ce je organizovati \v cvorove na standardni na\v cin, zasnovan na arhitekturi ra\v cunarstva u oblaku i prilago\dj en druga\v cije rasprostranjenom geografskom okru\v zenju pru\v zaju\'ci korisnicima mogu\'cnost da na najbolji mogu\'ci na\v cin organizuju \v cvorove i klastere po raznim geografskim oblastima kako bi opslu\v zivali samo lokalne korisnike u neposrednoj blizini --- model distribuiranog mikro okru\v zenja ra\v cunarstva u oblaku.}
	\item \textbf{Hipoteza:} \textit{Model distribuiranog mikro okru\v zenja ra\v cunarstva u oblaku se mo\v ze iskoristiti za upravljanje infrastrukturnim resursima u blizini korisnika. Ako je potrebno vi\v se resursa na jednoj strani, mo\v ze se uzeti odre\dj ena koli\v cina resursa i reorganizovati u skladu sa potrebama sistema, ili se mogu reorganizovati na bilo koji drugi \v zeljeni na\v cin.}
	\item \textbf{Hipoteza:} \textit{Mogu\'ce je predstaviti jasnu podelu nadle\v znosti za budu\'ci sistem, koji bi se pru\v zio korisnicima kao usluga, i uspostaviti dobro organizovan sistem u kojem svaki deo ima intuitivnu i jasnu ulogu.}
	\item \textbf{Hipoteza:} \textit{Mogu\'ce je predstaviti objedinjeni model, koji podr\v zava heterogene \v cvorove, sa jasnim setom tehni\v ckih zahteva koje budu\'ci \v cvorovi moraju ispuniti ako \v zele da postanu deo sistema.}
	\item \textbf{Hipoteza:} \textit{Mogu\'ce je predstaviti jasan aplikativni model, intuitivan korisnicima, kako bi se mogao iskoristiti puni potencijal predstavljenog re\v senja.}
\end{enumerate}

Iz prethodno definisanih hipoteza izvode se primarni ciljevi ove teze pri \v cemu o\v cekivani rezultati uklju\v cuju slede\'ce:

\begin{enumerate}[start=1,label={(\bfseries \arabic*)}]
	\item \textit{Definisanje formalnog pro\v sirivog modela sa jasnom podelom nadle\v znosti, po ugledu na organizaciju ra\v cunarstva u oblaku, koji bi bio prilago\dj en okru\v zenju gde se obrada i skladi\v stenje vrše bli\v ze korisnicima sa jasnim aplikativnim modelom koji \'ce mo\'ci da iskoristi novu prilago\dj enu arhitekturu. Ovaj cilj odnosi se na prvo istra\v ziva\v cko pitanje, a definisano je kroz poglavlje \ref{chapter:Micro_clouds}.}
	\item \textit{Definisani model je dostupniji i elasti\v can sa manje ka\v snjenja u pore\dj enju sa pojedina\v cnim malim serverima i mo\v ze se koristiti kao bilo koja druga usluga u oblaku. Ovaj cilj odnosi se na prvo istra\v ziva\v cko pitanje, a definisano je kroz poglavlja \ref{chapter:Micro_clouds} i \ref{chapter:Model usability}.}
	\item \textit{Implementacija i verifikacija prototipa kojim bi se potvrdila prakti\v cna primenljivost navedenog modela i identifikovale sve njegove prednosti i eventualni nedostaci. Ovo se odnosi na tre\'ce istra\v ziva\v cko pitanje i tema je poglavlja \ref{chapter:Micro_clouds} i \ref{chapter:Implementation}.}
\end{enumerate}

\noindent
Ovim istra\v zivanjem predstavlja se jedno mogu\'ce re\v senje za organizaciju geografski rasprostranjenog $\upmu$C-a sa EC \v cvorovima, zasnovano na organizaciji ra\v cunarstva u oblaku, prilago\dj eno obradi i skladi\v stenju podataka u blizini korisnika, a organizovano kroz tri koncepta: \emph{topologiju, region i klaster}. 

Primenom ovih koncepata mogu\'ce je opisivanje bilo kog geografskog podru\v cja u cilju dobijanja dostupnijeg i pouzdanijeg sistema. Organizacija i reorganizacija ovih elemenata vr\v si se dinami\v cki, opisom \v zeljenog stanja sistema, a veli\v cina regiona, klastera i topologija odre\dj uje se prema potrebama klijenata koje se opslu\v zuje.

Ishod istrazivanja je pro\v sirenje nalik na ra\v cunarstvo u oblaku koje pro\v siruje resurse van granica pojedina\v cnog \v cvora \v sto \v citav sistem kao i same aplikacije koje bi se izvr\v savale u okviru njega \v cini dostupnijim i pouzdanijim. Primarna primena jeste u organizaciji geografski distribuiranih ra\v cunarskih resursa na efikasan na\v cin tako da opslu\v zuje korisnike u neposrednoj blizini. 

Prototipsko re\v senje bazirano na prethodnom modelu se razvija kao alat otvorenog koda tako da je njegova primena mogu\'ca u razli\v citim sistemima od strane drugih istra\v ziva\v ckih ili razvojnih timova. Tako\dj e, mo\v ze se koristiti kao samostalno re\v senje tamo gde se kasnije mogu dodati potrebni podsistemi, ali tako\dj e pru\v za mogu\'cnost integracije u postoje\'ca re\v senja.

Predstavljeno re\v senje se mo\v ze koristiti kao pomo\'c ra\v cunarstvu u oblaku prilikom obrade velike kori\v cine podataka, kao sloj koji bi vr\v sio preliminarnu obradu na samom izvori\v stu podataka. Skladi\v stenje i obrada se vr\v si samo za podatke od zna\v caja, \v sto je jako bitno za tipove aplikacija koje bi se izvr\v savale u realnom vremenu zato \v sto sistem lokalno mo\v ze da reaguje znatno br\v ze nego udaljeni sistem ra\v cunarstva u oblaku.

Predstavljeno je preslikavanje ra\v cunarstva u oblaku na EC, i prikazan je formalni model sistema sa jasnom podelom nadle\v znosti i modelom aplikacije za budu\'ci EC koji bi bio ponu\dj en korisnicima kao usluga. Dati su primeri domena gde bi sistem mogao da se koristi zajedno sa primerima aplikacija od kojih bi korisnici imali benefit. 

\noindent
Teza je organizovana u pet poglavlja.

U \textbf{poglavlju~\ref{chapter:Intro}} dat je opis motivacije sa jasno definisanim istra\v ziva\v ckim pitanjima i hipotezama na koje \v zelimo da odgovorimo ovom tezom.

U \textbf{poglavlju~\ref{chapter:Field_overview}} predstavljen je kratak uvod u temu distribuiranih sistema, sa fokusom na podru\v cja koja su va\v zna za razumevanje ove teze i svih njenih delova.

Pokazano je \v sta su distribuirani sistemi ili kako neki sistem mo\v zemo opisati ili posmatrati kao distribuirani sistem. Predstavili smo probleme koje ovi sistemi stvaraju i za\v sto ih je tako te\v sko implementirati, koristiti i odr\v zavati.

Tako\dj e, predstavljeno je nekoliko primera distribuiranih ra\v cunarskih aplikacija koje mo\v zemo primeniti za efikasno iskori\v s\'cavanje velikog broja \v cvorova u distribuiranom sistemu. Dalje, pokazano je \v sta je skalabilnost i za\v sto je ona va\v zna za distribuirane sisteme sa nekoliko primera organizacionih mogu\'cnosti, poput peer-to-peer i master-slave sistema, kao i protokola za opis grupa ili zajednica \v cvorova koji sara\dj uju, a koji su va\v zni u distribuiranom okru\v zenju iz razli\v citih razloga. Dati su primeri raznih varijanti ra\v cunarstva u oblaku koje mo\v zemo iskoristiti za svoje potrebe.

Zatim, pokazano je nekoliko tehnika virtuelizacije koje se mogu koristiti i raspore\dj ivanje kako aplikacija, tako i infrastrukture. Prikazane su razne tehnike bitne za raspore\dj ivanje aplikacija i infrastrukture u okru\v zenju ra\v cunarstva u oblaku, ali i razliku izme\dj u distriburanih sistema i nekoliko modela koji se \v cesto smatraju distribuiranim poput paralelnog i decentralizovanog ra\v cunarstva.

Poglalvje \textbf{poglavlju~\ref{chapter:Review}} daje pregled relevatne literature iz oblasti disertacije.

Prikazali smo razli\v cite platforme, gde autori menjaju ili prilago\dj avaju postoje\'ca re\v senja (kao \v sto su Kubernetes ili OpenStack) da rade u oblastima poput ivi\v cnog ra\v cunarstva i mobilnog ra\v cunarstva. Dalje smo predstavili implementacije nekoliko platformi koje koriste \v cvorove, a koje su korisnici ponudili na dobrovoljnoj bazi, da bi se izvr\v sila nekakva obrada ili skladi\v stenje podataka na njima, kao na primer drop computing i Nebule izme\dj u ostalih.

Pokazali smo kako \v cvorovi mogu biti organizovani po geografskim podru\v cjima na zone, ali i kako mikro centri za obradu podataka mogu da pomognu ra\v cunarstvu u oblaku da prihvata zahteve lokalnih klijenata koji koriste resurse u neposrednoj blizini. Dalje smo opisali razli\v cite tehnike kako se zadaci sa mobilnih ure\dj aja mogu prebaciti na ivi\v cne \v cvorove, ali tako\dj e i razli\v cite modele primene koji bi mogli iskoristiti ove tehnike.

Na kraju ovog poglavlja dato je pore\dj ejne rezultata i doprinosa istrazivanja ove teze u odnosu na slicna istra\v zivanja.

\textbf{Poglavlje~\ref{chapter:Micro_clouds}} \v cini sr\v z ove teze. U ovom poglavlju razdvojili smo sve najva\v znije aspekte koje treba da zadovoljimo kako bi se omogu\'cilo re\v savanja problema kao \v sto su ka\v snjenje i obrada podataka, posebno u doba mobilnih ure\dj aja i IoT-a.

Predlo\v zeni model zasnovan je na $\upmu$DC-ima koji su zonski organizovani i koji \'ce opslu\v zivati lokalne korisnike ili korisnike u blizini. Predstavili smo model koji se zasniva na ra\v cunarstvu u oblaku, ali je prilago\dj en za druga\v ciji scenario i sli\v cne slu\v cajeve kori\v s\'cenja.

Pokazali smo kako mo\v zemo dinami\v cki formirati nove klastere, regione i topologije i kako ih mo\v zemo koristiti zajedno sa mobilnim ure\dj ajima i aplikacijama poput internet stvari (IoT). Ovaj novoformirani sistem oslanja se na jasan model podele nadle\v znosti, usvojen iz postoje\'cih istra\v zivanja i prilago\dj en za novu troslojnu arhitekturu. Formirani model mo\v ze da slu\v zi kao sloj za obradu podataka na samom izvori\v stu ili skoro na samom izvori\v stu, kao sloj za za\v stitu privatnosti korisnika i kontrolu sadr\v zaja koji se \v salje u oblak. Predstavljeni sistem izuzetno je prilagodljiv i podlo\v zan pro\v sirivanju prema razli\v citim dimenzijama, odnosno potrebama i zahtevima.

Predstavljeni model mo\v ze da obhvata ve\'ce ili manje geografske regione. Predstavili smo kako programeri mogu iskoristiti novu infrastrukturu i koji sve modeli aplikacija mogu postojati, ali i kako administratori mogu rasporediti razvijene servise na novoformiranu infrastrukturu koriste\'ci opisni ili deskriptivni model, umesto eksplicitnog slanja komandi i koraka sistemu.

Pred kraj ovog poglavlja, prikazano je kako se isti mo\v ze koristiti kao sastavni deo postoje\'cih sistema (kao skladi\v ste informacija o \v cvorovima) ili se mo\v ze koristiti kao novi model u kom mo\v zemo razviti nove podsisteme. Predstavili smo protokole za stvaranje takvog sistema i modelirali ih koriste\'ci formalne matemati\v cke metode ili, konkretno, teoriju asinhronih tipova sesija.

U \textbf{poglavlju~\ref{chapter:Implementation}} pokazali smo implementirani zasnovan na prethodno predstavljenom modelu. Ovde smo tako\dj e detaljno opisali operacije koje se mogu obaviti u prototipu, ali i kako se implementirani model uklapa i gde mu je mesto u prethodno opisanom modelu podele nadle\v znosti.

Dalje smo izneli rezultate na\v sih eksperimenata u kontrolisanom okru\v zenju kao i ograni\v cenja implementiranog radnog okvira u trenutnoj fazi razvoja. Tako\dj e, opisali smo mogu\'ce primene ovog sistema, ali i to gde bi ovaj model mogao da se koristi kada bi u\v sao u upotrebu.

\textbf{Poglablje}~\ref{chapter:Model usability} prikazuje upotrebljivost predlo\v zenog modela, sa mogu\'cim scenarijima primene. Predstavljen je i jedan konkretan primer kontrole saobra\'caja i usmeravanje ambulantih vozila i medicinskog osoblja do najbli\v ze bolnice u gradu Milanu, Italija, zahva\'cenim COVID-19 virusom, i kako bi predlo\v zeni sistem mogao da se iskoristi da pomogne medicinskom osoblju pri le\v cenju pacijenata, ali i istra\v ziva\v cima u boljem razumevanju samog virusa.

\textbf{Poglavlje~\ref{chapter:Conclusion}} predstavlja poslednje poglavlje ove teze. U ovom poglavlju sumirali smo doprinose ove teze, dali smo ograni\v cenja predlo\v zenog modela i ,ujedno, ove teze, ali i svega onoga \v cega moramo biti svesni \textbf{ako} ako takva tehnologija bude kori\v s\'cena u realnim situacijama. Na samom kraju poglavlja dali smo pregled \v sta se mo\v ze uraditi u pogledu budu\'cih pravaca istra\v zivanja.\\\\

\noindent
\textbf{Klju\v c re\v ci:} distribuirani sistemi, ra\v cunarstvo u oblaku, vi\v sestruko ra\v cunarstvo u oblaku, mikroservisi, softver kao servis, ivi\v cno ra\v cunarstvo, mikro ra\v cunarstvo u oblaku, veliki podaci, infrastruktura kao kod.