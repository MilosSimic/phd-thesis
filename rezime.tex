%!TEX root =  main.tex
\chapter*{Rezime}
\pagestyle{plain}
%

Rasprostranjenost distribuiranih softverskih sistema razli\v citih namena promenila je na\v cin na koji ljudi komuniciraju, sti\v cu znanja i vode biznise: skoro svi aspekti ljudskog \v zivota postali su povezani sa internetom. Ovaj sistem me\dj usobno povezanih ra\v cunarskih instanci napravio je veliki pozitivan uticaj na svakodnevni \v zivot, od brze i jednostavne komunikacije putem dru\v stvenih mre\v za i ra\v cunarskih platformi dostupnih putem interneta, do distribuiranih sistema za pla\' canja i kriptovaluta zasnovanih na blockchain tehnologijama.  
Me\dj utim, deljenje informacija, prava pristupa bazama podataka i prava pristupa ra\v cunarskim platformama, kao i deljenje drugih resursa otvara i nove probleme, me\dj u kojima su pitanja bezbednosti, pristupa\v cnosti i dostupnosti. Postoji veliki broj primera u kojima su napada\v ci (krakeri) uspeli da zloupotrebe previde programera koji su razvijali sisteme. Jedan takav skoriji primer je i gre\v ska koja je omogu\' cila nepravilno generisanje tokena za pristup li\v cnim profilima na Facebook-u. Tu gre\v sku je za sada nepoznati napada\v c uspeo da iskoristi da bi do\v sao do li\v cnih podataka sa skoro 50 miliona naloga~\cite{fb_attack}.
Takvi primeri jasno ukazuju na probleme kontrole deljenja i kori\v s\' cenja resursa u distribuiranim softverskim sistemima, problemima kojima se ova teza bavi kori\v s\' cenjem formalnih metoda.

Pouzdanost distriburanih softverskih sistema mo\v ze zavisiti od velikog broja faktora i \v cesto nije lako \v cak ni definisati \v sta se pod pouzdano\v s\' cu odre\dj enog sistema podrazumeva. Jedan od mogu\' cih pristupa kod dizajna i verifikacije takvih sistema je kori\v s\' cenje formalnih, matemati\v cki zasnovanih metoda. Formalne metode predstavljaju tehnike i alate za specifikaciju i verifikaciju kompleksnih (softverskih i hardverskih) sistema zasnovane na matemati\v ckim i logi\v ckim principima. Formalni dizajn obuhvata dve faze: formalnu specifikaciju i verifikaciju. U fazi formalne specifikacije (modeliranja) defini\v se se sistem koriste\' ci jezik modeliranja, naj\v ce\v s\' ce koriste\' ci preciznu matemati\v cku sintaksu i semantiku. Razvijaju\' ci formalnu specifikaciju, uglavnom nastaje i skup teorema koje opisuju osobine tog sistema. U fazi verifikacije, ove teoreme se precizno matemati\v cki dokazuju. %U posednjoj fazi se pristupa implementaciji tako \v sto se formalni model prevodi u programski kod. 
U konkurentnom ra\v cunarstvu, neki od poznatih formalnih modela koji se koriste za specifikaciju i verifikaciju osobina sistema su Petrijeve mre\v ze~\cite{petri1962kommunikation, DBLP:books/daglib/0032298}, komuniciraju\' ci automati sa kona\v cnim brojem stanja~\cite{DBLP:journals/jacm/BrandZ83} i procesni ra\v cuni~\cite{ DBLP:books/ph/Hoare85, DBLP:books/sp/Milner80, DBLP:journals/iandc/MilnerPW92a, DBLP:journals/iandc/MilnerPW92b}.


Cilj ove teze je da predstavi dve specifikacije zasnovane na dva procesna ra\v cuna koje tretiraju neke aspekte bezbednosti i kontrole pristupa u distribuiranim sistemima. Kona\v cni cilj je stvoriti uslove za bolje razumevanje koncepata izu\v cavanih u ovom radu i omogu\' citi njihovu kasniju ispravnu implementaciju.

Komunikacija putem distribuiranih sistema, ponekad uklju\v cuju\' ci interakcije sa nepoznatim i nepouzdanim korisnicima, je postala svakodnevna, a u nekim slu\v cajevima \v cak i nezaobilazna rutina. U mnogim situacijama razmenjena informacija je privatna i zahteva pa\v zljivo rukovanje i kori\v s\' cenje. Na primer, osetljivi privatni podaci, kao \v sto su broj kreditne kartice ili adresa, moraju biti otkriveni tokom kupovine putem interneta, ali sa druge strane ove informacije ne bi smele biti dalje deljene od strane korisnika ili aplikacije koji prima informaciju. Ovakvi primeri ukazuju na probleme kontrole deljenja informacija u distribuiranim sistemima. Dakle, deljenje informacija sa tre\' cim licima mo\v ze dovesti do ne\v zeljene diseminacije. \v Cak i u slu\v cajevima kada se korisnicima generalno mo\v ze verovati postoji mogu\' cnost previda koji mogu dovesti do zloupotreba.
 
Problem privatnosti mo\v ze i mora biti sagledan sa strane tehnologije ali i prava. Jedan od pionira koji su prou\v cavali privatnost iz obe perspektive je pravnik Alan Westin. On je primetio ,,da \' ce integracija kontrola privatnosti u nove tehnologije zahtevati sna\v zan napor...''~\cite{westin2003social}.
Iako nove tehnologije donose nove pretnje za kontrolu privatnosti, one mogu doneti i nove na\v cine za za\v stitu privatnosti~\cite{DBLP:conf/fm/TschantzW09}. 
Solove~\cite{solove2005taxonomy} uvodi taksonomiju i navodi \v cetiri vrste naru\v savanja privatnosti: sakupljanje informacija, invazija, diseminacija i obra\dj ivanje informacija. Nedovoljna kontrola nad deljenjem informacija u distribuiranim sistemima mo\v ze biti direktno povezana sa diseminacijom.


Solove daje dalju taksonomiju naru\v savanja privatnosti putem diseminacije, ali sve ove podvrste uglavnom prepoznaju \v stetu koja mo\v ze nastati kod otkrivanja i deljenja osetljivih informacija. Komunikacija me\dj u u\v cesnicima je centralni aspekt distribuiranih sistema, a kontrola protoka informacija u takvim sistemima \v cesto ima svoje pote\v sko\' ce. Entiteti u takvim sistemima mogu imati razli\v cita prava za manipulaciju odre\dj enim informacijama. Na primer, %u odnosu na informacije vezane za jedan bankovni ra\v cun mo\v zemo razlikovati da 
korisnik bankovnog ra\v cuna ima ovla\v s\' cenja da koristi broj kartice za pla\' canja putem interneta, mo\v ze povla\v citi odre\dj ena sredstava sa bankovnog ra\v cuna, itd. Sa druge strane, kod isplate banka mo\v ze izvr\v siti uvid u stanje kako bi proverila da li postoji dovoljno sredstava na ra\v cunu. Ako se za trenutak fokusiramo na kontrolu protoka informacija, mo\v zemo uo\v citi da bi broj kreditne kartice trebalo da mo\v ze poslati samo korisnik te kartice, ali ne i banka koja prima tu informaciju. To jest, banka ne bi trebalo da ima prava da prosle\dj uje informaciju u ovom slu\v caju. Za naru\v savanje diseminacije informacija, prosle\dj ivanje mo\v ze biti prepoznato kao jedna od glavnih meta gde kontrola mora biti uspostavljena.



Ako razmatramo prava koja entitet mo\v ze imati u odnosu na komunikacioni kanal, mo\v zemo razlikovati prava na kori\v s\' cenje kanala za slanje i \v citanje, pravo da se kreira novi kanal i da se po\v salje jedan njegov kraj drugom korisniku, prava da se prosle\dj uju primljena imena kanala, itd. Davanje prava o prosle\dj ivanju imena kanala svim entitetima apriori mo\v ze kasnije prouzokovati pote\v sko\' ce oko kontrole diseminacije, jer u tom slu\v caju kontrola mora da bude sprovedena u celom sistemu.

Posmatrajmo sada jedan jednostavan primer u kom se poverljivo ime kanala $\mathit{session}$ \v salje od jednog do drugog korisnika, kao \v sto je onaj naveden u odeljku Introduction. U ovom primeru, korisnik  $\mathit{Alice}$ kreira novi kanal i \v salje jedan njegov kraj korisniku $\mathit{Bob}$. Nakon sinhronizazije u kojoj se razmeni ime kanala, ova dva korisnika mogu napraviti privatnu sesiju na kanalu $\mathit{session}$.  
Me\dj utim, u na\v sem primeru $\mathit{Bob}$ odlu\v cuje da prosledi ime kanala $\mathit{session}$ nekom tre\' cem korisniku.

U nekim slu\v cajevima mo\v ze biti \v cak i po\v zeljno dati prava prosle\dj ivanja imena kanala nekim korisnicima. Na primer, zadaci mogu biti prosle\dj ivani od nadre\dj enog (eng. master) procesa do pot\v cinjenog (eng. slave) procesa, i tada pot\v cinjeni proces mo\v ze neprimetno da bude uklju\v cen u sesiju. U na\v sem primeru, ova situacija mo\v ze biti posmatrana kao problemati\v cna sa ta\v cke gledi\v sta korisnika $\mathit{Alice}$, jer ona i dalje veruje da drugi kraj kanala, koji ona smatra poverljivim, dr\v zi  $\mathit{Bob}$. 
Ako je 
 $\mathit{session}$ kanal koji je $\mathit{Alice}$ kreirala, %possibly 
kojim se mo\v ze pristupiti nekim njenim poverljivim podacima, i koji je poslat isklju\v civo korisniku $\mathit{Bob}$, 
mo\v zemo re\' ci da $\mathit{Bob}$ ne bi trebalo da stekne mogu\' cnost da ga dalje prosle\dj uje samo zato \v sto je u nekom trenutku primio ime kanala $\mathit{session}$. U svakom slu\v caju, mo\v zemo napraviti razliku izme\dj u ova dva slanja, jer prvo slanje je izveo korisnik koji je kreirao kanal ($\mathit{Alice}$), a u drugom je kanal zapravo prosle\dj en od strane u\v cesnika koji je primio kanal ($\mathit{Bob}$).

Nekoliko formalnih modela je do sada predlo\v zeno u svrhu opisivanja restrikovanog deljenja imena, kako bi se postiglo da ime mo\v ze biti razmenjeno samo u okviru unapred definisanog dela sistema. Takav je i model koji uvodi pojam grupe za imena~\cite{cardelli05} i model koji uvodi pojam skrivanja imena~\cite{Giunti}. 
Me\dj utim, u praksi imamo i slu\v cajeve u kojima ne postoji unapred definisan deo sistema u kom poverljiva informacija mo\v ze biti razmenjena. Na primer, u prethodnom primeru mo\v zemo re\' ci da $\mathit{Alice}$ mo\v ze u nekom trenutku sama da odlu\v ci da po\v salje ime kanala $\mathit{session}$ drugim u\v cenicima. Generalno, privatne informacije nekada moraju biti deljene i u otvorenim sistemima.




Drugi domen koji ova teza obra\dj uje je izu\v cavanje kontrole prava pristupa u distribuiranim softverskim sistemima. Za po\v cetak, mo\v zemo primetiti da kontrola prava pristupa ra\v cunarskim resursima postaje sve va\v znija, uprkos sve ve\' coj raspolo\v zivosti takvih resursa. Potreba za kontrolom pristupa mo\v ze biti motivisana mnogim faktorima, kao \v sto su privatnost, bezbednost i ipak postojanje nekog ograni\v cenja kapaciteta.
Primeri ograni\v cenog kapaciteta mogu biti direktno povezani sa fizi\v ckim ure\dj ajima, kao \v sto su \v stampa\v ci, mobilni telefoni i procesori, jer svi imaju fizi\v cki ograni\v cene mogu\' cnosti. Iako neki virtualni ure\dj aji, kao \v sto su deljena memorijska \' celija i web servis, imaju neograni\v cen potencijal, njihova dostupnost je \v cesto ograni\v cena.
 
Privatnost i bezbednost su neki od centralnih problema koji se pojavljuju kod razvoja distribuiranih sistema. Jedan od razloga je taj \v sto distribuirani sistemi postaju sve vi\v se heterogeni i kompleksni, a kontrola prava pristupa u takvim sistemima mo\v ze biti veoma te\v ska. Opravdanje za ovakve tvrdnje mo\v zemo na\' ci skoro svakodnevno, ve\' c pomenuti primer gre\v ske na Facebook-u je samo jedan u nizu. Takvi primeri su prouzrokovali milionske gubitke kompanija, ali jo\v s va\v znije, sigurnost i privatnost korisnika je u takvim situacijama bila izlo\v zena opasnosti. Formalni modeli i verifikacije mogu biti korak bli\v ze ka pouzdanijim distribuiranim softverskim sistemima~\cite{DBLP:journals/jlp/BugliesiCF17}.

%
Razli\v cite metode za kontrolu prava pristupa u distribuiranim softverskim sistemima razvijane su tokom godina. Njihov razvoj pratio je stalne promene u strukturi i veli\v cini sistema.
Za male sisteme, i za sisteme sa unapred definisanim brojem u\v cesnika, kontrola prava pristupa resursima obi\v cno se posti\v ze kori\v s\' cenjem lista za kontrolu pristupa (eng. access control lists - ACL). 
ACL metoda koristi liste sa pravima koje su dodeljene resursima. Pravo pristupa resursu mo\v ze biti odobreno samo korisniku koji je naveden kao subjekat sa odgovaraju\' cim pravom pristupa na listi datog resursa.

Iako ACL metod daje prirodan na\v cin za kontrolu prava pristupa, u velikim sistemima koji su dinami\v cni po pitanju broja i sastava u\v cesnika ovaj metod postaje te\v zak za implementaciju. Razlog za to je \v sto u ACL metodi svaka lista \v cuva podatke o svakom korisniku individualno, a to mo\v ze predstavljati veliki tro\v sak pri odr\v zavanju sistema. Na primer, posmatrajmo aplikaciju kao \v sto je Facebook, koju koristi preko milijardu korisnika. Imati liste korisnika koji mogu da pristupe resursima, kao \v sto su fotografije ili postovi svakog korisnika, mo\v ze postati neprakti\v cno. 

Upravljanje pristupom na osnovu uloga (eng. role-based access control method - RBAC)~\cite{sandhu1996role} 
je uvedeno kao alternativa ACL metodi.
RBAC metoda defini\v se skup uloga i svakom korisniku dodeljuje se jedna ili vi\v se uloga. Na primer, da bi sistem korisniku dozvolio ili odbio pristup fotografiji drugog korisnika na Facebook-u, ne mora se oslanjati na njegov identitet direktno. Prakti\v cnije re\v senje je proveriti da li korisnik koji poku\v sava da pristupi fotografiji ima ulogu ,,prijatelja'' sa vlasnikom fotografije.  
Pored svih prednosti (i mana) koje RBAC metoda ima u pore\dj enju sa ACL metodom, ona i dalje ima nedostatak da mora postojati centralni mehanizam za izdavanje i proveravanje uloga korisnika.


Upravljanje pristupom na osnovu klju\v ca (eng. capability-based method for access control)~\cite{zhao2013behavioural} je metoda koja je vi\v se prilago\dj ena decentralizovanim sistemima. U ovoj metodi, 
reference koje se ne mogu kopirati kreira i izdaje centralni mehanizam. Jednom izdata referenca ostaje kod korisnika i proverava se samo kada korisnik \v zeli da pristupi resursu. Dakle, u ovoj metodi centralni mehanizam ne mora da dr\v zi informacije o kontroli pristupa za svakog korisnika pojedina\v cno, dovoljno je da proverava validnost referenci (klju\v ceva) samo kada  je to potrebno. Tako\dj e, ove reference mogu biti delegirane izme\dj u dva u\v cesnika, bez potrebe da se o tome obavesti centralni mehanizam za kontolu pristupa. 
 
Jo\v s jedan domen koji obuhvata sli\v cne principe kao i poslednja navedena metoda za upravljanje pristupom je domen licenci: korisnik mo\v ze upotrebiti odre\dj enu aplikaciju samo pod uslovom da poseduje odgovaraju\' cu licencu. U ovom domenu tako\dj e mo\v zemo na\' ci pojam eksplicitne delegacije. Na primer, korisnik koji \v zeli da uposli aplikaciju na ra\v cunarskoj platformi dostupnoj putem interneta mo\v ze delegirati licencu za tu aplikaciju  koju ve\' c poseduje. Taj pojam poznat je pod nazivom Bring Your Own License~\cite{byol} (BYOL). 
Posebna vrsta licenci kao \v sto su licence za konkurentnu upotrebu (eng. concurrent use licenses) nudi dodatnu fleksibilnost kod kori\v s\' cenja~\cite{baratti2003license}. 
Kao primer, posmatrajmo jednu kompaniju koja koristi aplikaciju i koja poseduje odre\dj eni broj licenci potrebnih za kori\v s\' cenje te aplikacije. U slu\v caju licenci za konkurentnu upotrebu, licence mogu biti dostupne svim korisnicima u okviru domena date kompanije, ali broj licenci odre\dj uje gornju granicu za broj korisnika koji mogu koristiti aplikaciju u bilo kom trenutku~\cite{license_lp_comp}.

U ovoj tezi istra\v zujemo probleme formalnog, matemati\v cki zasnovanog, modeliranja i analize kontrolisanog kori\v s\' cenja i deljenja resursa u distribuiranim softverskim sistemima. Teza je organizovana u \v cetiri poglavlja.  

{\bf Prvo poglavlje} daje motivaciju za razvoj modela uvedenih u drugom i tre\' cem poglavlju teze.






{\bf Drugo poglavlje} ove teze daje jedan novi pristup za prou\v cavanje prvog problema koji smo do sada naveli: ograni\v cene diseminacije poverljivih informacija. U ovom poglavlju uvodimo formalni model koji ograni\v cava komunikacije koje se mogu okarakterisati kao prosle\dj ivanje. U tu svrhu predstavljen je ra\v cun nazvan \emph{Confidential $\pi$-calculus}, ili skra\' ceno $C_\pi$. Ovaj ra\v cun predstavlja jedan fragment \v cuvenog Milnerovog $\pi$-ra\v cuna~\cite{pi_calculus}, koji direktno u sintaksi onemogu\' cava prosle\dj ivanje primljenih imena. Jedini resursi u na\v sem modelu su imena kanala, i mi tretiramo imena kanala kao poverljive informacije. Glavna razlika u pore\dj enju sa originalnim $\pi$-ra\v cunom je ta \v sto u $C_\pi$-ra\v cunu jednom primljena imena kanala kasnije nije mogu\' ce poslati.
Ovo poglavlje teze se oslanja na publikovani rad
\begin{enumerate}
\bibitem{DBLP:journals/corr/abs-1902-0992712}
I.~Proki\'c.
\newblock The {C}pi-calculus: a model for confidential name passing.
\newblock In M.~Bartoletti, L.~Henrio, A.~Mavridou, and A.~Scalas, editors,
  {\em Proceedings 12th Interaction and Concurrency Experience, { ICE 2019}, {
  Copenhagen, Denmark, 20-21 June 2019}}, volume 304 of {\em Electronic
  Proceedings in Theoretical Computer Science}, pages 115--136. Open Publishing
  Association, 2019.
%\bibitem{DBLP:journals/corr/abs-1902-09927}
%I.~Proki{\'c}.
%\newblock The {$C_\pi$}-calculus: a model for confidential name passing.
%\newblock In {\em Interaction and Concurrency Experience, ICE 2019, Held as a
%  Satellite Workshop of the 14th International Federated Conference on
%  Distributed Computing Techniques, DisCoTec 2019, Copenhagen, Denmark, June
%  20-21, 2019}, EPTC, (to appear).
\end{enumerate}
ali ga dopunjava i pro\v siruje. Tako\dj e, ovde uvodimo novo pojednostavljeno kodiranje iz $\pi$-ra\v cuna u $C_\pi$-ra\v cun i predstavljamo kompletan dokaz operacione korespondencije za ovde uvedeno kodiranje.
Doprinosi ovog poglavlja u tezi su slede\' ci:
 \begin{itemize}
 \item  Uvo\dj enje novog, jednostavnog fragmenta $\pi$-ra\v cuna koji nam omogu\' cava da predstavimo komuniciranje poverljivih imena ograni\v cavanjem mogu\' cnosti prosle\dj ivanja imena. \v Cinjenica da je uvedeni model fragment uveliko izu\v cavanog $\pi$-ra\v cuna, daje nam mogu\' cnost da iskoristimo  ve\' c razvijene teorijske rezultate koji postoje za $\pi$-ra\v cun.
 \item Uvo\dj enje definicije osobine neprosle\dj ivanja i, kao provera dobre zasnovanosti, pokazivanje da svi procesi iz na\v seg $C_\pi$-ra\v cuna zadovoljavaju ovu osobinu.
 \item Koriste\' ci jaku bisimulaciju, bihevioralnu ekvivalenciju iz $\pi$-ra\v cuna, pokazan je jedan bihevioralni identitet koji potvr\dj uje da u na\v sem ra\v cunu mo\v zemo direktno predstaviti kreiranje zatvorenih domena za kanale.
 \item Data je detaljna diskusija o ekspresivnosti $C_\pi$-ra\v cuna na nekoliko pro\v sirenih primera, koji uklju\v cuju reprezentaciju kreiranja zatvorenih domena za kanale, autentikacije, zatvorenih i otvorenih grupa, od kojih svi mogu biti direktno predstavljeni u na\v sem modelu. 
 \item Uvedeno je novo kodiranje $\pi$-ra\v cuna u $C_\pi$-ra\v cun, \v cime je pokazano da je na\v s ra\v cun, iako predstavlja tek fragment $\pi$-ra\v cuna koji razmatra samo deo njegove sintakse, podjednako ekspresivan kao i $\pi$-ra\v cun. Tako\dj e, u ovom poglavlju dat je detaljan dokaz operacione korespondencije za uvedeno kodiranje.
 \end{itemize}


Centralni pojam svih formalnih modela za konkurentne i distribuirane sisteme je proces. Proces ozna\v cava entitet koji mo\v ze da komunicira sa drugim  takvim entitetima koriste\' ci zajedni\v cke komunikacione kanale. 
Neke od prvih i najvi\v se izu\v cavanih procesnih algebri su Milnerov ra\v cun komunikacionih sistema (eng. Calculus of Communicating Systems - $CCS$)~\cite{DBLP:books/sp/Milner80} i Hoareov ra\v cun komunikacionih sekvencijalnih procesa (eng. Communicating Sequential Processes - $CSP$)~\cite{DBLP:books/ph/Hoare85}. 
Napomenimo da je $CSP$ poslu\v zio kao osnovni model za programski jezik \emph{Go} koji je razvio Google. 
Za sveobuhvatniji pregled istorije razvoja procesnih algebri pogledati~\cite{DBLP:journals/tcs/Baeten05}.


Milnerov $CCS$-ra\v cun je jedan od prvih koji je formalno izu\v cavao konkurentne sisteme. Ovaj ra\v cun uvodi pojmove paralelne kompozicije, sinhronizacije slanja i primanja na istom imenu, kreiranja privatnih imena i sumacije (izbora). U ovoj tezi operator sumacije nije razmatran, ali verujemo da bi dodavanje ovog operatora moglo da se uradi na uobi\v cajen na\v cin. 
U $CCS$-ra\v cunu mo\v zemo definisati proces $\mathit{Alice} \parop \mathit{Bob}$ koji ozna\v cava dva konkurentna potprocesa $\mathit{Alice}$ i $\mathit{Bob}$, spojena operatorom paralelne kompozicije. Dva konkurentna procesa mogu da se sinhronizuju putem zajedni\v ckog kanala. Recimo, u procesu
\[
\overline{\mathit{chn}}.\mathit{Alice} \parop {\mathit{chn}}.\mathit{Bob}
\] 
proces na levoj strani $\overline{\mathit{chn}}.\mathit{Alice}$ mo\v ze da izvede akciju slanja na kanalu $\mathit{chn}$, dok proces na desnoj strani mo\v ze da izvede (dualnu) akciju primanja na istom kanalu. Nakon sinhronizacije po\v cetni proces se svodi na $\mathit{Alice} \parop \mathit{Bob}$.
U $CCS$-ra\v cunu procesi mogu i da kreiraju nova imena kanala, \v cime se modeluje stvaranje privatnih kanala koji nisu dostupni drugim procesima. U procesu 
\[
(\rest{\mathit{session}}\mathit{Alice}) \parop \mathit{Bob}
\]
ime kanala $\mathit{session}$ je poznato samo procesu $\mathit{Alice}$ i mo\v ze biti kori\v s\' ceno samo za sinhronizacije unutar tog procesa, dok proces $\mathit{Bob}$ nema nikakvu informaciju o postojanju tog kanala. Ono \v sto $CCS$-ra\v cun ne mo\v ze da predstavi direktno jeste mobilnost kanala. 

Tamo gde je Milner stao sa $CCS$-ra\v cunom, nastavio je sa $\pi$-ra\v cunom, koji pro\v siruje $CCS$ da bi dozvolio mobilnost komunikacionih kanala. U $\pi$-ra\v cunu procesi u toku sinhronizacije na kanalu mogu da razmenjuju imena kanala, time stvaraju\' ci nove konekcije me\dj u sobom. Nekoliko programskih jezika inspirisano je ovim modelom~\cite{ DBLP:conf/afp/FournetFMS02, DBLP:journals/entcs/MeredithR05,DBLP:conf/birthday/PierceT00,  DBLP:journals/jfp/SewellLWNAHV07, DBLP:conf/wecwis/ThiagarajanSPB02, DBLP:conf/birthday/WelchB04}. 

U $\pi$-ra\v cunu, mo\v zemo definisati proces
\[
\send{\mathit{chn}}\role\msg{\mathit{session}}.\mathit{Alice} \parop \receive{\mathit{chn}}\role\msg\NX.\mathit{Bob}
\]
gde na levoj strani paralelne kompozicije imamo proces koji je spreman da \v salje ime kanala $\mathit{session}$ putem kanala $\mathit{chn}$, dok na desnoj strani imamo proces koji je spreman da primi bilo koje ime kanala na kanalu $\mathit{chn}$, a zatim da ime $\NX$ (koje se jo\v s zove i ,,placeholder'') unutar procesa $\mathit{Bob}$ bude zamenjeno primljenim. 
Ovaj mehanizam daje novu dimenziju kada se kombinuje sa kreiranjem novih kanala, jer sada kreirani kanali mogu biti razmenjeni me\dj u procesima, \v cime se mogu stvarati privatne konekcije. 
Ovo je ujedno i poslednji sastojak koji nam je trebao da bismo u $\pi$-ra\v cunu modelovali primer sa prosle\dj ivanjem imena kanala koji smo ranije spominjali:
\[
(\rest{\mathit{session}}\send{\mathit{chn}}\role\msg{\mathit{session}}.\mathit{Alice}) \parop \receive{\mathit{chn}}\role\mgs{\NX}.\send{\mathit{forward}}\role\msg\NX.\mathit{Bob}'
\]
U ovom procesu kanal $\mathit{session}$ je poznat samo potprocesu sa leve strane paralelne kompozicije. Me\dj utim, nakon sinhronizacije sa potprocesom sa desne strane, po\v cetna konfiguracija evoluira u 
\[
\rest{\mathit{session}}(\mathit{Alice} \parop \send{\mathit{forward}}\role\msg{\mathit{session}}.\mathit{Bob}'')
\]
gde je ime privatnog kanala $\mathit{session}$ sada poznato i desnom potprocesu (to jest, $\mathit{Bob}''$ predstavlja proces koji se dobije od procesa  $\mathit{Bob}'$ kada sva pojavljivanja imena $\NX$ zamenimo imenom $\mathit{session}$). Dakle, u $\pi$-ra\v cunu domen privatnog imena (\v sto predstavlja deo sistema gde je ime poznato) se mo\v ze uve\' cati nakon sinhronizacije. Ako pretpostavimo da paralelno postoji i tre\' ci aktivni process  
\[
\rest{\mathit{session}}(\mathit{Alice} \parop \send{\mathit{forward}}\role\msg{\mathit{session}}.\mathit{Bob}'') \parop \receive{\mathit{forward}}\role\msg\NY.\mathit{Carol}
\]
onda proces koji obuhvata $\mathit{Bob}''$ mo\v ze sada proslediti ime kanala $\mathit{session}$ tom tre\' cem procesu putem kanala $\mathit{forward}$, a da o tome prethodno nije obavestio  $\mathit{Alice}$. Ova diseminacija imena mo\v ze dovesti do situacije u kojoj je privatnost $\mathit{Alice}$ kompromitovana. 

$C_\pi$-ra\v cun diskvalifikuje osobinu prosle\dj ivanja, te stoga $\receive{\mathit{chn}}\role\mgs{\NX}.\send{\mathit{forward}}\role\msg\NX.\mathit{Bob}'$ nije $C_\pi$ proces. Formalno, na\v s ra\v cun razlikuje imena kanala i imena promenljivih koje se pojavljuju u prefiksu primanja (eng. placeholder). Mi uvodimo dva disjunktna skupa imena, jedan ozna\v cen sa $\Chn$ koji \v cine imena kanala, i drugi ozna\v cen sa $\Var$ koji \v cine imena promenljivih. Ova distinkcija je iskori\v s\' cena kod definisanja jezika na\v seg modela, jedino imena iz skupa $\Chn$ mogu biti navedena kao imena za slanje u prefiksu koji defini\v se ovu akciju. 
Ovakvo sintaksno ograni\v cenje samo po sebi ne daje uop\v steno ograni\v cenje da se imena kanala, koja su posmatrana kao poverljiva informacija, ne mogu razmenjivati, niti ograni\v cava deo sistema u kom ime mo\v ze biti primljeno. Ono \v sto $C_\pi$ posti\v ze zapravo je lokalizacija dela sistema koji mo\v ze poslati ime kanala, a to je onaj deo gde je kanal prvobitno kreiran. Ukoliko je neophodno uspostaviti kontrolu nad slanjem imena nekog kanala, u $C_\pi$-ra\v cunu je dovoljno skoncentrisati se na deo sistema gde je kanal kreiran, dok bi, recimo, u $\pi$-ra\v cunu bilo neophodno kontrolu uspostaviti nad \v citavim delom sistema koji zna za dato ime. 

Ono \v sto je posledica specifi\v cnosti $C_\pi$-ra\v cuna je to da mo\v zemo razlikovati dva nivoa ovla\v s\' cenja koja proces mo\v ze imati u odnosu na neki kanal. Proces koji kreira kanal ima ovla\v s\' cenja da komunicira putem kanala, ali tako\dj e mo\v ze i da po\v salje ime kanala drugim procesima. Proces koji u nekom trenutku primi ime kanala sti\v ce pravo da komunicira putem tog kanala, ali ne i da dalje prosle\dj uje ime tog kanala. Prvi tip procesa u ovoj tezi je nazvan administrator kanala, a drugi korisnik kanala. Svaki administrator je istovremeno i korisnik, ali korisnik ne mora biti i administrator. 
Jo\v s jedna posledica lokalizacije dela sistema u kom se ime datog kanala mo\v ze poslati u tezi je iskori\v s\' cena i da poka\v ze kako $C_\pi$-ra\v cun mo\v ze biti iskori\v s\' cen za modelovanje autentikacije. Naime, sama mogu\' cnost slanja imena nekog kanala zapravo pripada samo administratoru kanala, a onom procesu koji prima to ime zapravo govori sa kojim procesom u tom trenutku komunicira (sa administratorom tog kanala).

Restrikcija koju $C_\pi$-ra\v cun pravi u odnosu na $\pi$-ra\v cun zapravo su\v stinski ne uti\v ce na ekspresivnu mo\' c, a to je i dokazano u samoj tezi. Sama ideja reprezentacije prosle\dj ivanja u $C_\pi$-ra\v cunu je izdvajanje procesa koji bi bili zadu\v zeni isklju\v civo za slanje odre\dj enog imena kanala. Drugi procesi bi, ukoliko \v zele da po\v salju neko ime kanala, zapravo umesto slanja samog imena prvo kontaktirali odgovaraju\' ci izdvojeni proces koji bi izvr\v sio slanje umesto njih. Ova ideja je u tezi formalizovana u kodiranju $\pi$-ra\v cuna u $C_\pi$-ra\v cun. 

%Both features add considerably to the expressive power of the $\pi$-calculus. 

Veliki broj teorijskih istra\v zivanja konkurentnih i distribuiranih sistema direktno je povezan sa $\pi$-ra\v cunom. Mnogi radovi koriste $\pi$-ra\v cun kao osnovni i pro\v siruju njegovu sintaksu kako bi stekli odgovaraju\' ci nivo apstrakcije da mo{\-}deluju poliadi\v cne komunikacije~\cite{DBLP:journals/njc/CarboneM03, DBLP:conf/concur/Milner92}, 
komunikacije vi\v seg reda~\cite{DBLP:conf/csl/Milner93}, 
distribuirane sisteme~\cite{DBLP:books/daglib/0018113}, sigurnost i privatnost~\cite{appliedpi,spi,cardelli05,pigroups,Giunti,hennessy05}, i mnoge druge aspekte, uklju\v cuju\' ci i kontrolu kori\v s\' cenja resursa koju razmatramo u tre\' cem poglavlju ove teze. 
Sa druge strane, deo istra\v ziva\v ca je koristio su\v zavanje sintakse $\pi$-ra\v cuna kako bi modelovali 
asinhrone komunikacije~\cite{ boudol:inria-00076939,DBLP:conf/ecoop/HondaT91}, unutra\v snju mobilnost~\cite{DBLP:journals/tcs/Sangiorgi96a}, i lokalizaciju~\cite{merro04}, a na\v s $C_\pi$-ra\v cun svakako spada u ovu kategoriju.




{\bf Tre\' ce poglavlje} predstavlja formalni model za izu\v cavanje kontrole pristupa resursima u distribuiranim softverskim sistemima. Ovaj ra\v cun u apstraktnom smislu modeluje ,,capabilities'' metodu za kontrolu pristupa, ali tako\dj e i licence za konkurentnu upotrebu, uvode\' ci pojam deljene autorizacije. Autorizacije se mogu definisati kao funkcije koje odre\dj uju prava i privilegije u odnosu na neki resurs. Kao i u drugom poglavlju, i ovde nam je fokus na sistemima kod kojih je komunikacija centralni pojam, tako da su jedini resursi koje ovde razmatramo zapravo imena komunikacionih kanala. Dakle, autorizacija defini\v se pravo da se koristi odre\dj eni komunikacioni kanal. Deljene autorizacije, koje posmatramo u ovom modelu, defini\v su prava da se kanal koristi konkurentno. Ovo zapravo zna\v ci da jedna autorizacija mo\v ze biti dostupna ve\' cem broju korisnika, ali da je u svakom trenutku mo\v ze koristiti najvi\v se jedan korisnik. 
U ovom modelu koristimo destilovane osobine deljenih autorizacija: domen, koji defini\v se deo sistema u kome je autorizacija implicitno dostupna; brojanje, koje defini\v se kapacitet; delegacija, koja defini\v se slanje i primanje samih autorizacija. 

Model predstavljen u tre\' cem poglavlju je zapravo ekstenzija $\pi$-ra\v cuna~\cite{pi_calculus}, koji se direktno oslanja na pretodno razvijeni ra\v cun sa autorizacijama~\cite{DBLP:journals/corr/GhilezanJPPV16, clar:eke}. Iz ra\v cuna sa autorizacijama~\cite{DBLP:journals/corr/GhilezanJPPV16} preuzeti su sintaksni konstrukti za domen i delegaciju autorizacija. U semanti\v ckom smislu, na\v s model modifikuje samo zna\v cenje domena autorizacije, kako bi dobili mogu\' cnost da obuhvatimo princip o brojanju autorizacija koji proisti\v ce iz prirode deljenih autorizacija izu\v cavanih ovde.
Tre\' ce poglavlje sistemati\v cno iznosi rezultate koji su prethodno predstavljeni u publikovanim radovima:
%
%
\begin{enumerate}
%
%
\bibitem{pr}
J.~Pantovi{\'c}, I.~Proki{\'c}, and H.~T. Vieira.
\newblock A calculus for modeling floating authorizations.
\newblock In C.~Baier and L.~Caires, editors, {\em Formal Techniques for
  Distributed Objects, Components, and Systems - 38th {IFIP} {WG} 6.1
  International Conference, {FORTE} 2018, Held as Part of the 13th
  International Federated Conference on Distributed Computing Techniques,
  DisCoTec 2018, Madrid, Spain, June 18-21, 2018, Proceedings}, volume 10854 of
  {\em Lecture Notes in Computer Science}, pages 101--120. Springer, 2018.
%
%
\bibitem{PRO}
I.~Proki{\'c}, J.~Pantovi{\'c}, and H.~T. Vieira.
\newblock A calculus for modeling floating authorizations.
\newblock {\em Journal of Logical and Algebraic Methods in Programming},
  107:136 -- 174, 2019.
%
%
%
\end{enumerate}
%
%
Glavni doprinosi ukupnog rada na modelu koji uvodi deljene autorizacije u $\pi$-ra\v cun u tre\' cem poglavlju ove teze su slede\' ci: 
\begin{itemize}
\item Definisanje novog formalnog ra\v cuna koji modeluje ve\' c spomenute pojmove domena, deljenih resursa, brojanja i delegacije, uvo\dj enjem pojma deljene autorizacije. 
%
\item Izu\v cavanje bihevioralne semantike ovog modela. Izvedena bihevioralna karakterizacija pokazuje specifi\v cnu prirodu deljenih autorizacija, naro\v cito  odnos izme\dj u konstrukta za domen autorizacije i konstrukta za paralelnu kompoziciju, koja reflektuje gore spomenuti princip brojanja. 
%
\item Uvo\dj enje tipskog sistema koji omogu\' cava izdvajanje procesa koji autorizovano koriste svoje kanale, \v cak i u prisustvu autorizacija koje su obezbe\dj ene od strane konteksta. Dokazivanje rezultata koji pokazuju da dobro tipiziran process ne samo da uvek koristi svoje kanale autorizovano, ve\' c to tako\dj e va\v zi i za sve njegove mogu\' ce evolucije. 
%
\item Pobolj\v sanje efikasnosti algoritma za proveru tipa uvo\dj enjem drugog tipskog sistema, za koji je pokazano rezultatom tipske korespondencije da je ekvivalentan sa prvim tipskim sistemom.
%
\item Prikazivanje pro\v sirenog primera inspirisanog pojmom \emph{Bring Your Own License} iz domena licenci, koji detaljnije opisuje uvedeni model i koji povezuje model sa njegovim mogu\' cim aplikacijama.
%
\item Na osnovu pomenutog pro\v sirenog primera pokazan je jedan konkretan pravac za primenu definisanog modela u programskim jezicima. Dat je primer koji posmatra jednu mogu\' cu ekstenziju programskog jezika Go\footnote{\url{https://golang.org}}.
\end{itemize}

Model sa deljenim autorizacijama uspostavlja dodatni nivo kontrole kori\v s\' cenja kanala u odnosu na $\pi$-ra\v cun. U na\v sem modelu, nije dovoljno da proces ima pristup kanalu, ve\' c dodatno mora imati i autorizaciju za kori\v s\' cenje tog kanala. Sama sintaksa $\pi$-ra\v cuna pro\v sirena je sa konstruktima za autorizacije i njihovo delegiranje me\dj u procesima. Na primer, proces
\[
    { \scope{\mathit{license}} (\mathit{Alice} \parop \mathit{Bob})}
\]
defini\v se da je jedna autorizacija za kori\v s\' cenje kanala $\mathit{license}$ dostupna procesima $\mathit{Alice}$ i $\mathit{Bob}$. Ako, recimo, proces $\mathit{Bob}$ prvi zapo\v cne komunikaciju na kanalu $\mathit{license}$ onda konfiguracija data gore postaje 
$$
 {\mathit{Alice} \parop \scope{\mathit{license}}\mathit{LicensedBob}}
$$
gde autorizacija $\mathit{license}$ vi\v se nije dostupna za $\mathit{Alice}$. Autorizacije mogu biti razmenjene u komunikaciji. Na primer, u
$$\scope{\mathit{license}} \scope{\mathit{auth}}\sauth{\mathit{auth}}\role\msg{\mathit{license}}.\mathit{UnlicensedBob} \parop \scope{\mathit{auth}}\rauth{\mathit{auth}}\role\msg{\mathit{license}}.\mathit{LicensedCarol}$$
proces na levoj strani ima autorizaciju da koristi kanale $\mathit{auth}$ i $\mathit{license}$, a prefiks defini\v se akciju slanja autorizacije za $\mathit{license}$ putem kanala $\mathit{auth}$. Sa desne strane, proces mo\v ze da primi autorizaciju za $\mathit{license}$ na kanalu $\mathit{auth}$, i za tu akciju ima odgovaraju\' cu autorizaciju. Nakon sinhronizacije dva procesa, dobijemo 
$$ 
\scope{\mathit{auth}}\mathit{UnlicensedBob} \parop \scope{\mathit{auth}}\scope{\mathit{license}}\mathit{LicensedCarol}
$$
gde autorizacija za $\mathit{license}$ prelazi sa leve na desnu stranu. 

Kao \v sto smo videli u prethodnom primeru, autorizacije zapravo omogu\' cavaju (ili u nedostatku istih, onemogu\' cavaju) komunikacije na kanalima. Ovo va\v zi ne samo za komunikacije u kojima se razmenjuju autorizacije, ve\' c i za komunikacije u kojima se razmenjuju imena kanala. Na primer, 
$$\scope{\mathit{comm}}\send{\mathit{comm}}\role\msg{\mathit{license}}.{\mathit{Alice}}
\parop \scope{\mathit{comm}}\receive{\mathit{comm}}\role\msg{\mathit{x}}.{\mathit{Dylan}}$$
predstavlja proces u kome ime kanala $\mathit{license}$ poslato na kanalu $\mathit{comm}$ od potprocesa sa leve strane, mo\v ze biti primljeno u potprocesu sa desne strane, jer za obe akcije postoje odgovaraju\' ce autorizacije. Sa druge strane, sinhronizacija u procesu 
$$\scope{\mathit{comm}}(\send{\mathit{comm}}\role\msg{\mathit{license}}.{\mathit{Alice}}
\parop \receive{\mathit{comm}}\role\msg{\mathit{x}}.{\mathit{Dylan}})$$ 
nije mogu\' ca jer za akcije slanja i primanja postoji samo jedna autorizacija, dok su potrebne dve. U ovoj tezi, ovakvi procesi, koji ne mogu da sinhronizuju svoje dualne akcije zbog nedostatka odgovoraju\' cih autorizacija nazivaju se gre\v skama.  

Da bi izdvojili procese koji nisu gre\v ske i koji ni u jednoj od mogu\' cih evolucija ne postaju gre\v ske, u tezi je predstavljen tipski sistem. Tipski sistem se sastoji od dodele tipova imenima i tipskih pravila, koja defini\v su uslove koje proces koji se proverava mora zadovoljiti. Ukoliko proces mo\v ze da pro\dj e definisanu tipsku proveru onda je on ,,bezbedan'', to jest, nije gre\v ska i ne svodi se na gre\v sku. Tipovi koje mi ovde dodeljujemo imenima zapravo govore o imenima koja mogu biti bezbedno komunicirana na kanalima. 
Na primer, posmatrajmo process
$$
\scope{\mathit{exam}}\scope{\mathit{minitest}}
\scope{\mathit{alice}}\receive{\mathit{alice}}\role\msg{x}.
\send{x}\role\msg{\mathit{value}}. \inact
%\parop 
%\scope{\mathit{bob}}\receive{\mathit{bob}}\role\msg{x}.
%\receive{x}\role\msg{\mathtt{Task}}. \mathit{DoTask})
$$
koji mo\v ze da primi ime kanala i da zatim po\v salje  $\mathit{value}$ na primljenom kanalu. Primanje na $\mathit{alice}$ je autorizovano direktno jer je odgovaraju\' ca autorizacija prisutna. Sa druge strane, kasnije slanje je autorizovano samo za imena  $\mathit{exam}$ i $\mathit{minitest}$. Ako mo\v zemo da osiguramo da na kanalu $\mathit{alice}$ samo imena $\mathit{exam}$ i $\mathit{minitest}$ mogu biti komunicirana, tada je ovaj process bezbedan.
Stoga, imenu  $\mathit{alice}$ dodeljujemo tip 
 $ \{\mathit{alice}\}( \{\mathit{exam},\mathit{minitest}\} ( \emptyset ))$, i to obele\v zavamo sa 
 $$\mathit{alice}:\{\mathit{alice}\}( \{\mathit{exam},\mathit{minitest}\} ( \emptyset ))$$ 
 kako bismo ozna\v cili  da je $\mathit{alice}$ finalno ime (uporediti sa tipom od $x$ datim dole), i da 
kanal mo\v ze biti kori\v s\' cen isklju\v civo za komuniciranje imena $\mathit{exam}$ i $\mathit{minitest}$.  Poslednja informacija u tipu govori da $\mathit{exam}$  i $\mathit{minitest}$
ne mogu biti kori\v s\' ceni za komunikacije (ozna\v ceno sa $\emptyset$). 
Sa druge strane, tip koji bismo dodelili promenljivoj u ovom procesu je $x:\{\mathit{exam},\mathit{minitest}\} ( \emptyset )$, jer $x$ mo\v ze biti zamenjeno imenima $\mathit{exam}$ ili $\mathit{minitest}$, za koje onda treba obezbediti autorizacije. Dakle, za samo ime $\NX$ autorizacija nije prisutna u procesu u kom se nalazi prefiks, ali autorizacije za dve mogu\' ce zamene imena jesu, zbog \v cega ovakve indirektne autorizacije zovemo kontekstualne autorizacije. Tipski sistem predstavljen u ovoj tezi tretira i direktne i kontekstualne autorizacije. Tipske pretpostavke skupljaju se u tipsko okru\v zenje, obi\v cno obele\v zeno sa $\Delta$, i u odnosu na takvo oku\v zenje vr\v si se provera procesa pomo\' cu pravila koja se defini\v su za svaki sintati\v cki konstrukt pojedina\v cno. Tipsko tvr\dj enje, koje je oblika $\Delta\vdash_\rho \PP$, govori da proces $\PP$ koristi svoje kanale kako je to propisano u tipskom okru\v zenju $\Delta$ i da proces poseduje dovoljno autorizacija ako bi bio sme\v sten u kontekst koji bi mu obezbedio dodatne autorizacije navedene u multiskupu $\rho$. Ove ideje su tako\dj e formalizovane u tre\' cem poglavlju teze. 


{\bf \v Cetvrto poglavlje} sadr\v zi sa\v zetak postignutih rezultata kandidata, pregled literature i razmatra pravce daljih istra\v zivanja.\\
\newline
\textbf{Klju\v c re\v ci:} distributed systms, cloud computing, micro clouds, edge computing
