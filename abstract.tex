%!TEX root =  main.tex
\chapter*{Abstract}
%\pagestyle{empty}
Edge computing brings cloud services closer to the edge of the network, where data originates and dramatically reduces network latency of cloud. It is a bridge linking clouds and the users making the foundation for novel interconnected applications.

However, edge computing still faces many challenges like remote configuration, well defined native applications model, and limited node capacity. It lacks geo-organization and clear separation of concerns. As such edge computing is hard to be offered as a service, for future real-time user-centric applications. 

This thesis presents the dynamic organization of geo-distributed edge nodes into micro data-centers and forming micro-clouds to cover any arbitrary area and expand capacity, availability, and reliability. We use a cloud organization as an influence with adaptations for a different environment, and we present a model for edge applications utilizing these adaptations. 

We argue that the presented model can be integrated into existing solutions or used as a base for the development of future systems. Furthermore, we give a clear separation of concerns for the proposed model. With the separation of concerns setup, edge-native applications model, and a unified node organization, we are moving towards the idea of edge computing as a service, like any other utility in cloud computing. 

The first chapter give introduction to area of distributed systems, narrowing it down to only parts that are important for further understanding of the other chapters and the rest of the thesis in general.

The second chapter, show related work from different areas that are connected or that influenced this thesis. This chapter as well show what is the current state of the art in industry and academia, but also describe positoin of this thesis compred to the related reserach.

The third chapter, propose model that is influenced with cloud computing architectural organizations, but adopted for different environemtn. We presetn how we can separate geographic area into micro data-centers that are zonaly organzied to serve local population and, and forme them dinamicaly. This chapter also give formal models for all protocols used for creation of such a system with separation of concerns, applications models and present limitations of this thesis.

The forth chapter present implemented framweork that is based on model described in chapter three. We describe architecture and in detail every operation framework is able to do with limitations of the framework.

The fith and the last chapter conclud this thesis, and present future work that should be done.\\ 

\noindent
\textbf{Key words:} distributed systems, cloud computing, micro clouds, edge computing, platform, as a service model.