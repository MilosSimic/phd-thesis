%!TEX root =  main.tex
\chapter*{Abstract}
%\pagestyle{empty}
Cloud computing is facing some serious latency issues due to huge volumes of data that need to be transferred from the place where data is generated to the cloud. For some types of applications, this is not acceptable. 

One of the possible solutions to this problem is the idea to brings cloud services closer to the edge of the network, where data originates. This idea is called edge computing, and it is advertised that a dramatically reduces the network latency as a bridge that links the users and the clouds, and as such it makes the foundation for future interconnected applications.

Edge computing is a relatively new area of research
and still faces many challenges like geo-organization and a clear separation of concerns, but also remote configuration, well defined native applications model, and limited node capacity. Because of these issues, edge computing is hard to be offered as a service, for future real-time user-centric applications. 

This thesis presents the dynamic organization of geo-distributed edge nodes into micro data-centers and forming micro-clouds to cover any arbitrary area and expand capacity, availability, and reliability. We use a cloud organization as an influence with adaptations for a different environment with a clear separation of concerns, and native applications model that can leverage the newly formed system.

We argue that the presented model can be integrated into existing solutions or used as a base for the development of future systems. Furthermore, we give a clear separation of concerns for the proposed model. With the separation of concerns setup, edge-native applications model, and a unified node organization, we are moving towards the idea of edge computing as a service, like any other utility in cloud computing. 

The first chapter of this thesis, gives an introduction to the area of distributed systems, narrowing it down to only parts that are important for further understanding of the other chapters and the rest of the thesis in general.

The second chapter, show related work from different areas that are connected or that influenced this thesis. This chapter as well shows what is the current state of the art in industry and academia, but also describes the position of this thesis compared to the related research.

The third chapter proposes a model that is influenced by cloud computing architectural organizations but adopted for a different environment. We present how we can separate the geographic area into micro data-centers that are zonally organized to serve the local population and, and form them dynamically. This chapter also gives formal models for all protocols used for the creation of such a system with separation of concerns, applications models, and present limitations of this thesis.

The fourth chapter presents an implemented framework that is based on the model described in chapter three. We describe the architecture and in detail, every operation framework can do with limitations of the framework.

The fifth and the last chapter concludes this thesis and presents future work that should be done.\\\\ 

\noindent
\textbf{Key words:} distributed systems, cloud computing, micro clouds, edge computing, platform, as a service model.