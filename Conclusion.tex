%!TEX root =  main.tex
\chapter{Conclusion}\label{chapter:Conclusion}
%
This chapter gives the summary of contributions for this thesis in Section~\ref{sec:summary_of_contributions}. Section~\ref{sec:limitations} presents limitations of our model, while Section~\ref{sec:future_work} presents future work.
%
%
\section{Contributions of the thesis}\label{sec:summary_of_contributions}
%
This thesis presents a possible solution to how to organize geo-distributed EC nodes into $\upmu$Cs that will be able to serve the requests of the nearby population. We have introduced an extension of $\upmu$DCs based on proven abstractions from cloud computing like zones and regions but adapted for different usage scenarios.

These easy to understand, yet powerful abstractions with slight adaptations, allowed us to cover any arbitrary vast geographic area, and yield a more available and reliable system forming the $\upmu$C model. These abstractions are easy to organize and reorganize, and $\upmu$C size is determined by the population needs descriptively. 

We have also presented the cloud to ECC mapping, showing differences between two architecture models. Furthermore, we have given a formal model of the system and its protocols used to form such a system, with clear SoC and native application model for future $\upmu$Cs infrastructure and service development. The thesis also presents a proof of concept implementation and discusses integration into existing solutions, limitation of such a system with a few applications that could be used in.

Chapter~\ref{chapter:Intro} presents the motivation for this thesis with problem are, hypothesis combined with goals and research questions this thesis is built on.

A short introduction to the topic of distributed system is given in Chapter~\ref{chapter:Field_overview}, with a short introduction to the topic of distributed systems, with a focus only on the areas that are important for the understanding of this thesis. 

It is shown what distributed systems are, or, at least consensus how some systems can be described as distributed. We present problems these systems create, and why they are so hard to implement and maintain. 

It also presents a few distributed computing applications, that we can use to employ nodes in the distributed system. Further on it is shown what scalability is and why it is important for distributed systems, with few organizational ideas like peer-to-peer and membership that protocol that is important in a distributed environment for various reasons.

Virtualization techniques are then described that can be used to pack and deploy applications and infrastructure, how to implement various deployment techniques especially in the CC environment, but also the difference between DS and few models that are usually confused as distributed like parallel computing and decentralized computing.

Chapter~\ref{chapter:Review} shows related work done by various other researchers or, companies, focused again only on things that are related to this thesis.

We show different platform options, where people change the existing solutions like Kubernetes or OpenStack to made them work in areas like edge computing and mobile computing. Implementations of a few new platforms to use volunteer nodes to do some computation and storage on them with drop computing and systems like Nebula are shown further.

It is show how nodes can be organized to split some geographic area into zones, and show how $\upmu$DCs can help CC to serve requests from the local population. Different offloading techniques are used today, how to offload tasks from mobile devices closer to fog or edge nodes, but also various application models that could harness these offloading techniques and nodes organizations.

In the end, the position of this thesis, compared to other similar models, is presented.

Chapter~\ref{chapter:Micro_clouds} presents the heart and soul of this thesis. We dissect all important aspects that we need to have to help CC with latency issues, Big Data with huge volumes of data especially in the age of mobile devices and IoT.

Our model is based around $\upmu$DCs that are zonally organized, that will serve local population and population nearby. We present a model that is based on CC but adapted for different scenarios and use cases.

We show how we can dynamically form new clusters, regions, and topologies and how we can use them in the new age of mobile devices and IoT. These newly formed systems or system od systems will have clear the SoC, adopted from existing research to three-tier architecture. The formed model will serve as a pre-processing layer, firewall, or privacy layers, and it is adjustable in various dimensions.

The presented model can be as huge as the whole state, as small as a single household and all in between. This is a matter of agreement and a matter of choice. We present how developers can use this new infrastructure and what possible models of applications could exist, and how operations can deploy developed services onto existing infrastructure.

In the end, we present the repercussions of this model, and how it can be used as an integral part of existing systems to serve as topology storage or as a new model on top of which new subsystems and applications can be developed. We present protocols for the creation of such a system and model them using asynchronous session types. The system follows a formal model, and it is easy to extend.

Finally, we give limitations of such a system and things we must be aware of, \textbf{if} such technology is going to be used in real-life scenarios.

Chapter~\ref{chapter:Implementation} presents an implemented framework that is based on knowledge compiled from previous chapters. We also present in detail operations that could be done in the framework, where it fits in the presented SoC model.

It further presents the results of our experiments in a controlled environment, what, the limitations of the framework at the current stage, and possible applications, and where this model could be used and beneficial.

The thesis is concluded in this last chapter with what  is done, what can be done in the future in form of future work.
%
%
\section{Limitations}\label{sec:limitations}
%
The model proposed in this thesis has some llimitations that we must be aware of, either to work on improvements or use the model as is. When talking about small-scale servers and $\upmu$Cs, we must be aware of a few things.

\begin{enumerate}[start=1,label={(\bfseries \arabic*)}]
	\item We must be aware that not all organizations will be able to deploy $\upmu$Cs, due to the high initial investments required~\cite{MonsalveCC18}. We can rely on government authorities, large cloud providers, or other big companies to build the initial infrastructure for their own needs, and lease it to others similar to the cloud. The general public can use them, similarly to the cloud -- pay as you go, model.
	\item There is no guarantee that existing public cloud providers will allow nodes that are not built, resigned, or deployed by them. If we are building a private cloud, then  we can make a different decision. One way to resolve this issue is that the whole platform becomes open-source so that public cloud providers can engage in the development, and eventually use them as a solution.
	\item These small-scale servers must be out of reach of people and protected in some way so that not everyone has access to them. Some degree of physical security must be implemented.
	\item The places where these small scale servers will be deployed must have a stable internet connection, and the ability to integrate SDN or other similar technologies, so that complex network topologies could be implemented properly.
	\item These servers can have some open architecture or could be custom built by other providers. In both cases, they must be able to satisfy rules that are presented in~\ref{sec:separation_of_concerns}.
	\item Splitting the processing into two parts and the possibility that users can be responsible for $\upmu$Cs may raise some legal concerns. Either to develop interesting applications, use them as a firewall or simply use them as a privacy level for data, there must be a legal agreement that might not be that easy to achieve.
\end{enumerate}
%
%
\subsection{Discussion}
%
The specialized infrastructure in models like~\cite{BaccarelliNSSA17, GuoRG20, JeonK19, BCAK19, ChiariniRAMG13} is required to solve a single problem. The model proposed in this thesis is more oriented towards a wider specter of applications, without the need for specialized hardware or software. Users should build applications, similarly as they build them for the cloud. Even existing applications models (e.g., microservices, serverless functions) could be transferred from cloud to $\upmu$C.

One advantage is that specialized models are developed and optimized for a specific use case taking the maximum out of existing hardware and software. Compared to the presented model (in terms of speed), in some situations, they might outperform the proposed model. On the other hand, the proposed model offers much more freedom for development (in terms of agility and applicability). This development freedom gives the users a new platform for creating interesting human-centered applications spanning over CC and $\upmu$Cs~\cite{VillariCF17}.

Specialized platforms usually require special types of applications, while the proposed model does not limit users (in terms of development tools and techniques), as long as their application could be virtualized in some way (e.g., using virtual machines, containers, or unikernels). The developers may reuse existing knowledge to develop their applications.

The proposed model allows the organization of storage and processing resources according to priority if a catastrophic event (e.g., COVID-19 see section~\ref{sec:covid_example}) hits the human population. Humanity can organize its resources and manage its digital infrastructure, where developers make applications that will help its citizens in those tough times.
%
%
\section{Future work}\label{sec:future_work}
%
The work on the $\upmu$Cs is at an early stage and leaves many open questions. As a part of our current and future work, we are planning to extend the proposed model in different directions. Future work might be separated into three options:
 
\begin{enumerate}[start=1,label={(\bfseries \arabic*)}]
	\item features that operations people (eg. DevOps and SREs) users can benefit from;
	\item features developers might benefit from;
	\item infrastructure features, that both previous groups can benefit from;
\end{enumerate}

\noindent
For the first group, the first thing that should be implemented is remote cluster management, using configurations, security credentials, and actions over nodes in one or multiple clusters. On formed clusters, the users should be able to do remote configurations that nodes and/or applications can use and set up the data without going from node to node.

The system should also be extended with namespaces for usage in environments with many users in multiple teams -- multi-tenant environments. Namespaces provide the separation on virtual clusters, running on the same physical hardware. Speaking about multi-tenancy, we are also planning to implement role-based access control integrate with authentification and authorization services, with the addition of controlling different users with quotas, using rate limiting and resource limiting.

Also one possible approach would be to extend resources dynamically using infrastructure auto-scaling mechanism, so that system can grow and shrink as needed.

We are also planning to implement a full architecture and applications monitoring, alerting, and reporting that would be helpful to any administrator of such a system. We might also consider rethinking networking and making network isolation so that once formed topologies can communicate within themselves, and a possibility to specify 
strategies of communication with other topologies.

Queueing system mentioned in section~\ref{sec:queueing} should be extended so that users and/or operations people can easily add new queues and possibly assign a role for them.

We should extend the access pattern so that users can issue commands to $\upmu$Cs directly instead of going over the cloud master process only. Here we need to implement synchronization in multi-cloud deployment.

For the last part in this operations section, we should also think about continuous integration, deployment, and delivery of services onto the infrastructure, and as well as various UI dashboards that can be customized to present different aspects of the system.

Another direction for future work is the implementation which developers could benefit from. The first thing that should be implemented here is the complete application framework so that users can start developing services that can do something. We should also implement a framework and maybe domain-specific language for the use case where users just want to pre-process the data before sending it to the cloud, in a more convenient way than writing the whole application.

Users can develop their applications with different models: 

\begin{enumerate}[start=1,label={(\bfseries \arabic*)}]
	\item \textbf{mPaaS}, where the platform is doing all the management and offers a simple interface for developers to deploy their applications;
	\item \textbf{mCaaS}, if users require more control over resources requirements, deployment and orchestration decisions;
	\item \textbf{mSaaS}, users can develop their solutions only using $\upmu$Cs, but this is not advised at the moment;
\end{enumerate}

\noindent
The second would be file system and database APIs that users can use to store their data. We should also provide an interface for extensions so that others can create their databases following different models from which developers can benefit, but also integrating existing ones.

The last part of the future work be extending the current system with tunable replicating strategies for the data, in case that any part of the topology fails for whatever reason, data would not be lost. Furthermore, we should provide tunable CC synchronization models that could be used.

We should implement a scheduling system for user-developed applications so that we can put applications into the formed architecture. And last but not least, we are planning to add several security layers to protect a system in general from malicious users.

We should investigate compression methods to reduce data stored and sent via the network. These tests should be conducted on ARM devices with existing methods, or maybe we can create a ground for new compression methods and techniques.

It is stated in section~\ref{sec:repercussion} of this thesis that this model could be integrated into existing solutions. Our efforts should go as well on integrating this system with existing solutions so that they can benefit from this hierarchical and geo-distributed nodes organization in the same way or almost the same way as the stand-alone solution would.
%
%